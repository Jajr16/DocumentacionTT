% !TeX root = ../ejemplo.tex

\section{Evaluaciones a Título de Suficiencia (ETS)}

\subsection{Definición y contexto}
En el Instituto Politécnico Nacional (IPN), incluyendo unidades académicas como la Escuela Superior de Cómputo (ESCOM), la acreditación de cada unidad de aprendizaje se realiza semestralmente a través de 3 evaluaciones ordinarias. Si un alumno no acredita alguna unidad de aprendizaje, tendrá la oportunidad de presentar una evaluación extraordinaria. Estos procedimientos están detallados en el programa de estudios y se encuentran especificados en el calendario académico.

El alumno que no logre acreditar una o más de las unidades de aprendizaje en la que se haya inscrito podrá optar por acreditarlas mediante Evaluación a Título de Suficiencia, de acuerdo con lo establecido en el Artículo 39 del Reglamento Interno del IPN, que señala: 

\begin{quote}
	``La evaluación del aprendizaje se llevará a cabo a través de exámenes ordinarios, extraordinarios y a título de suficiencia, cuyos requisitos y procedimientos de elaboración, presentación y exención, así como de otros mecanismos de evaluación continua, se realizarán en los términos que fijen los planes y programas de estudio, el presente Reglamento y los reglamentos respectivos.''
\end{quote}

Existen dos rondas de ETS:
\begin{itemize}
	\item ETS Ordinario: Esta es la primera oportunidad que tiene el alumno para acreditar la materia en la que no obtuvo una calificación aprobatoria. Los ETS ordinarios generalmente se aplican al finalizar el semestre, permitiendo al estudiante demostrar sus conocimientos sin necesidad de repetir el curso completo. 
	\item ETS Especiales: Si el alumno no acredita la materia en el ETS ordinario, puede optar por presentar un ETS Especial. Esta es la segunda oportunidad que tiene el alumno para pasar la materia. Esta evaluación adicional suele programarse el primer viernes del nuevo semestre, brindando al estudiante una opción rápida para regularizar sus situación académica y continuar avanzando en su plan de estudios. 
\end{itemize}

\subsection{Procedimiento para realizar un ETS}
\begin{itemize}
	\item Pagar en caja, verificar que estén correctos los siguientes datos: Nombre, Boleta, Carrera y Número de unidades de aprendizaje.
	\item Acudir a ventanilla de gestión escolar para generar créditos en el “SAES”.
	\item Una vez generados los créditos, inscribe las unidades de aprendizaje en la página del “SAES”.
	\item Entregar en ventanilla de gestión escolar, el comprobante de inscripción de ES generador por SAES, y el recibo de pago para dar fin a la inscripción al ETS. 
	\item  Acudir el día y la hora establecida en el calendario.
\end{itemize}

\subsection{Problemáticas identificadas}
\begin{itemize}
	\item Suplantación de identidad: La verificación manual de credenciales es vulnerable a fraudes, ya que no existe un sistema automatizado que valide la autenticidad del estudiante.
	\item El personal de seguridad no cuenta con herramientas para verificar si un alumno está inscrito en un ETS el día de su aplicación.
\end{itemize}

Estas limitaciones reflejan la importancia de implementar un sistema integral que incorpore tecnologías como el reconocimiento facial y los códigos QR, con el propósito de asegurar procesos más seguros, eficientes y transparentes durante la aplicación de los ETS.