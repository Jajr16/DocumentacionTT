% !TeX root = ../ejemplo.tex

\section{Spring Boot}
Spring Boot es una herramienta que sirve para desarrollar tanto aplicaciones web, como microservicios en Java. Este se basa en Spring Framework, el cual es un framework para el desarrollo de aplicaciones y contenedores de inversión de control de código abierto, igualmente para Java.
Spring Boot permite el desarrollo de aplicaciones web y microservicios de manera rápida y fácil gracias a 3 características que tiene, las cuales son:
\begin{itemize}
    \item \textbf{Configuración automática:} Con esto se refiere a que Spring Boot lo que hace es analizar las dependencias incluidas en un proyecto, como lo son bases de datos, servidores, entre otras cosas y decide qué configuraciones aplicar. Esto se basa en anotaciones que usualmente empiezan con un "@", gracias a esto, podemos desarrollar aplicaciones de manera más rápida y eficiente, pues nos ahorra todo el trabajo de realizar método por método. \cite{CitaAJ01}
    \item \textbf{Enfoque obstinado de la configuración:} Con base en las mejores prácticas de la programación, Spring Boot toma ciertas decisiones de configuración que funcionan bien para la mayoría de casos, esto se refiere a configuraciones de puertos, rutas, entre otras cosas. Además, en caso de que nosotros queramos agregar configuraciones adicionales, o bien, cambiar la configuración de algo de nuestro proyecto, nos va a crear un archivo de configuración en donde podremos especificarle de qué manera lo queremos. \cite{CitaAJ02}
    \item \textbf{Capacidad de crear aplicaciones independientes:} Nos permite crear aplicaciones que se ejecutan de forma independiente, es decir, sin necesidad de un servidor web externo. Esto lo logra gracias a que en el proceso de inicialización, se integra un servidor web en la aplicación, o que le permite funcionar de manera autónoma. \cite{CitaAJ01}
\end{itemize}

\section{Java Persistence API (JPA)}
Antes de irnos a la definición de JPA, hay que entender qué es la persistencia de los datos. La persistencia de los datos es un medio mediante el cual una aplicación puede recuperar información desde un sistema de almacenamiento y hacer que esta persista. Ahora bien, JPA lo que hace es proporcionarnos varias funciones para poder gestionar la persistencia y la correlación de objetos, es decir, lo que hace es proveernos con una serie de interfaces que podemos utilizar para implementar la capa de persistencia de nuestra aplicación. \cite{CitaAJ03}

Algunas de las ventajas que nos presenta JPA es que nos permite hacer el mapeo de entidades, es decir, el definir cómo se relacionan las clases de nuestra aplicación con los elementos de nuestra base de \cite{CitaAJ03}. Esto abarca temas como las relaciones entre clases y tablas en nuestra abse de datos, propiedades de las clases y los campos de las tablas e incluso la relación entre diferentes clases y las claves exrternas de nuestras tablas. 

Además, JPA ofrece la capacidad de interactuar con diferentes sistemas de gestión de bases de datos, ya que no usa sentencias SQL o de algún tipo de base de datos en específico, por lo que mejora la portabilidad y escalabilidad de las aplicaciones. \cite{CitaAJ04}