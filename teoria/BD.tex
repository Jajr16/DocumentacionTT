% !TeX root = ../ejemplo.tex

\section{Bases de datos (BD)}
Para que todo sistema pueda funcionar, debe de tener una forma de almacenar toda su información, ya sea información de los usuarios o para almacenar imágenes, cualquier tipo de información que se requiera almacenar, es necesario usar una base de datos. Pero en sí, ¿qué es una base de datos?.
Una base de datos, según Microsoft \cite{CitaAJ08} , es una herramienta para recopilar y organizar información. Estas pueden almacenar información sobre personas, productos, pedidos u otras cosas. Las bases de datos pueden empezar desde un simple documento de textos o archivos Excel, pero conforme más va creciendo esta base de datos empiezan a aparecer redundancias, por lo que es mejor optar por usar una base de datos creada por un sistema gestor de bases de datos. 
Un sistema gestor de bases de datos es un software constituido por una serie de programas dirigidos a crear, gestionar y administrar la información que se encuentra en una base de datos. El principal objetivo de estos sistemas es servir de interfaz entre los usuarios y las aplicaciones para facilitar la organización de los datos \cite{CitaAJ06}.
Por otro lado, dentro del mundo de las bases de datos, existen diferentes tipos de bases de datos, y cada tipo de base de datos tiene una organización diferente y propósito diferente. A continuación, les hablaremos de ciertos tipos de bases de datos:

\subsection{Bases de datos relacionales:}
Este tipo de bases de datos es el más usado en sistemas que necesitan almacenar una gran cantidad de información que está relacionada entre sí. Ahora bien, una base de datos relacional, según Google \cite{CitaAJ09}, es una forma de estructurar información en tablas, filas y columnas. La ventaja de este tipo de bases de datos es que tiene la capacidad de establecer relaciones entre la información mediante tablas, esto nos ayuda a visualizar mejor la información sobre la relación entre los datos.
Las bases de datos relacionales se componen de una serie de conceptos clave que necesitamos desarrollar para mejorar la comprensibilidad acerca de este tipo de bases de datos:

\begin{itemize}
    \item \textbf{Tablas:} Las tablas, en una base de datos relacional, son objetos que contienen todos los datos de dicho objeto. Como se mencionó anteriormente, estas tablas se organizan en filas y columnas. Aquí, cada fila representa un registro único y cada columna un campo dentro del registro. La manera en que este tipo de tablas guardan datos únicos es por medio de una clave principal, la cuál se va a explicar a continuación. \cite{CitaAJ10}
    \item \textbf{Clave principal:} Como ya se mencionó, las tablas tienen una clave principal, la cuál suele ser una columna o un conjunto de columnas cuyos valores identifican la forma única de cada fila de la tabla. Estas columnas se denominan claves principales de la tabla y aunque la clave principal sea una combinación de columnas, esta combinación es única dentro de la tabla. \cite{CitaAJ11}
    \item \textbf{Clave externa:} Esta es una columna o combinación de columnas que se usa para establecer un vínculo entre los datos de dos tablas. Cuando una tabla tiene dentro de sus columnas la clave principal de otra tabla, se dice que esa primera tabla está referenciando a la otra por medio de una clave externa. \cite{CitaAJ12}
    \item \textbf{Sistema de gestión de bases de datos (SGBD):} Son sistemas que ayudan a controlar las bases de datos. Estos sistemas actúan como interfaz entre los usuarios y las bases de datos, y se encargan justamente de gestionar los datos y las bases de datos como tal. En otras palabras, un sistema de gestión de bases de datos es un software utilizado para gestionar, almacenar y recuperar bases de datos, a su vez, proporciona una interfaz que permite a los usuarios leer, crear, borrar y actualizar datos. \cite{CitaAJ06}
\end{itemize}

\subsubsection{PostgreSQL:}
PostgreSQL es una SGBD relacional, el cual, a diferencia de otros este soporta tipos de datos relacionales y no relacionales. Este fue creado con el propósito de soportar cagras de trabajo desde pequeñas aplicaciones hasta sistemas complejos de procesamiento de datos a gran escala. 
Para este proyecto, se hará uso de este SGBD, ya que tiene compatibilidad con los diferentes lenguajes que vamos a usar, como lo es Java, y además, es libre de restricciones de licencia. \cite{CitaAJ05}

\subsection{Bases de datos no relacionales:}
Este tipo de bases de datos no siguen el esquema de filas y columnas como las bases de datos relacionales, en su lugar, este tipo de bases de datos usan un modelo de almacenamiento que está optimizado para los requisitos del tipo de dato que van a guardar. Dentro de este tipo de bases de datos existe una gran variedad, hay bases de datos que almacenan su información como pares clave/valor simple, como formatos JSON, o como un grafo que consta de bordes y vértices. Lo que caracteriza a este tipo de bases de datos es que no usan el modelo relacional. \cite{CitaAJ12}
