%=========================================================
\section{Modelo dinámico}	
\label{cap:modDinamico}

En la presente sección, se aborda en detalle el modelo dinámico del sistema propuesto, el cual constituye un pilar fundamental para la comprensión exhaustiva de su comportamiento y diseño, también, en este se explican las interacciones esenciales entre los actores que intervienen en el sistema y las funcionalidades que este ofrece.


Para facilitar la interpretación de las interacciones funcionales, en primer lugar, se describe la notación visual empleada en los diagramas de casos de uso. Esta notación, basada en la codificación por colores (detallada en la Figura \ref{fig:Notacion}), permite identificar de manera inmediata la prioridad y el estado de implementación de cada funcionalidad del sistema, proporcionando una visión clara de su relevancia y avance dentro del ciclo de desarrollo.


Seguidamente, se presenta el diagrama de estructura de usuarios (ilustrado en la Figura \ref{fig:EstructuraU}), el cual organiza jerárquicamente a los actores del sistema según sus roles y especializaciones. Esta representación visual facilita la comprensión de las diferentes categorías de usuarios, sus relaciones y los niveles de acceso que ostentan dentro del sistema, estableciendo un marco claro para entender sus interacciones posteriores.


Después, se procederá a la descripción individualizada de cada uno de los actores identificados que participan en el sistema. Para cada actor, se definirán de manera precisa sus responsabilidades específicas dentro del contexto del sistema y se detallarán los procesos clave en los que se involucran, sentando las bases para comprender su rol y su interacción con las funcionalidades del sistema.


A partir de esta comprensión de los actores y sus roles, se presentarán los diagramas de casos de uso para los sistemas móvil \ref{fig:casosDeUso1}  y web \ref{fig:casosDeUso2}. Estos diagramas ilustrarán gráficamente las funcionalidades específicas que ofrece cada plataforma, resaltando las particularidades y adaptaciones diseñadas para cada entorno de usuario.
Finalmente, se proporcionará una descripción exhaustiva de cada uno de los casos de uso seleccionados como necesarios. Para cada caso de uso, se detallarán sus precondiciones, el flujo principal de eventos, los posibles flujos alternativos y las postcondiciones, ofreciendo una visión completa de la funcionalidad esencial del sistema y su comportamiento ante diversas situaciones.


\subsection{Notación}

%\newpage

\begin{figure}[htbp!]
	\begin{center}
		\fbox{\includegraphics[width=.35\textwidth]{images/Notacion}}
		\caption{Análisis de viabilidad de los casos de uso.}
		\label{fig:Notacion}
	\end{center}
\end{figure}

A continuación, se detalla la notación empleada para la representación de los diagramas de casos de uso, la cual es fundamental para comprender la prioridad y el estado de implementación de cada funcionalidad del sistema. En la Figura \ref{fig:Notacion} se presenta la leyenda completa de esta notación, donde se utilizan cuatro colores distintos para clasificar los casos de uso según su análisis y desarrollo.


Los casos de uso representados en color verde fueron aquellos que, tras un análisis exhaustivo, se determinaron como prioritarios para la consecución de los objetivos principales del sistema y que, a la fecha, han sido completamente realizados o implementados. Esta categoría engloba las funcionalidades esenciales que sustentan la propuesta de valor de nuestro trabajo terminal.


Por otro lado, los casos de uso identificados con el color rojo fueron aquellos que, durante la etapa de análisis, se consideraron como no prioritarios para el alcance inicial del proyecto. En consecuencia, estas funcionalidades no fueron desarrolladas en la implementación del sistema.


Adicionalmente, se presenta una categoría de casos de uso en color amarillo. Estos casos fueron inicialmente considerados como prioritarios durante la fase de análisis; sin embargo, debido a diversos factores surgidos durante el desarrollo, se tomó la decisión de reconsiderar su prioridad y finalmente catalogarlos como no prioritarios, quedando fuera del alcance de la implementación actual.


Finalmente, los casos de uso representados en color gris son aquellos que, tras un análisis más detallado durante la etapa de desarrollo, se identificaron como componentes integrales de otros casos de uso adyacentes.


\subsection{Estructura de usuarios }

\begin{figure}[htbp!]
	\begin{center}
		\fbox{\includegraphics[width=.8\textwidth]{images/EstructuraU}}
		\caption{Estructura de los usuarios.}
		\label{fig:EstructuraU}
	\end{center}
\end{figure} 

En la Figura \ref{fig:EstructuraU} se ilustra la estructura jerárquica de los usuarios que interactúan con el sistema propuesto, detallando los diferentes tipos de roles y sus niveles de acceso. En el Nivel General, se encuentra el actor Usuario, que representa la abstracción más alta de cualquier entidad que interactúa con el sistema.

Del Usuario se derivan los Niveles Primarios: Empleado y Alumno. El Empleado engloba a todo el personal que labora en la institución, mientras que el Alumno representa a la población estudiantil.

Dentro del Nivel Primario de Empleado, encontramos los Subniveles de Empleado: Personal de seguridad y Docente. Estos roles poseen responsabilidades y permisos específicos dentro del sistema, diferenciándose de otros tipos de empleados.

También derivado del Empleado, se encuentra el Nivel Administrativo, representado por el actor Personal administrativo. Este nivel agrupa a roles con funciones de gestión y administración dentro del sistema, y se especializa en: Personal de gestión escolar, Presidente de academia, Jefe de departamento y Personal de la DAE (Dirección de administración Escolar). Cada uno de estos roles dentro del Nivel Administrativo tendrá permisos y accesos adaptados a sus responsabilidades específicas.

A continuación, se procederá a describir en detalle las responsabilidades y los procesos específicos de cada uno de los actores presentes en esta jerarquía.

%\newpage



%\newpage

%\newpage

\newpage

%---------------------------------------------------------
\subsection{Descripción de actores}

%---------------------------------------------------------
\begin{Usuario}{\hypertarget{tAlumno}{\subsubsection{Alumno}}}{
    Se refiere a las personas inscritas dentro de algún plan de estudios ofertado en la unidad académica.
}

\item[Responsabilidades:] \cdtEmpty
\begin{itemize}
    \item Asistir puntualmente a las clases, prácticas y evaluaciones.
    \item Respetar a docentes, compañeros y personal administrativo.
    \item Cumplir con los requisitos y actividades de las asignaturas inscritas, incluyendo tareas, proyectos y exámenes.
    \item Realizar oportunamente los trámites escolares como inscripciones, reinscripciones, solicitudes de documentos oficiales, etc.
    \item Portar credencial institucional en todo momento.
\end{itemize}
\textbf{Procesos clave:} \cdtEmpty
\begin{itemize}
    \item Inscribirse a asignaturas.
    \item Consultar calificaciones.
    \item Realizar pagos.
\end{itemize}
\end{Usuario}

\begin{Usuario}{\hypertarget{tPersonalSeguridad}{\subsubsection{Personal de seguridad}}}{
    Se refiere a las personas registradas como empleados y que permiten o no el acceso a la unidad académica.
}


\item[Responsabilidades:] \cdtEmpty
\begin{itemize}
    \item Supervisar el acceso y la salida de alumnos, personal docente y visitantes, asegurándose de que cumplan con los protocolos establecidos.
    \item Verificar la identificación de las personas que ingresan a las instalaciones.
    \item Responder de manera oportuna a incidentes o emergencias dentro de las instalaciones.
    \item Brindar apoyo al personal, docentes o alumnos en caso de accidentes o situaciones de riesgo.
\end{itemize}
\end{Usuario}

\begin{Usuario}{\hypertarget{tDocenteAplicador}{\subsubsection{Docente}}}{
    Se refiere a las personas registradas como empleados que dan clases a los alumnos y supervisan los ETS asignados.
}


\item[Responsabilidades:] \cdtEmpty
\begin{itemize}
    \item Impartir las clases de manera clara, puntual y completa, cumpliendo con los objetivos de aprendizaje.
    \item Diseñar y aplicar instrumentos de evaluación justos, objetivos y alineados con los contenidos del curso.
    \item Orientar a los alumnos en el desarrollo de competencias y habilidades.
    \item Resolver dudas o problemáticas académicas dentro y fuera del aula, cuando sea necesario.
    \item Registrar la asistencia de los alumnos y reportar incidencias graves.
    \item Cumplir con la entrega de calificaciones y reportes en tiempo y forma.
\end{itemize}
\end{Usuario}

\begin{Usuario}{\hypertarget{tPersonalGestion}{\subsubsection{Personal de gestión escolar}}}{
    Se refiere a las personas registradas como empleados y personal administrativo que realiza los procesos administrativos dentro de la ESCOM.
}
\item[Responsabilidades:] \cdtEmpty
\begin{itemize}
    \item Gestionar el proceso de inscripción y reinscripción de los alumnos, verificando que cumplan con los requisitos establecidos.
    \item Mantener y actualizar el historial académico de los alumnos en los sistemas institucionales.
    \item Revisar y validar actas de nacimiento, certificados y otros documentos oficiales requeridos para el registro de los alumnos.
    \item Atender solicitudes y problemáticas relacionadas con registros, certificados, bajas temporales y procesos extraordinarios.
    \item Brindar orientación a alumnos y docentes sobre trámites escolares, fechas importantes y normatividad académica.
\end{itemize}
\end{Usuario}

\begin{Usuario}{\hypertarget{tPersonalDAE}{\subsubsection{Personal de la DAE}}}{
    Se refiere a las personas registradas como empleados y personal administrativo que realiza los procesos administrativos dentro de la DAE.
}

\item[Responsabilidades:] \cdtEmpty
\begin{itemize}
    \item Fomentar y coordinar actividades extracurriculares que complementen la formación académica, como talleres, conferencias, eventos culturales y deportivos.
    \item Supervisar programas de apoyo académico, como tutorías, orientación educativa y psicológica.
    \item Difundir información sobre programas de movilidad académica, intercambios, convenios nacionales e internacionales y programas de servicio social.
    \item Gestionar y emitir las credenciales oficiales del IPN para los alumnos.
    \item Verificar que los documentos requeridos para la emisión de la credencial estén completos y sean válidos.
    \item Coordinar el proceso de inscripción de los alumnos.
    \item Actualizar y mantener los registros académicos y administrativos de los alumnos. 

\end{itemize}
\end{Usuario}


\begin{Usuario}{\hypertarget{tPresidente}{\subsubsection{Presidente de academia}}}{
Se refiere a las personas registradas como empleados y personal administrativo que lidera la academia de una unidad académica o área de conocimiento dentro de la ESCOM.
}
\item[Responsabilidades:] \cdtEmpty
\begin{itemize}
    \item Convocar y presidir las reuniones de academia, donde se toman decisiones sobre planes y programas de estudio.
    \item Coordinar la creación o actualización de planes y programas de estudio conforme a las necesidades del mercado laboral y las directrices institucionales.
    \item Verificar que los contenidos impartidos por los docentes sean consistentes con los objetivos de los programas.
    \item Detectar necesidades de capacitación entre los docentes y promover cursos o talleres. 
\end{itemize}
\end{Usuario}

\begin{Usuario}{\hypertarget{tJefe}{\subsubsection{Jefe de departamento}}}{
Se refiere a las personas registradas como empleados y personal administrativo que supervisa las actividades de una o más unidades académicas dentro de ESCOM.
}
\item[Responsabilidades:] \cdtEmpty
\begin{itemize}
\item Administrar los recursos humanos y materiales asignados al departamento.
\item Supervisar la implementación de los programas de estudio y el cumplimiento de los objetivos educativos.
\item Promover y coordinar proyectos de investigación, desarrollo tecnológico o vinculación relacionados con el departamento.
\item Atender quejas, sugerencias o problemas que surjan en el departamento, ya sea entre docentes o alumnos.
\end{itemize}
\end{Usuario}



\newpage

\subsection{Diagramas de casos de uso}

A continuación se muestran los diagramas de casos de uso:


\begin{figure}[htbp!]
	\begin{center}
		\fbox{\includegraphics[width=.9\textwidth]{images/casosDeUso1}}
		\caption{Diagrama de casos de uso del sistema movil.}
		\label{fig:casosDeUso1}
	\end{center}
\end{figure}

En la Figura \ref{fig:casosDeUso1} se presenta el diagrama de casos de uso del sistema móvil, el cual constituye el núcleo central del presente trabajo terminal. Este diagrama detalla la estructura funcional del sistema, así como las interacciones específicas de los siguientes actores: Alumno, Docente, Personal de seguridad, Presidente de academia y Jefe de departamento. La notación de colores empleada en este diagrama (detallada en la Figura \ref{fig:Notacion}) permite identificar la prioridad y el estado de cada caso de uso.
Como se observa, la mayoría de los casos de uso se presentan en color verde, indicando su prioridad y realización dentro del proyecto. Sin embargo, existen algunas excepciones que merecen una explicación detallada:


El caso de uso CU-03 (Consultar notificaciones) se presenta en color amarillo. Esta decisión se tomó durante la fase de desarrollo, al identificar que su implementación completa a lo largo del ciclo de vida del proyecto presentaba una complejidad significativa. Adicionalmente, se consideró que el valor añadido de esta funcionalidad a la experiencia general del usuario no justificaba el esfuerzo requerido para su desarrollo en el alcance actual del proyecto.


Los casos de uso CU-04 (Consultar periodos de ETS asignados al docente) y CU-16 (Consultar periodos de ETS inscritos del alumno) también se identifican con el color amarillo. La determinación de no priorizar la separación de la consulta de ETS por periodos se basó en la evaluación de su impacto en la funcionalidad principal del sistema. Se concluyó que esta distinción no aportaba un valor significativo a la experiencia del usuario y que la revisión especifica de ETS por periodo carecía de relevancia práctica dentro del contexto del sistema propuesto.


El caso de uso CU-40 (Solicitar desbloqueo de cuenta) se presenta en color amarillo debido a una limitación técnica inherente a la arquitectura supuesta del sistema. Al considerar que el sistema móvil opera en conexión directa con los registros del SAES (Sistema de Administración Escolar), la funcionalidad de bloquear o cambiar contraseñas desde la aplicación móvil se consideró inviable y fuera del alcance, ya que estas operaciones se gestionarían directamente a través de la plataforma del SAES.


Finalmente, el caso de uso CU-10 (Consultar lista de asistencia de alumnos inscritos al ETS) se presenta en color gris y se encuentra enlazado mediante una relación de inclusión al caso de uso CU-08 (Consultar lista de alumnos inscritos a un ETS). Esta decisión de diseño se tomó con el objetivo de optimizar la experiencia del usuario, integrando la visualización de ambas listas, es decir se muestra la lista de los alumnos inscritos al ETS y el estado de asistencia (si se presentó o no al ETS).


Los demás casos de uso presentados en color verde representan las funcionalidades prioritarias y ya implementadas del sistema móvil, las cuales serán explicadas en la sección de detalles de los casos de uso.


\newpage

%\newpage
\begin{figure}[htbp!]
	\begin{center}
		\fbox{\includegraphics[width=.9\textwidth]{images/casosDeUso2}}
		\caption{Diagrama de casos de uso del sistema web.}
		\label{fig:casosDeUso2}
	\end{center}
\end{figure}


En la Figura \ref{fig:casosDeUso2} se presenta el diagrama de casos de uso correspondiente al sistema web. Este diagrama ilustra las funcionalidades a las que tienen acceso principalmente el Personal de la DAE y el Personal de gestión escolar, complementando las funcionalidades ofrecidas por el sistema móvil. Al igual que en el diagrama del sistema móvil, se utiliza la notación de colores previamente definida para indicar la prioridad y el estado de implementación de cada caso de uso.

La mayoría de los casos de uso en este diagrama se presentan en color verde, lo que indica que fueron considerados prioritarios y han sido realizados. Estos casos de uso abarcan la gestión de información y procesos administrativos relevantes para los roles del Personal de la DAE y el Personal de gestión escolar.

Sin embargo, es importante destacar la relación existente entre los casos de uso CU-21 (Dar de alta a alumno), CU-22 (Crear credencial) y CU-23 (Capturar fotografía estudiantil). Inicialmente concebidos como funcionalidades separadas, durante la fase de desarrollo se tomó la decisión de unificarlos dentro del caso de uso CU-21. Esta integración se realizó con el objetivo de optimizar la experiencia del usuario, permitiendo que el Personal de la DAE realice el alta del alumno, la captura de su fotografía y la creación de la credencial en una única pantalla o flujo de proceso. 

\subsection{Detalle de los casos de uso}

%---------------------------------------------------------
% CASOS DE USO

% !TeX root = ../ejemplo.tex

%--------------------------------------
\label{CU-01}
\begin{UseCase}{CU-01}{Iniciar sesión del sistema móvil}{
		
		Permitir que solo los usuarios registrados en el sistema móvil puedan acceder a este, además de separar completamente las funciones de los usuarios del sistema móvil.
	}
	\UCitem{Versión}{\color{Gray}1}
	\UCitem{Autor}{\color{Gray}Huertas Ramírez Daniel Martín.}
	\UCitem{Supervisa}{\color{Gray}De la Cruz de la Cruz Alejandra.}
	\UCitem{Actor}{ \hyperlink{tDocente}{Docente}, \hyperlink{tPersonalSeguridad}{Personal de seguridad}, \hyperlink{tAlumno}{Alumno}, \hyperlink{tPresidente}{Presidente de academia} y \hyperlink{tJefe}{Jefe de departamento} }
	\UCitem{Propósito}{Que el usuario del sistema móvil pueda acceder al sistema móvil y sus funciones específicas.}
	\UCitem{Entradas}{ En caso del \hyperlink{tPersonalSeguridad}{Personal de seguridad}, ingresará con su CURP y Contraseña. En caso del alumno con su Boleta y Contraseña. En caso del \hyperlink{tDocente}{Docente}, \hyperlink{tPresidente}{Presidente de academia} y \hyperlink{tJefe}{Jefe de departamento} se usará su RFC y su Contraseña}
	\UCitem{Origen}{Pantalla táctil}
	\UCitem{Salidas}{Saludo del sistema y mención de su nombre.}
	\UCitem{Destino}{Pantalla \IUref{IUE01}{Pantalla de saludo del docente} si es un docente, a la \IUref{IUE02}{Pantalla de saludo del personal de seguridad} si es un personal de seguridad, a la \IUref{IUE03}{Pantalla de saludo del alumno} si es un alumno y a la \IUref{IUE06}{Pantalla de saludo del presidente de academia/jefe de departamento} si es un presidente de academia o jefe de departamento.}
	\UCitem{Precondiciones}{El usuario debe estar registrado en el sistema móvil.}
	\UCitem{Postcondiciones}{El usuario accede al sistema móvil y podrá realizar las acciones pertinentes a su rol.}
	\UCitem{Errores}{
		E1: Cuando falte algún dato requerido entonces el sistema muestra el mensaje \textbf{\hyperref[msg:CU01-E1]{MSG-CU01-E1}}{``Por favor, completa todos los campos.''}
		
		E2: Cuando los datos no concuerdan con ninguna cuenta de usuario, se muestra el mensaje \textbf{\hyperref[msg:CU01-E2]{MSG-CU01-E2}} ``Datos incorrectos.''
		
		E3: Cuando se pierde la conexión durante el proceso, los procesos se cancelan y se muestra el mensaje \textbf{\hyperref[msg:CU01-E3]{MSG-CU01-E3}} ``Conexión perdida.''
	}
	\UCitem{Tipo}{Caso de uso primario}
	\UCitem{Observaciones}{}
\end{UseCase}
%--------------------------------------

\begin{UCtrayectoria}
	\UCpaso[\UCactor] El usuario inicia la aplicación y accede a la pantalla \IUref{IU01}{Iniciar sesión del sistema móvil}.
	\UCpaso[\UCactor] Si el usuario es un alumno, introduce su número de boleta y su contraseña. Si el usuario es Personal de seguridad, introduce su CURP y contraseña. Si el usuario es Docente, Presidente de academia o Jefe de departamento, introduce su RFC y contraseña en el sistema a través de la \IUref{IU01}{Iniciar sesión del sistema móvil}\label{CU01.introduceDatos}.
	\UCpaso[\UCactor] El usuario confirma la operación presionando el botón \IUbutton{Entrar}.
	\UCpaso El sistema verifica que todos los datos requeridos hayan sido capturados.
	\UCpaso El sistema verifica que el usuario esté registrado en el sistema.
	\UCpaso El sistema verifica que la contraseña proporcionada corresponde al usuario identificado.
	\UCpaso El sistema determina el rol del usuario que ha iniciado sesión.
	\UCpaso La sesión se inicia con éxito.
	\UCpaso El usuario es redirigido a la {Pantalla \IUref{IUE01}{Pantalla de saludo del docente} si es un docente, a la \IUref{IUE02}{Pantalla de saludo del personal de seguridad} si es un personal de seguridad, a la \IUref{IUE03}{Pantalla de saludo del alumno} si es un alumno y a la \IUref{IUE06}{Pantalla de saludo del presidente de academia/jefe de departamento} si es un presidente de academia o jefe de departamento}.
\end{UCtrayectoria}





\newpage

% !TeX root = ../ejemplo.tex

%--------------------------------------
\label{CU-02}
\begin{UseCase}{CU-02}{Consultar calendario escolar}{
		Permitir que los usuarios vean el calendario escolar y puedan solicitar al sistema que les mencione cuántos días faltan para que inicie el próximo periodo de ETS.
	}
	\UCitem{Versión}{\color{Gray}1}
	\UCitem{Autor}{\color{Gray}Huertas Ramírez Daniel Martín}
	\UCitem{Supervisa}{\color{Gray}De la Cruz de la Cruz Alejandra.}
	\UCitem{Actor}{\hyperlink{tUsuario}{Usuario}}
	\UCitem{Propósito}{Que el usuario consulte cuántos días faltan para el próximo periodo de ETS.}
	\UCitem{Entradas}{Solicitud de consulta.}
	\UCitem{Origen}{Pantalla táctil}
	\UCitem{Salidas}{Mensaje indicando los días faltantes para el periodo de ETS o si ya es periodo de ETS.}
	\UCitem{Destino}{Pantalla de la aplicación móvil.}
	\UCitem{Precondiciones}{El usuario debe haber iniciado sesión.}
	\UCitem{Postcondiciones}{El usuario recibe información sobre la fecha del próximo periodo de ETS.}
	\UCitem{Errores}{
		E1: Cuando no se ha registrado el siguiente periodo de ETS, el sistema muestra el mensaje \textbf{``Aún no está registrado el siguiente periodo de ETS.''}.
		
		E2: Cuando se pierde la conexión durante el proceso, los procesos se cancelan y se muestra el mensaje \textbf{ ``Conexión perdida.''}
	}
	\UCitem{Tipo}{Caso de uso primario}
	\UCitem{Observaciones}{Ninguna}
\end{UseCase}
%--------------------------------------

\begin{UCtrayectoria}
	\UCpaso[\UCactor] El usuario accede a la pantalla \IUref{IU02}{Pantalla Consultar calendario escolar} para la app móvil mediante el botón con forma de calendario visible en cualquier pantalla excepto la de inicio de sesión\label{CU02.accedePantalla}.
	\UCpaso[\UCactor] El usuario solicita conocer cuántos días faltan para que el periodo de ETS inicie, oprimiendo el botón \IUbutton{Calcular cuántos días faltan para el periodo de ETS}.
	\UCpaso El sistema recupera la fecha de inicio del próximo periodo de ETS registrada en el sistema.
	\UCpaso El sistema calcula la diferencia en días entre la fecha actual y la fecha de inicio del próximo periodo de ETS.
	\UCpaso Si la fecha de inicio del periodo de ETS es posterior a la fecha actual, el sistema muestra el mensaje: \textbf{ ``Faltan (cantidad de días) días para el periodo de ETS.''}
	\UCpaso Si la fecha de inicio del periodo de ETS es igual o anterior a la fecha actual, el sistema muestra el mensaje: \textbf{ ``Actualmente es periodo de ETS.''}
	\UCpaso [\UCactor] El usuario visualiza la cantidad de días que faltan para el periodo del ETS o si ya es el periodo de ETS.
\end{UCtrayectoria}








\newpage

% !TeX root = ../ejemplo.tex

%--------------------------------------
\begin{UseCase}{CU-03}{Consultar notificaciones}{
    Permitir a los usuarios revisar sus notificaciones con más detenimiento y establecerlas como leídas.
}
    \UCitem{Versión}{\color{Gray}1}
    \UCitem{Autor}{\color{Gray}Huertas Ramírez Daniel Martín}
    \UCitem{Supervisa}{\color{Gray}De la Cruz de la Cruz Alejandra.}
    \UCitem{Actor}{\hyperlink{Usuario}{Usuario}}
    \UCitem{Propósito}{Que el usuario revise sus notificaciones con detenimiento y las establezca como leídas.}
    \UCitem{Entradas}{Ninguna}
    \UCitem{Origen}{Pantalla táctil}
    \UCitem{Salidas}{Menciona que la notificación seleccionada ha sido establecida como leída.}
    \UCitem{Destino}{Ninguno}
    \UCitem{Precondiciones}{El usuario debe de haber iniciado sesión.}
    \UCitem{Postcondiciones}{El usuario revisó sus notificaciones.}
    \UCitem{Errores}{
        E1: Cuando el usuario no tiene notificaciones el sistema muestra el mensaje {\bf MSG-8}{``Actualmente no hay notificaciones.''}    
    }
    \UCitem{Tipo}{Caso de uso primario}
    \UCitem{Observaciones}{}

\end{UseCase}
%-------------------------------------- 

\begin{UCtrayectoria}
    \UCpaso[\UCactor] El usuario accede a la pantalla \IUref{IU03}{Pantalla Consultar notificaciones}\label{CU03.introduceDatos} para la app móvil  mediante el botón con forma de campana en cualquier pantalla excepto los inicios de sesión.
    \UCpaso[\UCactor] El usuario decide consultar sus notificaciones más actuales \Trayref{A}.
    \UCpaso[\UCactor] El usuario marco como leídas las notificaciones que acaba de leer mediante el botón con forma de palomita especifico de cada notificación.
    \UCpaso El sistema marca como leídas las notificaciones.

\end{UCtrayectoria}

\begin{UCtrayectoriaA}{A}{El usuario quiere consultar las notificaciones de una fecha específica}
	\UCpaso[\UCactor] El usuario usa el buscador de la parte superior para buscar notificaciones según su fecha.
	\UCpaso[\UCactor] El usuario marco como leídas las notificaciones que acaba de leer mediante el botón con forma de palomita especifico de cada notificación.
	\UCpaso El sistema marca como leídas las notificaciones.

\end{UCtrayectoriaA}



\newpage

% \IUref{IUAdmPS}{Administrar Planta de Selección}
% \IUref{IUModPS}{Modificar Planta de Selección}
% \IUref{IUEliPS}{Eliminar Planta de Selección}

%-------------------------------------- COMIENZA descripción del caso de uso.

%\begin{UseCase}[archivo de imágen]{UCX}{Nombre del Caso de uso}{
%--------------------------------------
% !TeX root = ../ejemplo.tex
\begin{UseCase}{CU-04}{Consultar periodos de ETS asignados al docente}{
	Este caso de uso permite al docente consultar los periodos de ETS que tiene asignados. 
}
\UCitem{Versión}{\color{Gray}1.0} 
\UCitem{Autor}{\color{Gray}De la cruz De la cruz Alejandra}
\UCitem{Supervisa}{\color{Gray}Huertas Ramírez Daniel Martín }
\UCitem{Actor}{\hyperlink{Docente}{Docente}}
\UCitem{Propósito}{Permitir al docente consultar los periodos de ETS que le han sido asignados.}
\UCitem{Entradas}{Ninguna}
\UCitem{Origen}{Pantalla táctil}
\UCitem{Salidas}{Lista de periodos de ETS asignados.}
\UCitem{Destino}{\IUref{IU04}{Pantalla Periodo de ETS}}
\UCitem{Precondiciones}{El docente debe estar autenticado en el sistema.}
\UCitem{Postcondiciones}{El docente ha consultado los periodos de ETS asignados.}
\UCitem{Errores}{
		E1: El sistema no puede recuperar la información de los periodos y muestra el mensaje {\bf MSG-9}{``Error al consultar la base de datos. Intente nuevamente más tarde''.}
		
		E2: No hay periodos de ETS asignados al docente y se muestra el mensaje {\bf MSG-10}{``No tienes periodos de ETS asignados.''}
}
\UCitem{Tipo}{Se entiende del CU-01 Iniciar sesión del sistema móvil}
\UCitem{Observaciones}{Ninguna}
\end{UseCase}
%--------------------------------------
\begin{UCtrayectoria}
\UCpaso[\UCactor] El docente accede a la \IUref{IUE01}{Pantalla saludo del docente} después de haber iniciado sesión.
\UCpaso[\UCactor] El docente presiona el botón \IUbutton{Consultar periodos de ETS}.
\UCpaso El sistema verifica que el docente cuente con periodos de ETS. \Trayref{A}.
\UCpaso El sistema busca en la base de datos los periodos de ETS asignados al docente.
\UCpaso El sistema despliega la lista de periodos de ETS asignados al docente en la \IUref{IU04}{Pantalla Periodo de ETS}.
\end{UCtrayectoria}
%--------------------------------------        
\begin{UCtrayectoriaA}{A}{No hay periodos asignados al docente}
\UCpaso El sistema verifica y no encuentra registros de periodos de ETS asignados al docente.
\UCpaso El sistema muestra un mensaje: {\bf MSG-10}{`` No tienes periodos de ETS asignados.''}
\UCpaso[\UCactor] El docente presiona el botón \IUbutton{Regresar} para volver a la pantalla anterior.
\UCpaso Fin de la trayectoria alternativa.
\end{UCtrayectoriaA}
%--------------------------------------        
\begin{UCtrayectoriaA}{B}{Error en la conexión con la base de datos}
\UCpaso El sistema muestra un mensaje de error: {\bf MSG-9}{``Error al consultar la base de datos. Intente nuevamente más tarde.''}
\UCpaso[\UCactor] El docente resiona el botón \IUbutton{Aceptar} para cerrar el mensaje.
\UCpaso[\UCactor] El docente puede intentar la consulta nuevamente o presionar el botón \IUbutton{Regresar} para volver a la pantalla anterior.
\UCpaso Fin de la trayectoria alternativa.
\end{UCtrayectoriaA}
%-------------------------------------- TERMINA descripción del caso de uso.

\newpage

% \IUref{IUAdmPS}{Administrar Planta de Selección}
% \IUref{IUModPS}{Modificar Planta de Selección}
% \IUref{IUEliPS}{Eliminar Planta de Selección}

%-------------------------------------- COMIENZA descripción del caso de uso.

%\begin{UseCase}[archivo de imágen]{UCX}{Nombre del Caso de uso}{
%--------------------------------------
\label{CU-05}
\begin{UseCase}{CU-05}{Consultar ETS asignados}{
		Este caso de uso permite al docente consultar los ETS que tiene asignados.
	}
	\UCitem{Versión}{\color{Gray}2.0}
	\UCitem{Autor}{\color{Gray}De la cruz De la cruz Alejandra}
	\UCitem{Supervisa}{\color{Gray}Huertas Ramírez Daniel Martín}
	\UCitem{Actor}{\hyperlink{tDocente}{Docente}}
	\UCitem{Propósito}{Permitir al docente consultar los ETS que le han sido asignados.}
	\UCitem{Entradas}{Solicitud de consulta.}
	\UCitem{Origen}{Pantalla táctil}
	\UCitem{Salidas}{Lista de ETS asignados (Unidad de Aprendizaje, Periodo, Fecha, Turno) o indicación de que no hay ETS asignados.}
	\UCitem{Destino}{\IUref{IU06}{Pantalla de Consultar ETS}}
	\UCitem{Precondiciones}{El docente debe haber iniciado sesión en el sistema (\textbf{\hyperref[CU-01]{CU-01 Iniciar sesión del sistema móvil}}).}
	\UCitem{Postcondiciones}{El docente ha consultado los ETS asignados.}
	\UCitem{Errores}{
		E1: El sistema no puede recuperar la información de los ETS asignados y se muestra el mensaje \textbf{\hyperref[msg:CU05-E1]{MSG-CU05-E1}}{``Error al consultar la base de datos. Intente nuevamente más tarde.''}.
		
		E2: Cuando se pierde la conexión durante el proceso, los procesos se cancelan y se muestra el mensaje \textbf{\hyperref[msg:CU05-E2]{MSG-CU05-E2}} ``Conexión perdida.''
	}
	\UCitem{Tipo}{Caso de uso primario}
	\UCitem{Observaciones}{Ninguna}
\end{UseCase}
%--------------------------------------
\begin{UCtrayectoria}
	\UCpaso[\UCactor] El docente accede a la \IUref{IU05}{Pantalla de Consultar ETS} desde el menú principal.
	\UCpaso El sistema recupera la lista de todos los ETS asignados al docente.
	\UCpaso El sistema muestra la lista de ETS, incluyendo para cada uno: Unidad de Aprendizaje, Periodo, Fecha y Turno.
	\UCpaso[\UCactor] El docente visualiza la lista de ETS asignados.
	\UCpaso[\UCactor] El docente selecciona un ETS de la lista.
	\UCpaso El sistema redirige al docente a la pantalla \IUref{IU06}{Pantalla Información de ETS}.
	\UCpaso[\UCactor] Opcionalmente, el docente puede presionar el botón \IUbutton{Todos} para ver todos los ETS asignados.
	\UCpaso[\UCactor] Opcionalmente, el docente puede presionar el botón \IUbutton{Mis ETS} para ver los ETS que tiene pendientes por aplicar o supervisar.
\end{UCtrayectoria}

%--------------------------------------
\begin{UCtrayectoriaA}{A}{No hay ETS asignados}
	\UCpaso El sistema verifica que no hay ETS asignados al docente.
	\UCpaso El sistema muestra un mensaje indicando que no hay ETS asignados.
	\UCpaso[\UCactor] El docente visualiza el mensaje.
	\UCpaso[\UCactor] El docente presiona el botón \IUbutton{Regresar} para volver a la pantalla anterior.
	\UCpaso Fin de la trayectoria alternativa.
\end{UCtrayectoriaA}

%--------------------------------------
\begin{UCtrayectoriaA}{B}{Error en la conexión con la base de datos}
	\UCpaso El sistema intenta recuperar la lista de ETS asignados.
	\UCpaso Ocurre un error en la conexión con la base de datos.
	\UCpaso El sistema muestra el mensaje de error: \textbf{\hyperref[msg:CU05-E1]{MSG-CU05-E1}}{``Error al consultar la base de datos. Intente nuevamente más tarde.''}.
	\UCpaso[\UCactor] El docente presiona el botón \IUbutton{Aceptar} para cerrar el mensaje.
	\UCpaso[\UCactor] El docente puede intentar la consulta nuevamente o presionar el botón \IUbutton{Regresar} para volver a la pantalla anterior.
	\UCpaso Fin de la trayectoria alternativa.
\end{UCtrayectoriaA}

%-------------------------------------- TERMINA descripción del caso de uso.
\newpage

% \IUref{IUAdmPS}{Administrar Planta de Selección}
% \IUref{IUModPS}{Modificar Planta de Selección}
% \IUref{IUEliPS}{Eliminar Planta de Selección}

%-------------------------------------- COMIENZA descripción del caso de uso.

%\begin{UseCase}[archivo de imágen]{UCX}{Nombre del Caso de uso}{
%--------------------------------------
% !TeX root = ../ejemplo.tex
\label{CU-06}
\begin{UseCase}{CU-06}{Mostrar información de los ETS asignados}{
		Este caso de uso permite al docente visualizar la información detallada de los ETS que tiene asignados.
	}
	\UCitem{Versión}{\color{Gray}2.0}
	\UCitem{Autor}{\color{Gray}De la cruz De la cruz Alejandra}
	\UCitem{Supervisa}{\color{Gray}Huertas Ramírez Daniel Martín}
	\UCitem{Actor}{\hyperlink{tDocente}{Docente}}
	\UCitem{Propósito}{Permite al docente visualizar la información detallada de cada ETS que tiene asignado.}
	\UCitem{Entradas}{Selección de un ETS.}
	\UCitem{Origen}{\IUref{IU05}{Pantalla de Consultar ETS}}
	\UCitem{Salidas}{Detalle de ETS asignado: Tipo de ETS, Unidad de Aprendizaje, Periodo, Fecha, Turno, Cupo, Duración, Salones (con Tipo de Salón por cada salón), Hora. O indicación de error.}
	\UCitem{Destino}{\IUref{IU06}{Pantalla Información de ETS}}
	\UCitem{Precondiciones}{El docente debe haber iniciado sesión (\textbf{\hyperref[CU-01]{CU-01 Iniciar sesión del sistema móvil}}) y haber consultado la lista de ETS asignados (\textbf{\hyperref[CU-05]{CU-05 Consultar ETS asignados}}).}
	\UCitem{Postcondiciones}{El docente ha visualizado la información detallada del ETS seleccionado.}
	\UCitem{Errores}{
		E1: El sistema no puede recuperar la información detallada del ETS y muestra el mensaje \textbf{\hyperref[msg:CU06-E1]{MSG-CU06-E1}}{``Ocurrió un error al desplegar los detalles del ETS.''}.
		E2: Error en la conexión con la base de datos y muestra el mensaje \textbf{\hyperref[msg:CU06-E2]{MSG-CU06-E2}}{``Error al consultar la base de datos. Intente nuevamente más tarde.''}.
	}
	\UCitem{Tipo}{Caso de uso primario}
	\UCitem{Observaciones}{Si el docente es el aplicador del ETS, visualizará un botón llamado "Solicitar remplazo" que lo lleva a la \IUref{IU07}{Pantalla Solicitar remplazo}. También visualiza un botón llamado "Ir a la lista de alumnos" que lo lleva a la \IUref{IU08}{Pantalla Lista de alumnos inscritos a un ETS}.}
\end{UseCase}
%--------------------------------------
\begin{UCtrayectoria}
	\UCpaso[\UCactor] El docente selecciona el ETS que desea visualizar desde la \IUref{IU05}{Pantalla de Consultar ETS}.
	\UCpaso[\UCsist] El sistema recupera la información detallada del ETS seleccionado. \Trayref{A}, \Trayref{B}
	\UCpaso El sistema despliega la información detallada del ETS en la \IUref{IU06}{Pantalla Información de ETS}, incluyendo: Tipo de ETS, Unidad de Aprendizaje, Periodo, Fecha, Turno, Cupo, Duración, Salones (con Tipo de Salón por cada salón) y Hora.
	\UCpaso[\UCactor] El docente visualiza la información detallada del ETS.
	\UCpaso Si hay alumnos inscritos en el ETS, el docente visualiza un botón \IUbutton{Ir a la lista de alumnos}.
	\UCpaso[\UCactor] Si el docente presiona el botón \IUbutton{Ir a la lista de alumnos}, el sistema lo redirige a la \IUref{IU13}{Pantalla consultar lista de alumnos inscritos a un ETS}.
	\UCpaso Si el docente es el aplicador del ETS, visualiza un botón \IUbutton{Solicitar remplazo}.
	\UCpaso[\UCactor] Si el docente presiona el botón \IUbutton{Solicitar remplazo}, el sistema lo redirige a la \IUref{IU07}{Pantalla Solicitar remplazo}.
\end{UCtrayectoria}
%--------------------------------------
\begin{UCtrayectoriaA}{A}{No hay detalles disponibles para el ETS seleccionado}
	\UCpaso El sistema muestra el mensaje: \textbf{\hyperref[msg:CU06-E1]{MSG-CU06-E1}}{``Ocurrió un error al desplegar los detalles del ETS.''}
	\UCpaso[\UCactor] El docente presiona el botón \IUbutton{Regresar} para volver a la lista de ETS (\IUref{IU05}{Pantalla de Consultar ETS}).
	\UCpaso Fin de la trayectoria alternativa.
\end{UCtrayectoriaA}

%--------------------------------------
\begin{UCtrayectoriaA}{B}{Error en la conexión con la base de datos}
	\UCpaso El sistema intenta recuperar la información detallada del ETS.
	\UCpaso Ocurre un error en la conexión con la base de datos.
	\UCpaso El sistema muestra el mensaje de error: \textbf{\hyperref[msg:CU06-E2]{MSG-CU06-E2}}{``Error al consultar la base de datos. Intente nuevamente más tarde.''}.
	\UCpaso[\UCactor] El docente presiona el botón \IUbutton{Aceptar} para cerrar el mensaje.
	\UCpaso[\UCactor] El docente puede intentar la consulta nuevamente o presionar el botón \IUbutton{Regresar} para volver a la pantalla anterior (\IUref{IU05}{Pantalla de Consultar ETS}).
	\UCpaso Fin de la trayectoria alternativa.
\end{UCtrayectoriaA}

%-------------------------------------- TERMINA descripción del caso de uso.


\newpage

% \IUref{IUAdmPS}{Administrar Planta de Selección}
% \IUref{IUModPS}{Modificar Planta de Selección}
% \IUref{IUEliPS}{Eliminar Planta de Selección}

%-------------------------------------- COMIENZA descripción del caso de uso.

%\begin{UseCase}[archivo de imágen]{UCX}{Nombre del Caso de uso}{
%--------------------------------------
\begin{UseCase}{CU-07}{Solicitar remplazo }{
	Este caso de uso permite a un docente solicitar un reemplazo para que aplique el ETS cuando no pueda asistir.
}
\UCitem{Versión}{\color{Gray}1.0}
\UCitem{Autor}{\color{Gray}De la cruz De la cruz Alejandra}
\UCitem{Supervisa}{\color{Gray}Huertas Ramírez Daniel}
\UCitem{Actor}{\hyperlink{PersonalAcademico}{Docente}}
\UCitem{Propósito}{Permitir al docente responsable de un ETS solicitar apoyo de otro profesor para llevar a cabo la aplicación, notificando al jefe de departamento o al presidente de academia.}
\UCitem{Entradas}{Identificador del ETS y razón por la que se pide el remplazo.}
\UCitem{Origen}{Pantalla táctil}
\UCitem{Salidas}{
	Se muestra el mensaje \textbf{MSG-13}: ``Solicitud exitosa. La solicitud de reemplazo ha sido registrada correctamente'', indicando que la solicitud fue enviada con éxito.
	
	Se muestra el mensaje \textbf{MSG-14}: ``Ya existe una solicitud pendiente para este ETS'', indicando que ya se ha realizado una solicitud previa para el ETS.}
\UCitem{Destino}{\IUref{IU06}{Pantalla Información de ETS}}
\UCitem{Precondiciones}{El docente debe estar autenticado en el sistema y tener un ETS asignado.}
\UCitem{Postcondiciones}{La solicitud ha sido enviada al jefe de departamento y/o al presidente de academia.}
\UCitem{Errores}{
	E1: Cuando se pierde la conexión durante el proceso, los procesos se cancelan y se muestra el mensaje {\bf MSG-4}  ``Error al enviar la solicitud''
}
\UCitem{Tipo}{Se entiende del CU-06 Mostrar información de los ETS asignados}
\UCitem{Observaciones}{Ninguna}
\end{UseCase}
%--------------------------------------
\begin{UCtrayectoria}
\UCpaso[\UCactor] El docente accede a la \IUref{IU06}{Pantalla Información de ETS}.
\UCpaso[\UCactor] El docente selecciona la opción \IUbutton{Solicitar reemplazo} para el ETS asignado.
\UCpaso El sistema redirige al docente a la pantalla \IUref{IU07}{ Solicitar remplazo }.
\UCpaso El sistema muestra el ETS selecciona.
\UCpaso[\UCactor] El docente ingresa la razón para solicitar el remplazo.
\UCpaso[\UCactor] El docente oprime el botón \IUbutton{Enviar solicitud} [Trayectoria A] [Trayectoria B] .
\UCpaso El sistema envía la solicitud de remplazo al jefe de departamento y/o al presidente de academia.
\end{UCtrayectoria}
%--------------------------------------        
\begin{UCtrayectoriaA}{A}{Ya existe una solicitud pendiente para el ETS}
	\UCpaso El sistema valida que exista una solicitud previa no resuelta.
	\UCpaso El sistema muestra el mensaje: {\bf MSG-14}{``Ya existe una solicitud pendiente para este ETS''}.
	\UCpaso[\UCactor] El docente presiona \IUbutton{Aceptar} para regresar a \IUref{IU06}{Pantalla Información de ETS}.
\end{UCtrayectoriaA}
%-------------------------------------- TERMINA descripción del caso de uso.
\begin{UCtrayectoriaA}{B}{Error al enviar la solicitud}
	\UCpaso El sistema detecta un fallo en el servidor o conexión.
	\UCpaso Muestra el mensaje: {\bf MSG-16}{``Error al enviar la solicitud''}.
	\UCpaso[\UCactor] El docente selecciona:
	\begin{itemize}
		\item \IUbutton{Cancelar} para volver a \IUref{IU06}{Pantalla Información de ETS}.
	\end{itemize}
\end{UCtrayectoriaA}

\newpage

% \IUref{IUAdmPS}{Administrar Planta de Selección}
% \IUref{IUModPS}{Modificar Planta de Selección}
% \IUref{IUEliPS}{Eliminar Planta de Selección}

%-------------------------------------- COMIENZA descripción del caso de uso.

%\begin{UseCase}[archivo de imágen]{UCX}{Nombre del Caso de uso}{
%--------------------------------------
\label{CU-08}
\begin{UseCase}{CU-08}{Consultar lista de alumnos inscritos a un ETS}{
		
		Este caso de uso permite al docente consultar la lista de los alumnos inscritos a un ETS asignado.
		
		Ademas, el caso de uso CU-10 (Consultar lista de asistencia de alumnos inscritos a los ETS) se ha unificado con CU-08 (Consultar lista de alumnos inscritos a un ETS) para mejorar la eficiencia y la experiencia del usuario. La funcionalidad de visualización de la asistencia, previamente contemplada en CU-10, se integra ahora directamente en la \IUref{IU13}{Pantalla consultar lista de alumnos inscritos a un ETS} de CU-08, presentando la lista de alumnos y su estado de asistencia en una única interfaz.
	}
	\UCitem{Versión}{\color{Gray}2.0}
	\UCitem{Autor}{\color{Gray}De la cruz De la cruz Alejandra}
	\UCitem{Supervisa}{\color{Gray}Huertas Ramírez Daniel Martín}
	\UCitem{Actor}{\hyperlink{tDocente}{Docente}}
	\UCitem{Propósito}{Permitir al docente visualizar la lista de alumnos inscritos en un ETS para verificar su asistencia y acceder a la creación de reportes dentro del periodo permitido.}
	\UCitem{Entradas}{Solicitud de ver la lista de alumnos.}
	\UCitem{Origen}{\IUref{IU06}{Pantalla Información de ETS}}
	\UCitem{Salidas}{Lista de los alumnos inscritos al ETS (Boleta y Nombre) junto con un icono representativo del estado de su asistencia. Mensajes informativos sobre el periodo de reporte y permisos.}
	\UCitem{Destino}{\IUref{IU13}{Pantalla consultar lista de alumnos inscritos a un ETS}.}
	\UCitem{Precondiciones}{El docente debe haber iniciado sesión (\textbf{\hyperref[CU-01]{CU-01 Iniciar sesión del sistema móvil}}) y haber consultado la información detallada del ETS (\textbf{\hyperref[CU-06]{CU-06 Mostrar información de los ETS asignados}}).}
	\UCitem{Postcondiciones}{El docente ha visualizado la lista de asistencia de los alumnos inscritos al ETS y puede acceder a la creación de reportes si se encuentra dentro del periodo permitido y tiene los permisos necesarios.}
	\UCitem{Errores}{
		E1: El sistema pierde la conexión al intentar recuperar la lista de asistencia y se muestra el mensaje \textbf{\hyperref[msg:CU08-E2]{MSG-CU08-E2}}{ ``Error al consultar la base de datos. Intente nuevamente más tarde.''}.
	}
	\UCitem{Tipo}{Caso de uso primario}
	\UCitem{Observaciones}{Cada alumno en la lista se presenta con dos botones: uno con su boleta y nombre, y otro con un icono representativo de su estado de asistencia. El botón con el icono redirige a la \textbf{Pantalla Reporte} (\IUref{IUXX}{Pantalla Reporte}). El botón con la boleta y nombre del alumno redirige a la \IUref{IU20}{Pantalla mostrar la foto e información del alumno} **solo si la hora actual se encuentra entre 10 minutos antes del inicio del ETS y 2 horas después del inicio del ETS.** Si se quiere conocer todos los iconos y su significado, porfavor revisar \textbf{\hyperref[sec:tablaIconosAsistencia]{Tabla de Iconos de Estado de Asistencia}.}}
\end{UseCase}
%--------------------------------------
\begin{UCtrayectoria}
	\UCpaso[\UCactor] El docente presiona el botón \IUbutton{Ir a la lista de alumnos} desde la pantalla \IUref{IU06}{Pantalla Información de ETS}.
	\UCpaso El sistema verifica la existencia del ETS. \Trayref{C}
	\UCpaso El sistema recupera la lista de alumnos inscritos en el ETS junto con su estado de asistencia. \Trayref{A} \Trayref{B}
	\UCpaso El sistema muestra la lista de alumnos en la \IUref{IU13}{Pantalla consultar lista de alumnos inscritos a un ETS}. Cada alumno se muestra con un botón que contiene su boleta y nombre (habilitado según el periodo de reporte), y un botón con el icono representativo de su estado de asistencia.
	\UCpaso[\UCactor] El docente visualiza la lista de alumnos.
	\UCpaso Para cada alumno, el sistema verifica si la hora actual se encuentra dentro del periodo permitido para generar reportes (10 minutos antes hasta 2 horas después del inicio del ETS). \Trayref{D}, \Trayref{E}
	\UCpaso[\UCactor] Si el periodo es válido, al presionar el botón con la boleta y nombre del alumno, el sistema lo redirige a la \IUref{IU20}{Pantalla mostrar la foto e información del alumno}.
	\UCpaso[\UCactor] Al presionar el botón con el icono, el sistema lo redirige a la \textbf{Pantalla Reporte} (\IUref{IUXX}{Pantalla Reporte}).
\end{UCtrayectoria}

%--------------------------------------
\begin{UCtrayectoriaA}{A}{El ETS no tiene alumnos inscritos}
	\UCpaso El sistema muestra el mensaje: \textbf{\hyperref[msg:CU08-A1]{MSG-CU08-A1}}{ ``No hay alumnos inscritos al ETS.''}.
	\UCpaso[\UCactor] El docente presiona el botón \IUbutton{Regresar} para volver a la pantalla \IUref{IU06}{Pantalla Información de ETS}.
	\UCpaso Fin de la trayectoria alternativa.
\end{UCtrayectoriaA}
%--------------------------------------
\begin{UCtrayectoriaA}{B}{Error en la conexión con la base de datos}
	\UCpaso El sistema intenta recuperar la lista de alumnos inscritos.
	\UCpaso Ocurre un error en la conexión con la base de datos.
	\UCpaso El sistema muestra el mensaje de error: \textbf{\hyperref[msg:CU08-E2]{MSG-CU08-B2}}{ ``Error al consultar la base de datos. Intente nuevamente más tarde.''}.
	\UCpaso[\UCactor] El docente presiona el botón \IUbutton{Aceptar} para cerrar el mensaje.
	\UCpaso[\UCactor] El docente puede intentar la consulta nuevamente.
	\UCpaso Fin de la trayectoria alternativa.
\end{UCtrayectoriaA}
%--------------------------------------
\begin{UCtrayectoriaA}{C}{El ETS no existe}
	\UCpaso El sistema no encuentra el ETS correspondiente.
	\UCpaso El sistema muestra el mensaje de error: \textbf{\hyperref[msg:CU08-C1]{MSG-CU08-C1}}{ ``El ETS seleccionado no es válido.''}.
	\UCpaso[\UCactor] El docente presiona el botón \IUbutton{Regresar} para volver a la pantalla \IUref{IU06}{Pantalla Información de ETS}.
	\UCpaso Fin de la trayectoria alternativa.
\end{UCtrayectoriaA}
%--------------------------------------
\begin{UCtrayectoriaA}{D}{Aún no es periodo para crear reportes}
	\UCpaso El sistema verifica que la hora actual es anterior a 10 minutos antes de la hora de inicio del ETS.
	\UCpaso El sistema muestra el mensaje: \textbf{\hyperref[msg:CU08-D1]{MSG-CU08-D1}}{ ``Aún no es periodo para crear los reportes. Faltan (tiempo)''}.
	\UCpaso El botón con la boleta y nombre del alumno se muestra deshabilitado o no interactivo.
	\UCpaso[\UCactor] El docente visualiza el mensaje.
	\UCpaso Fin de la trayectoria alternativa.
\end{UCtrayectoriaA}
%--------------------------------------
\begin{UCtrayectoriaA}{E}{El periodo para registrar los reportes ha concluido}
	\UCpaso El sistema verifica que la hora actual es posterior a 2 horas después de la hora de inicio del ETS.
	\UCpaso El sistema muestra el mensaje: \textbf{\hyperref[msg:CU08-ET2]{MSG-CU08-ET2}}{ ``El periodo para registrar los reportes ha concluido.''}.
	\UCpaso El botón con la boleta y nombre del alumno se muestra deshabilitado o no interactivo.
	\UCpaso[\UCactor] El docente visualiza el mensaje.
	\UCpaso Fin de la trayectoria alternativa.
\end{UCtrayectoriaA}

%-------------------------------------- TERMINA descripción del caso de uso.

\newpage

% \IUref{IUAdmPS}{Administrar Planta de Selección}
% \IUref{IUModPS}{Modificar Planta de Selección}
% \IUref{IUEliPS}{Eliminar Planta de Selección}

%-------------------------------------- COMIENZA descripción del caso de uso.

%\begin{UseCase}[archivo de imágen]{UCX}{Nombre del Caso de uso}{
%--------------------------------------
\begin{UseCase}{CU-09}{Tomar asistencias a los ETS}{
		Este caso de uso permite al docente registrar la asistencia de los alumnos inscritos a un ETS asignado.
	}
	\UCitem{Versión}{\color{Gray}1.0}
	\UCitem{Autor}{\color{Gray}De la cruz De la cruz Alejandra}
	\UCitem{Supervisa}{\color{Gray}Huertas Ramírez Daniel Martín}
	\UCitem{Actor}{\hyperlink{PersonalAcademico}{Docente}}
	\UCitem{Propósito}{Permitir al docente registrar la asistencia de los alumnos que estén inscritos a un ETS.}
	\UCitem{Entradas}{-}
	\UCitem{Origen}{Pantalla táctil}
	\UCitem{Salidas}{Confirmación del registro de asistencia de los alumnos.}
	\UCitem{Destino}{\IUref{IU08}{Lista de asistencia de ETS.}}
	\UCitem{Precondiciones}{El docente debe estar autenticado y asignado al ETS correspondiente.}
	\UCitem{Postcondiciones}{La asistencia de los alumnos ha sido registrada en el sistema para el ETS.}
	\UCitem{Errores}{
		\begin{itemize}
			\item El ETS seleccionado no tiene alumnos inscritos y se muestra el mensaje {\bf MSG-16}{``No hay alumnos inscritos en este ETS.''}
			\item El sistema pierde la conexión al intentar registrar la asistencia y se muestra el mensaje {\bf MSG-9}{``Error al consultar la base de datos. Intente nuevamente más tarde.''}
			\item El sistema no logro activar la camara y se muestra el mensaje {\bf MSG-17}{``No se pudo activar la cámara o reconocer la identidad. Intente nuevamente.''}
		\end{itemize}
	}
	\UCitem{Tipo}{Se entiende del CU-10 Consultar lista de asistencia de alumnos inscritos a los ETS}
	\UCitem{Observaciones}{}
\end{UseCase}
%--------------------------------------
\begin{UCtrayectoria}
	\UCpaso[\UCactor] El docente presiona el recuadro con la información del alumno que desee desde la pantalla \IUref{IU08}{Lista de asistencia de ETS}.
	\UCpaso El sistema le muestra una imagen del alumno al docente.
	\UCpaso[\UCactor] El docente no esta seguro de la identidad del alumno, por lo que decide usar el reconocimiento facial, accediendo a la pantalla \IUref{IU17}{Pantalla Reconocimiento facial} presionando el texto de \IUbutton{Estatus}. \Trayref{A} \Trayref{B}
	\UCpaso El sistema activa la cámara y comienza el proceso de reconocimiento facial de los alumnos presentes \IUref{IU17}{Pantalla Reconocimiento facial}. \Trayref{C} \Trayref{D}
	\UCpaso El sistema analiza la imagen de cada alumno y muestra un indicador:
	\begin{itemize}
		\item Verde: El sistema está casi seguro de que la persona es quien dice ser y muestra las características coincidentes.
		\item Amarillo: El sistema no está seguro y necesita la ayuda del docente para confirmar la identidad, mostrando tanto las características coincidentes como las que no coinciden. \Trayref{E}
		\item Rojo: El sistema está casi seguro de que la persona no es quien dice ser y muestra las características que no coinciden. \Trayref{F}
	\end{itemize}
	\UCpaso[\UCactor] El docente revisa las características del alumno y decide si confirma la identidad del alumno cuando el indicador marque el color Amarillo. 
	\UCpaso El sistema marca la asistencia de cada alumno.
	\UCpaso[\UCactor] El docente confirma el registro de asistencia.
	\UCpaso El sistema guarda la asistencia y muestra un mensaje de confirmación: {\bf MSG-15}{``Asistencia registrada exitosamente.''}
	\UCpaso El sistema muestra la lista de asistencia de los alumnos en la \IUref{IU08}{Pantalla Lista de asistencia de ETS.}
\end{UCtrayectoria}
%--------------------------------------        
\begin{UCtrayectoriaA}{A}{El Docente está seguro de la identidad del alumno sin la necesidad de usar el reconocimiento facial y decide que le registrara asistencia sin la necesidad del reconocimiento facial}
	\UCpaso[\UCactor] El docente presiona el botón \IUbutton{Registrar asistencia}.
	\UCpaso El sistema actualiza la asistencia.
	\UCpaso Fin de la trayectoria alternativa.
\end{UCtrayectoriaA}
%--------------------------------------        
\begin{UCtrayectoriaA}{B}{El Docente no está seguro de la identidad del alumno sin la necesidad de usar el reconocimiento facial y decide que no le registrara asistencia}
	\UCpaso[\UCactor] El docente presiona el botón \IUbutton{No registrar asistencia}.
	\UCpaso El sistema no actualiza la asistencia.
	\UCpaso Fin de la trayectoria alternativa.
\end{UCtrayectoriaA}        
%--------------------------------------        
\begin{UCtrayectoriaA}{C}{El ETS no tiene alumnos inscritos}
	\UCpaso El sistema muestra un mensaje: {\bf MSG-16}{``No hay alumnos inscritos en este ETS.''}
	\UCpaso[\UCactor] El docente presiona el botón \IUbutton{Regresar} para volver a la pantalla anterior.
	\UCpaso Fin de la trayectoria alternativa.
\end{UCtrayectoriaA}
%--------------------------------------        
\begin{UCtrayectoriaA}{D}{Error en la conexión con la base de datos}
	\UCpaso El sistema muestra un mensaje de error: {\bf MSG-9}{``Error al consultar la base de datos. Intente nuevamente más tarde.''}
	\UCpaso[\UCactor] El docente presiona el botón \IUbutton{Aceptar} para cerrar el mensaje.
	\UCpaso[\UCactor] El docente puede intentar registrar la asistencia.
	\UCpaso Fin de la trayectoria alternativa.
\end{UCtrayectoriaA}
%--------------------------------------        
\begin{UCtrayectoriaA}{E}{El sistema detecta incertidumbre en la identidad de un alumno}
	\UCpaso El sistema detecta las características coincidentes como las que no coinciden con el alumno.
	\UCpaso[\UCactor] El docente revisa la información proporcionada y toma una decisión sobre la identidad del alumno.
	\UCpaso[\UCactor] El docente confirma o corrige la asistencia.
	\UCpaso El sistema actualiza la asistencia según la confirmación del docente.
	\UCpaso El sistema muestra la lista de asistencia de los alumnos actualizada en la \IUref{IU08}{Pantalla Lista de asistencia de ETS}.
	\UCpaso Fin de la trayectoria alternativa.
\end{UCtrayectoriaA}
%--------------------------------------        
\begin{UCtrayectoriaA}{F}{El sistema identifica que el alumno no coincide con la foto registrada}
	\UCpaso El sistema muestra al docente las características que no coinciden.
	\UCpaso[\UCactor] El docente revisa las discrepancias y decide marcar al alumno como ausente o realizar una verificación adicional.
	\UCpaso El sistema actualiza la asistencia.
	\UCpaso El sistema muestra la lista de asistencia de los alumnos actualizada en la \IUref{IU08}{Pantalla Lista de asistencia de ETS}.
	\UCpaso Fin de la trayectoria alternativa.
\end{UCtrayectoriaA}

%-------------------------------------- TERMINA descripción del caso de uso.
\newpage

% \IUref{IUAdmPS}{Administrar Planta de Selección}
% \IUref{IUModPS}{Modificar Planta de Selección}
% \IUref{IUEliPS}{Eliminar Planta de Selección}

%-------------------------------------- COMIENZA descripción del caso de uso.

%\begin{UseCase}[archivo de imágen]{UCX}{Nombre del Caso de uso}{
%--------------------------------------
\begin{UseCase}{CU-10}{Consultar lista de asistencia de alumnos inscritos a los ETS}{
		Este caso de uso permite al docente visualizar la asistencia de los alumnos inscritos a un ETS asignado.
	}
	\UCitem{Versión}{\color{Gray}1.0}
	\UCitem{Autor}{\color{Gray}De la cruz De la cruz Alejandra}
	\UCitem{Supervisa}{\color{Gray}Huertas Ramírez Daniel Martin}
	\UCitem{Actor}{\hyperlink{Docente}{Docente}}
	\UCitem{Propósito}{Permitir al docente registrar la asistencia de los alumnos que estén inscritos a un ETS.}
	\UCitem{Entradas}{Ninguna}
	\UCitem{Origen}{Pantalla táctil}
	\UCitem{Salidas}{Confirmación del registro de asistencia de los alumnos.}
	\UCitem{Destino}{\IUref{IU08}{Lista de asistencia de ETS.}}
	\UCitem{Precondiciones}{El docente debe estar autenticado y asignado al ETS correspondiente.}
	\UCitem{Postcondiciones}{La asistencia de los alumnos ha sido registrada en el sistema para el ETS.}
	\UCitem{Errores}{
		\begin{itemize}
			\item El ETS seleccionado no tiene alumnos inscritos y se muestra el mensaje {\bf MSG-16}{``No hay alumnos inscritos en este ETS.''}
			\item El sistema pierde la conexión al intentar registrar la asistencia y se muestra el mensaje {\bf MSG-9}{``Error al consultar la base de datos. Intente nuevamente más tarde.''}
		\end{itemize}
	}
	\UCitem{Tipo}{Se entiende del CU-10 consultar lista de asistencia de alumnos inscritos a los ETS}
	\UCitem{Observaciones}{}
\end{UseCase}
%--------------------------------------
\begin{UCtrayectoria}
	\UCpaso[\UCactor] El docente presiona el botón \IUbutton{Tomar asistencia} desde la pantalla \IUref{IU13}{Pantalla Lista de alumno}.
	\UCpaso El sistema activa la cámara y comienza el proceso de reconocimiento facial de los alumnos presentes. \Trayref{A} \Trayref{B}
	\UCpaso El sistema analiza la imagen de cada alumno y muestra un indicador:
	\begin{itemize}
		\item Verde: El sistema está casi seguro de que la persona es quien dice ser y muestra las características coincidentes.
		\item Rojo: El sistema está casi seguro de que la persona no es quien dice ser y muestra las características que no coinciden. \Trayref{D}
		\item Amarillo: El sistema no está seguro y necesita la ayuda del docente para confirmar la identidad, mostrando tanto las características coincidentes como las que no coinciden. \Trayref{C}
	\end{itemize}
	\UCpaso[\UCactor] El docente revisa las características del alumno y decide si confirma la identidad del alumno cuando el indicador marque el color Amarillo. 
	\UCpaso El sistema marca la asistencia de cada alumno.
	\UCpaso[\UCactor] El docente confirma el registro de asistencia.
	\UCpaso El sistema guarda la asistencia y muestra un mensaje de confirmación: {\bf MSG-18}{``Asistencia registrada exitosamente.''}
	\UCpaso El sistema muestra la lista de asistencia de los alumnos en la \IUref{IU08}{Pantalla Lista de asistencia de ETS.}
\end{UCtrayectoria}
%--------------------------------------        
\begin{UCtrayectoriaA}{A}{El ETS no tiene alumnos inscritos}
	\UCpaso El sistema muestra un mensaje: {\bf MSG-17}{``No hay alumnos inscritos en este ETS.''}
	\UCpaso[\UCactor] El docente presiona el botón \IUbutton{Regresar} para volver a la pantalla anterior.
	\UCpaso Fin de la trayectoria alternativa.
\end{UCtrayectoriaA}
%--------------------------------------        
\begin{UCtrayectoriaA}{B}{Error en la conexión con la base de datos}
	\UCpaso[\UCactor] El docente muestra un mensaje de error: {\bf MSG-9}{``Error al consultar la base de datos. Intente nuevamente más tarde.''}
	\UCpaso[\UCactor] El docente presiona el botón \IUbutton{Aceptar} para cerrar el mensaje.
	\UCpaso[\UCactor] El docente puede intentar registrar la asistencia.
	\UCpaso Fin de la trayectoria alternativa.
\end{UCtrayectoriaA}

%-------------------------------------- TERMINA descripción del caso de uso.
\newpage

% !TeX root = ../ejemplo.tex

%--------------------------------------
\label{CU-11}
\begin{UseCase}{CU-11}{Mostrar la foto e información del alumno}{
		Permitir que los docentes puedan revisar el reporte de asistencia del alumno, donde podrán ver información detallada y elementos multimedia asociados al registro de asistencia.
	}
	\UCitem{Versión}{\color{Gray}2.0}
	\UCitem{Autor}{\color{Gray}Huertas Ramírez Daniel Martín}
	\UCitem{Supervisa}{\color{Gray}De la cruz De la cruz alejandra.}
	\UCitem{Actor}{\hyperlink{tDocente}{Docente}}
	\UCitem{Propósito}{Permitir a los docentes revisar los detalles de la asistencia de un alumno específico a un ETS.}
	\UCitem{Entradas}{Selección de un alumno desde la \IUref{IU13}{Pantalla consultar lista de alumnos inscritos a un ETS}.}
	\UCitem{Origen}{\IUref{IU13}{Pantalla consultar lista de alumnos inscritos a un ETS}}
	\UCitem{Salidas}{Información detallada del alumno y su reporte de asistencia, incluyendo (si aplica): foto de la credencial, foto del reconocimiento facial (con precisión), boleta, nombre completo, CURP, carrera, unidad de aprendizaje del ETS, periodo del ETS, turno del ETS, materia del ETS, tipo de ETS, fecha de ingreso, hora de ingreso, nombre del docente aplicador, razón del reporte, motivo del rechazo. Mensajes indicando la ausencia de reporte o de imágenes.}
	\UCitem{Destino}{\IUref{IU20}{Pantalla mostrar la foto e información del alumno}}
	\UCitem{Precondiciones}{El docente debe haber iniciado sesión (\textbf{\hyperref[CU-01]{CU-01 Iniciar sesión del sistema móvil}}) y haber consultado la lista de alumnos inscritos al ETS (\textbf{\hyperref[CU-08]{CU-08 Consultar lista de alumnos inscritos a un ETS}}).}
	\UCitem{Postcondiciones}{El docente ha visualizado la información detallada del alumno seleccionado y su reporte de asistencia.}
	\UCitem{Errores}{
		E1: Cuando se pierde la conexión durante el proceso, los procesos se cancelan y se muestra el mensaje \textbf{\hyperref[msg:CU11-E1]{MSG-CU11-E1}} ``El proceso no se pudo realizar por un fallo de red.''
	}
	\UCitem{Tipo}{Caso de uso primario}
	\UCitem{Observaciones}{En caso de que las imágenes (foto de credencial o reconocimiento facial) no se carguen, se mostrará el texto Sin Imagen en su lugar.}
\end{UseCase}
%--------------------------------------
\begin{UCtrayectoria}
	\UCpaso[\UCactor] El docente selecciona un alumno de la lista en la \IUref{IU13}{Pantalla consultar lista de alumnos inscritos a un ETS}.
	\UCpaso El sistema recupera la información del alumno y su reporte de asistencia. \Trayref{A}, \Trayref{B}
	\UCpaso El sistema muestra en la \IUref{IU20}{Pantalla mostrar la foto e información del alumno}: boleta, nombre completo, CURP, carrera, unidad de aprendizaje del ETS, periodo del ETS, turno del ETS, materia del ETS, tipo de ETS, fecha de ingreso, hora de ingreso, nombre del docente aplicador y razón del reporte.
	\UCpaso Si el reporte es un rechazo, se muestra el motivo del rechazo.
	\UCpaso Si existe una foto de la credencial, se muestra. Si no, se muestra "Sin Imagen".
	\UCpaso Si hubo verificación por reconocimiento facial, se muestra la foto del reconocimiento facial y la precisión. Si no, no se muestra esta información.
	\UCpaso[\UCactor] El docente revisa la información del alumno.
\end{UCtrayectoria}

%--------------------------------------
\begin{UCtrayectoriaA}{A}{No se ha creado reporte para este alumno (dentro del periodo)}
	\UCpaso El sistema verifica que no existe un reporte de asistencia para el alumno seleccionado y que la hora actual no ha superado las 2 horas posteriores al inicio del ETS.
	\UCpaso El sistema muestra el mensaje: \textbf{\hyperref[msg:CU11-A1]{MSG-CU11-A1}} ``No se ha creado reporte para este alumno.''
	\UCpaso[\UCactor] El docente visualiza el mensaje.
	\UCpaso Fin de la trayectoria alternativa.
\end{UCtrayectoriaA}

%--------------------------------------
\begin{UCtrayectoriaA}{B}{El alumno no se presentó al ETS (fuera del periodo)}
	\UCpaso El sistema verifica que no existe un reporte de asistencia para el alumno seleccionado y que la hora actual ha superado las 2 horas posteriores al inicio del ETS.
	\UCpaso El sistema muestra el mensaje: \textbf{\hyperref[msg:CU11-B1]{MSG-CU11-B1}} ``El alumno no se presentó al ETS.''
	\UCpaso[\UCactor] El docente visualiza el mensaje.
	\UCpaso Fin de la trayectoria alternativa.
\end{UCtrayectoriaA}
%-------------------------------------- 





\newpage

% \IUref{IUAdmPS}{Administrar Planta de Selección}
% \IUref{IUModPS}{Modificar Planta de Selección}
% \IUref{IUEliPS}{Eliminar Planta de Selección}

%-------------------------------------- COMIENZA descripción del caso de uso.

%\begin{UseCase}[archivo de imágen]{UCX}{Nombre del Caso de uso}{
%--------------------------------------
\label{CU-12}
\begin{UseCase}{CU-12}{Consultar alumno mediante código QR de la credencial}{
		Este caso de uso permite al personal de seguridad consultar la información de un alumno mediante el escaneo del código QR de su credencial.
	}
	\UCitem{Versión}{\color{Gray}1.0}
	\UCitem{Autor}{\color{Gray}De la cruz De la cruz Alejandra}
	\UCitem{Supervisa}{\color{Gray}Huertas Ramírez Daniel Martín}
	\UCitem{Actor}{\hyperlink{PS}{Personal de Seguridad}}
	\UCitem{Propósito}{Permitir al personal de seguridad acceder a la información del alumno mediante el escaneo del código QR de su credencial.}
	\UCitem{Entradas}{Código QR de la credencial del alumno.}
	\UCitem{Origen}{Cámara de escaneo de QR}
	\UCitem{Salidas}{Información del alumno}
	\UCitem{Destino}{\IUref{IU11}{Pantalla Credencial del alumno}}
	\UCitem{Precondiciones}{El sistema debe tener conectividad con la base de datos y el personal de seguridad debe estar autenticado en el sistema.}
	\UCitem{Postcondiciones}{
		La credencial del alumno es visible
		
		La información del alumno se muestra correctamente
	}
	\UCitem{Errores}{
			E1: No se puede cargar la imagen de la credencial {\bf ``No se pudo cargar la imagen de la credencial''}
			
			E2: No se encontraron datos del alumno {\bf ``No se encontrarón datos''}.
			}
	\UCitem{Tipo}{Se entiende del CU01 Iniciar sesión de personal escolar móvil }
	\UCitem{Observaciones}{Este caso de uso es esencial para validar la identidad de los alumnos al acceder a las instalaciones.}
\end{UseCase}

%--------------------------------------
\begin{UCtrayectoria}
	\UCpaso El sistema activa la cámara para capturar el código QR de la credencial \IUref{IU10}{Pantalla Código QR} después de haber iniciado sesión.
	\UCpaso[\UCactor] El personal de seguridad escanea el codigo QR de la credencial del alumno.
	\UCpaso El sistema recupera la imagen de la credencial asociada al código QR. \Trayref{A}
	\UCpaso El sistema recupera los datos personales y académicos del alumno asociado al código QR. \Trayref{B}
	\UCpaso El sistema recupera la fotografía actual del alumno desde la base de datos.
	\UCpaso Despliega la información del alumno en la \IUref{IU11}{Pantalla Credencial del alumno}.
\end{UCtrayectoria}
%--------------------------------------        
\begin{UCtrayectoriaA}{A}{Error al cargar la imagen de la credencial}
	\UCpaso El sistema detecta que no puede cargar la imagen de la credencial.
	\UCpaso El sistema muestra un mensaje: {\bf ``No se pudo cargar la imagen de la credencial''}.
	\UCpaso El sistema muestra los datos del alumno sin la imagen de la credencial.
\end{UCtrayectoriaA}
%--------------------------------------        
\begin{UCtrayectoriaA}{B}{No se encontraron datos del alumno}
	\UCpaso El sistema detecta que no hay un registro para el alumno.
	\UCpaso El sistema muestra un mensaje: {\bf ``No se encontraron datos del alumno''}.

\end{UCtrayectoriaA}
%-------------------------------------- TERMINA descripción del caso de uso.

\newpage

% \IUref{IUAdmPS}{Administrar Planta de Selección}
% \IUref{IUModPS}{Modificar Planta de Selección}
% \IUref{IUEliPS}{Eliminar Planta de Selección}

%-------------------------------------- COMIENZA descripción del caso de uso.

%\begin{UseCase}[archivo de imágen]{UCX}{Nombre del Caso de uso}{
%--------------------------------------
\begin{UseCase}{CU-13}{Buscar alumno por boleta}{
		Este caso de uso permite al personal de seguridad buscar la información de un alumno utilizando su número de boleta.
	}
	\UCitem{Versión}{\color{Gray}1.0}
	\UCitem{Autor}{\color{Gray}De la cruz De la cruz Alejandra}
	\UCitem{Supervisa}{\color{Gray}Huertas Ramírez Daniel Martín}
	\UCitem{Actor}{\hyperlink{PS}{Personal de Seguridad}}
	\UCitem{Propósito}{Permitir al personal de seguridad acceder a la información del alumno mediante su número de boleta.}
	\UCitem{Entradas}{Número de boleta del alumno.}
	\UCitem{Origen}{Pantalla táctil}
	\UCitem{Salidas}{Información del alumno.}
	\UCitem{Destino}{\IUref{IU12}{Pantalla Buscar alumno por boleta}}
	\UCitem{Precondiciones}{El sistema debe tener conectividad con la base de datos y el personal de seguridad debe estar autenticado en el sistema.}
	\UCitem{Postcondiciones}{El personal de seguridad ha consultado la información del alumno utilizando su número de boleta.}
	\UCitem{Errores}{
			E1: El número de boleta ingresado no corresponde a ningún alumno registrado y se muestra el menssaje {\bf MSG-21}{``número de boleta ingresado no corresponde a ningún alumno registrado''}.
			
			E2: El sistema no puede recuperar la información del alumno y se muestra el mensaje {\bf MSG-9}{``Error al consultar la base de datos. Intente nuevamente más tarde.''}}
	\UCitem{Tipo}{Se entiende del CU01 Iniciar sesión de personal escolar móvil }
	\UCitem{Observaciones}{Este caso de uso es esencial para validar la identidad de los alumnos al acceder a las instalaciones mediante la búsqueda por número de boleta.}
\end{UseCase}

%--------------------------------------
\begin{UCtrayectoria}
	\UCpaso[\UCactor] El personal de seguridad despues de iniciar sesion el personal de seguirdad accede a la \IUref{IUE02}{Pantalla de saludo del personal de seguridad}.
	\UCpaso[\UCactor] El personal de seguridad selecciona la opción \IUbutton{Consultar alumno} y es redirigido a la pantalla \IUref{IU12}{Pantalla Buscar alumno por boleta}"
	\UCpaso[\UCactor] El personal de seguridad ingresa el boleta.
	\UCpaso El sistema verifica el número de boleta y busca en la base de datos la información del alumno correspondiente. \Trayref{A}.
	\UCpaso Despliega la información del alumno en la \IUref{IU12}{Pantalla Buscar alumno}.
\end{UCtrayectoria}
%--------------------------------------        
\begin{UCtrayectoriaA}{A}{Alumno no registrado}
	\UCpaso[\UCactor] El personal de seguridad muestra un mensaje: {\bf MSG-21}{``número de boleta ingresado no corresponde a ningún alumno registrado''}
	\UCpaso[\UCactor] El personal de seguridad presiona el botón \IUbutton{Regresar} para intentar una nueva búsqueda o regresar a la pantalla anterior.
	\UCpaso Fin de la trayectoria alternativa.
\end{UCtrayectoriaA}

%--------------------------------------        
\begin{UCtrayectoriaA}{B}{Error de conexión con la base de datos}
	\UCpaso[\UCactor] El personal de seguridad muestra un mensaje de error: {\bf MSG-9}{``Error al consultar la base de datos. Intente nuevamente más tarde.''}
	\UCpaso[\UCactor] El personal de seguridad presiona el botón \IUbutton{Aceptar} para cerrar el mensaje y puede intentar la consulta nuevamente.
	\UCpaso Fin de la trayectoria alternativa.
\end{UCtrayectoriaA}

%-------------------------------------- TERMINA descripción del caso de uso.

\newpage

% \IUref{IUAdmPS}{Administrar Planta de Selección}
% \IUref{IUModPS}{Modificar Planta de Selección}
% \IUref{IUEliPS}{Eliminar Planta de Selección}

%-------------------------------------- COMIENZA descripción del caso de uso.

%\begin{UseCase}[archivo de imágen]{UCX}{Nombre del Caso de uso}{
%--------------------------------------
\begin{UseCase}{CU-14}{Buscar alumno por nombre}{
		Este caso de uso permite al personal de seguridad buscar la información de un alumno utilizando su nombre.
	}
	\UCitem{Versión}{\color{Gray}1.0}
	\UCitem{Autor}{\color{Gray}De la cruz De la cruz Alejandra}
	\UCitem{Supervisa}{\color{Gray}Huertas Ramírez Daniel Martín}
	\UCitem{Actor}{\hyperlink{Personal de Seguridad}{Personal de Seguridad}}
	\UCitem{Propósito}{Permitir al personal de seguridad acceder a la información del alumno mediante su nombre.}
	\UCitem{Entradas}{\hyperlink{Alumno.Nombre}{Nombre del alumno}}
	\UCitem{Origen}{Pantalla táctil}
	\UCitem{Salidas}{Información del alumno.}
	\UCitem{Destino}{\IUref{IU12}{Pantalla Buscar alumno}}
	\UCitem{Precondiciones}{El sistema debe tener conectividad con la base de datos y el personal de seguridad debe estar autenticado en el sistema.}
	\UCitem{Postcondiciones}{El personal de seguridad ha consultado la información del alumno utilizando su nombre.}
	\UCitem{Errores}{
			E1: El nombre ingresado no corresponde a ningún alumno registrado y muestra el mensaje {\bf MSG-22}{``Alumno no registrado''}.
			E2: El sistema no puede recuperar la información del alumno y muestra el mensaje {\bf MSG-9}{``Error al consultar la base de datos. Intente nuevamente más tarde.''}
	}
	\UCitem{Tipo}{Se entiende del CU01 Iniciar sesión de personal escolar móvil }
	\UCitem{Observaciones}{Este caso de uso es esencial para validar la identidad de los alumnos al acceder a las instalaciones mediante la búsqueda por nombre.}
\end{UseCase}

%--------------------------------------
\begin{UCtrayectoria}
	\UCpaso[\UCactor] El personal de seguridad accede a la pantalla \IUref{IUE02}{Pantalla de saludo del personal de seguridad} después de haber iniciado sesión.
	\UCpaso[\UCactor] El personal de seguridad selecciona la opción \IUbutton{Consultar alumno}"
	\UCpaso[\UCactor] El personal de seguridad ingresa el nombre del alumno en la barra de búsqueda.
	\UCpaso El sistema verifica el nombre y busca en la base de datos la información del alumno correspondiente. \Trayref{A}.
	\UCpaso El sistema despliega la información del alumno en la \IUref{IU12}{Pantalla Buscar alumno}.
\end{UCtrayectoria}
%--------------------------------------        
\begin{UCtrayectoriaA}{A}{Alumno no registrado}
	\UCpaso[\UCactor] El personal de seguridad muestra un mensaje: {\bf MSG-22}{``Alumno no registrado''}
	\UCpaso[\UCactor] El personal de seguridad presiona el botón \IUbutton{Regresar} para intentar una nueva búsqueda o regresar a la pantalla anterior.
	\UCpaso Fin de la trayectoria alternativa.
\end{UCtrayectoriaA}

%--------------------------------------        
\begin{UCtrayectoriaA}{B}{Error de conexión con la base de datos}
	\UCpaso[\UCactor] El personal de seguridad muestra un mensaje de error: {\bf MSG-9}{``Error al consultar la base de datos. Intente nuevamente más tarde.''}
	\UCpaso[\UCactor] El personal de seguridad presiona el botón \IUbutton{Aceptar} para cerrar el mensaje y puede intentar la consulta nuevamente.
	\UCpaso Fin de la trayectoria alternativa.
\end{UCtrayectoriaA}

%-------------------------------------- TERMINA descripción del caso de uso.

\newpage

\begin{UseCase}{CU-15}{Registrar asistencia}{
	Este caso de uso permite al personal de seguridad registrar la entrada de los alumnos mediante el escaneo de credenciales y la búsqueda por nombre o boleta.
}
\UCitem{Versión}{1.0}
\UCitem{Autor}{\color{Gray}De la cruz De la cruz Alejandra}
\UCitem{Supervisa}{\color{Gray}Huertas Ramírez Daniel Martín}
\UCitem{Actor}{\hyperlink{PS}{Personal de Seguridad}}
\UCitem{Propósito}{Facilitar el registro de entrada de los alumnos a las instalaciones.}
\UCitem{Entradas}{Nombre del alumno o número de boleta y Escaneo de la credencial del alumno.}
\UCitem{Origen}{Pantalla táctil y Cámara con lector de códigos QR}
\UCitem{Salidas}{Confirmación de entrada registrada.}
\UCitem{Destino}{Pantalla del sistema.}
\UCitem{Precondiciones}{
		El sistema debe tener acceso a la base de datos de alumnos.
		
		El personal de seguridad debe estar autenticado en el sistema.

}
\UCitem{Postcondiciones}{
		Se registra la entrada del alumno en el sistema.
}
\UCitem{Errores}{
		E1: No se encuentra la información del alumno Y se muestra  {\bf MSG-22}{``Alumno no registrado''}.

		E2: Error de conexión con la base de datos y se muestra {\bf MSG-9}{``Error al consultar la base de datos. Intente nuevamente más tarde.''}.

}
\UCitem{Tipo}{Se entiende del CU-12 Consultar alumno mediante código QR de la credencial, CU-13 Buscar alumno por boleta y del CU-14 Buscar alumno por nombre }
\UCitem{Observaciones}{Ninguno}
\end{UseCase}
%--------------------------------------
\begin{UCtrayectoria}
\UCpaso[\UCactor] El personal de seguridad selecciona al alumno desde la \IUref{IU12}{Pantalla Buscar alumno} o escanea la codigo QR de la credencial del alumno desde la \IUref{IU10}{Pantalla Código QR}.
\UCpaso El sistema obtiene la boleta del alumno.
\UCpaso El sistema verifica la información y busca en la base de datos por boleta. \Trayref{A} \Trayref{B}
\UCpaso El sistema despliega los datos del alumno y solicita confirmación al personal de seguridad para registrar la entrada.
\UCpaso El sistema le muestra una imagen del alumno al personal de seguridad.
\UCpaso[\UCactor] El personal de seguridad no está seguro de la identidad del alumno, por lo que decide usar el reconocimiento facial. \Trayref{C} \Trayref{D}
\UCpaso El sistema activa la cámara y comienza el proceso de reconocimiento facial del alumno presente \IUref{IU17}{Pantalla Reconocimiento facial}. 
\UCpaso El sistema analiza la imagen del alumno y muestra un indicador:
\begin{itemize}
	\item Verde: El sistema está casi seguro de que la persona es quien dice ser y muestra las características coincidentes.
	\item Rojo: El sistema está casi seguro de que la persona no es quien dice ser y muestra las características que no coinciden. 
	\item Amarillo: El sistema no está seguro y necesita la ayuda del docente para confirmar la identidad, mostrando tanto las características coincidentes como las que no coinciden. 
\end{itemize}
\UCpaso[\UCactor] El personal de seguridad decide si confirmar o no el registro de entrada.
\UCpaso El sistema guarda el registro y muestra un mensaje de confirmación: {\bf MSG-23}{``Entrada registrada exitosamente.''} o {\bf MSG-24}{``Entrada no registrada .''} respectivamente.
\end{UCtrayectoria}
%--------------------------------------
\begin{UCtrayectoriaA}{A}{Alumno no registrado o no encontrado}
\UCpaso El sistema muestra un mensaje: {\bf MSG-22}{``Alumno no registrado''}
\UCpaso[\UCactor] El personal de seguridad puede intentar nuevamente el escaneo.
\UCpaso Fin de la trayectoria alternativa.
\end{UCtrayectoriaA}
%--------------------------------------
\begin{UCtrayectoriaA}{B}{Error de conexión con la base de datos}

\UCpaso El sistema muestra un mensaje de error: {\bf MSG-9}{``Error al consultar la base de datos. Intente nuevamente más tarde.''}
\UCpaso[\UCactor] El personal de seguridad presiona el botón \IUbutton{Aceptar} para cerrar el mensaje y puede intentar la consulta nuevamente.
\UCpaso Fin de la trayectoria alternativa.
\end{UCtrayectoriaA}
%--------------------------------------        
\begin{UCtrayectoriaA}{C}{El personal de seguridad está seguro de la identidad del alumno sin la necesidad de usar el reconocimiento facial y decide que le registrara asistencia sin la necesidad del reconocimiento facial}
\UCpaso[\UCactor] El personal de seguridad presiona el botón \IUbutton{Registrar asistencia}.
\UCpaso El sistema actualiza la asistencia.
\UCpaso Fin de la trayectoria alternativa.
\end{UCtrayectoriaA}
%--------------------------------------        
\begin{UCtrayectoriaA}{D}{El personal de seguridad está seguro de la identidad del alumno sin la necesidad de usar el reconocimiento facial y decide que no le registrara asistencia}
\UCpaso[\UCactor] El personal de seguridad presiona el botón \IUbutton{No registrar asistencia}.
\UCpaso El sistema no actualiza la asistencia.
\UCpaso Fin de la trayectoria alternativa.
\end{UCtrayectoriaA}  


\newpage

%% !TeX root = ../ejemplo.tex

%--------------------------------------
\begin{UseCase}{CU-xx}{Iniciar sesión de alumnos}{
		Permitir que alumno pueda acceder al sistema, además de separar completamente las funciones de el alumno y el personal escolar.
	}
	\UCitem{Versión}{\color{Gray}1}
	\UCitem{Autor}{\color{Gray}Huertas Ramírez Daniel Martín}
	\UCitem{Supervisa}{\color{Gray}Ulises Vélez Saldaña.}
	\UCitem{Actor}{\hyperlink{Alumno}{Alumno}}
	\UCitem{Propósito}{Que el alumno pueda acceder al sistema móvil y sus funciones específicas. }
	\UCitem{Entradas}{\hyperlink{Alumno.Boleta}{Boleta}, \hyperlink{Alumno.Contraseña}{Contraseña}}
	\UCitem{Origen}{Teclado}
	\UCitem{Salidas}{Saludo del sistema, mención de su nombre.}
	\UCitem{Destino}{Pantalla \IUref{IUE03}{Pantalla de Menú de alumnos}}
	\UCitem{Precondiciones}{El alumno debe estar registrado en la ESCOM.}
	\UCitem{Postcondiciones}{El alumno accede al sistema y podrá realizar las acciones pertinentes a su cargo.}
	\UCitem{Errores}{
		E1: Cuando falta algún dato requerido entonces el sistema muestra el mensaje {\bf MSG1-}{``Los campos no están correctamente llenados.''}
		
		E2: Cuando la cuenta esta bloqueada el sistema no deja entrar al alumno y muestra el mensaje {\bf MSG2-}``Su cuenta esta bloqueada.''
		
		E3: Cuando la contraseña no corresponde a la boleta ingresado el sistema no permite el acceso al alumno y se muestra el mensaje {\bf MSG6-} ``La boleta o la contraseña no corresponden con ningún alumno.''
		
		E4: Cuando se pierde la conexión durante el proceso, los procesos se cancelan y se muestra el mensaje {\bf MSG4-}  ``El proceso no se pudo realizar por un fallo de red.''
		
		E5: Cuando se intenta iniciar varias veces sesión sin éxito la cuenta es bloqueada por seguridad y se muestra el mensaje {\bf MSG5-}  ``Su cuenta ha sido bloqueada por la gran cantidad de intentos de inicio sesión fallidos''.
	}
	\UCitem{Tipo}{Caso de uso primario}
	\UCitem{Observaciones}{}
\end{UseCase}
%--------------------------------------

\begin{UCtrayectoria}
	\UCpaso[\UCactor] Introduce su boleta y contraseña en el sistema vía la  \IUref{IU13}{Pantalla de Iniciar sesión de alumno escolar móvil}\label{CU16.introduceDatos}.
	\UCpaso[\UCactor] Confirma la operación presionando el botón \IUbutton{Entrar}.
	\UCpaso Verifica que todos los datos requeridos hayan sido capturados.
	\UCpaso Verifica que el alumno este registrado en el sistema.
	\UCpaso Verifica que la cuenta del alumno no este bloqueada.
	\UCpaso Verifica que la contraseña corresponda a la boleta.
	\UCpaso Verifica que tipo acceso tiene este alumno.
	\UCpaso La sesión es iniciada con éxito.
	\UCpaso El alumno es redirigido a la pantalla \IUref{IUE03}{Pantalla de Menú de alumnos}.
	
\end{UCtrayectoria}








\newpage

% \IUref{IUAdmPS}{Administrar Planta de Selección}
% \IUref{IUModPS}{Modificar Planta de Selección}
% \IUref{IUEliPS}{Eliminar Planta de Selección}

%-------------------------------------- COMIENZA descripción del caso de uso.

%\begin{UseCase}[archivo de imágen]{UCX}{Nombre del Caso de uso}{
%--------------------------------------
\begin{UseCase}{CU-16}{Consultar periodos de ETS inscritos del alumno}{
		Este caso de uso permite al alumno consultar los periodos de ETS. 
	}
	\UCitem{Versión}{\color{Gray}1.0}
	\UCitem{Autor}{\color{Gray}De la cruz De la cruz Alejandra}
	\UCitem{Supervisa}{\color{Gray}Huertas Ramírez Daniel Martín}
	\UCitem{Actor}{\hyperlink{Alumno}{Alumno}}
	\UCitem{Propósito}{Permitir al alumno consultar los periodos de ETS.}
	\UCitem{Entradas}{Ninguna}
	\UCitem{Origen}{Pantalla táctil}
	\UCitem{Salidas}{Lista de periodos de ETS.}
	\UCitem{Destino}{\IUref{IU14}{Pantalla Periodo de ETS alumno}}
	\UCitem{Precondiciones}{El alumno debe estar autenticado en el sistema.}
	\UCitem{Postcondiciones}{El alumno ha consultado los periodos de ETS.}
	\UCitem{Errores}{
			E1: El sistema no puede recuperar la información de los periodos y muestra el mensaje {\bf MSG-9}{``Error al consultar la base de datos. Intente nuevamente más tarde''}.
			
			E2:  No hay periodos de ETS y muestra el mensaje  {\bf MSG-25}{``No hay periodos de ETS''}. 
	}
	\UCitem{Tipo}{Se extiende del CU-01 Iniciar sesión del sistema móvil}
	\UCitem{Observaciones}{Ninguna}
\end{UseCase}
%--------------------------------------
\begin{UCtrayectoria}
	\UCpaso[\UCactor] El alumno accede a la \IUref{IUE03}{Pantalla Menú del alumno} después de haber iniciado sesión.
	\UCpaso[\UCactor] El alumno selecciona la opción \IUbutton{Consultar periodo de ETS}.
	\UCpaso El sistema verifica que el alumno cuente con periodos de ETS. \Trayref{A}.
	\UCpaso El sistema busca en la base de datos los periodos de ETS asignados al alumno.
	\UCpaso El sistema despliega la lista de periodos de ETS asignados al alumno en la \IUref{IU14}{Pantalla Periodo de ETS alumno}.
\end{UCtrayectoria}
%--------------------------------------        
\begin{UCtrayectoriaA}{A}{No hay periodos}
	\UCpaso El sistema verifica y no encuentra registros de periodos de ETS.
	\UCpaso El sistema muestra un mensaje: {\bf MSG-25}{``No hay periodos de ETS''}
	\UCpaso[\UCactor] El alumno presiona el botón \IUbutton{Regresar} para volver a la pantalla anterior.
	\UCpaso Fin de la trayectoria alternativa.
\end{UCtrayectoriaA}
%--------------------------------------        
\begin{UCtrayectoriaA}{B}{Error en la conexión con la base de datos}
	\UCpaso El sistema muestra un mensaje de error: {\bf MSG-9}{``Error al consultar la base de datos. Intente nuevamente más tarde.''}
	\UCpaso[\UCactor] El alumno presiona el botón \IUbutton{Aceptar} para cerrar el mensaje.
	\UCpaso[\UCactor] El alumno puede intentar la consulta nuevamente o presionar el botón \IUbutton{Regresar} para volver a la pantalla anterior.
	\UCpaso Fin de la trayectoria alternativa.
\end{UCtrayectoriaA}
%-------------------------------------- TERMINA descripción del caso de uso.
\newpage

% \IUref{IUAdmPS}{Administrar Planta de Selección}
% \IUref{IUModPS}{Modificar Planta de Selección}
% \IUref{IUEliPS}{Eliminar Planta de Selección}

%-------------------------------------- COMIENZA descripción del caso de uso.

%\begin{UseCase}[archivo de imágen]{UCX}{Nombre del Caso de uso}{
%--------------------------------------

\begin{UseCase}{CU-17}{Consultar ETS inscritos}{
		\label{CU-17}
		Este caso de uso permite al alumno consultar los ETS en los que se ha inscrito, con la opción de buscar por nombre.
	}
	\UCitem{Versión}{\color{Gray}2.0}
	\UCitem{Autor}{\color{Gray}De la cruz De la cruz Alejandra}
	\UCitem{Supervisa}{\color{Gray}Huertas Ramírez Daniel Martín}
	\UCitem{Actor}{\hyperlink{Alumno}{Alumno}}
	\UCitem{Propósito}{Permitir al alumno consultar los ETS en los que se ha inscrito, con la posibilidad de filtrar por nombre del ETS.}
	\UCitem{Entradas}{Selecciona el boton \IUbutton{Listado de ETS} en la \IUref{IUE03}{Pantalla de saludo del alumno}.}
	\UCitem{Origen}{Pantalla táctil.}
	\UCitem{Salidas}{Lista de ETS inscritos (con nombre, periodo, fecha, turno) o indicación de que no hay ETS inscritos.}
	\UCitem{Destino}{\IUref{IU15}{Pantalla Consultar ETS del alumno}.}
	\UCitem{Precondiciones}{El alumno debe estar autenticado en el sistema.}
	\UCitem{Postcondiciones}{El alumno ha consultado los ETS inscritos para el periodo seleccionado.}
	\UCitem{Errores}{
		
			E1: Cuando no se puede conectar con la base de datos se muestra el mensaje \textbf{ ``Error al consultar la base de datos. Intente nuevamente más tarde.''}
			
			E2: Cuando no se puede conectar con el servidor se muestra el mensaje \textbf{ ``Conexión perdida.''}
		
	}
	\UCitem{Tipo}{Caso de uso primario}
	\UCitem{Observaciones}{Incluye una barra de búsqueda para filtrar los ETS inscritos por nombre.}
\end{UseCase}
%--------------------------------------
\begin{UCtrayectoria}
	\UCpaso[\UCactor] El alumno selecciona el botón \IUbutton{Listado de ETS} en la \IUref{IUE03}{Pantalla de saludo del alumno}.
	\UCpaso El sistema muestra la \IUref{IU15}{Pantalla Consultar ETS del alumno} con una barra de búsqueda.
	\UCpaso El sistema recupera la lista de todos los ETS inscritos para el periodo seleccionado.
	\UCpaso[\UCactor] Opcionalmente, el alumno puede ingresar un término en la barra de búsqueda para filtrar los ETS por nombre.
	\UCpaso El sistema muestra la lista de ETS inscritos (filtrada por el término de búsqueda, si se proporcionó), incluyendo para cada uno: nombre, periodo, fecha y turno. \Trayref{A}
	\UCpaso[\UCactor] El alumno visualiza la lista de ETS inscritos.
	\UCpaso[\UCactor] El alumno puede seleccionar un ETS de la lista para ver más detalles.
	\UCpaso[\UCactor] Opcionalmente, el alumno puede presionar un botón para ver todos los ETS inscritos (sin filtro).
\end{UCtrayectoria}

%--------------------------------------
\begin{UCtrayectoriaA}{A}{No hay ETS inscritos en el periodo seleccionado}
	\UCpaso El sistema verifica que no hay ETS inscritos para el periodo seleccionado (o que no coinciden con el término de búsqueda).
	\UCpaso El sistema muestra el mensaje \textbf{ ``No hay ETS inscritos.''}
	\UCpaso[\UCactor] El alumno visualiza el mensaje y puede intentar una nueva búsqueda o regresar.
	\UCpaso Fin de la trayectoria alternativa.
\end{UCtrayectoriaA}

%--------------------------------------
\begin{UCtrayectoriaA}{B}{Error en la conexión con la base de datos}
	\UCpaso El sistema intenta recuperar la lista de ETS inscritos.
	\UCpaso Ocurre un error en la conexión con la base de datos.
	\UCpaso El sistema muestra el mensaje de error: \textbf{ ``Error al consultar la base de datos. Intente nuevamente más tarde.''}
	\UCpaso[\UCactor] El alumno presiona el botón \IUbutton{Aceptar} para cerrar el mensaje.
	\UCpaso[\UCactor] El alumno puede intentar la consulta nuevamente o presionar el botón \IUbutton{Regresar} para volver a la pantalla anterior.
	\UCpaso Fin de la trayectoria alternativa.
\end{UCtrayectoriaA}

%--------------------------------------
\begin{UCtrayectoriaA}{C}{Conexión perdida}
	\UCpaso Durante la recuperación de la lista de ETS inscritos, se pierde la conexión.
	\UCpaso El sistema muestra el mensaje de error: \textbf{ ``Conexión perdida.''}
	\UCpaso[\UCactor] El alumno verifica su conexión a internet e intenta nuevamente.
	\UCpaso Fin de la trayectoria alternativa.
\end{UCtrayectoriaA}

%-------------------------------------- TERMINA descripción del caso de uso.

\newpage

% \IUref{IUAdmPS}{Administrar Planta de Selección}
% \IUref{IUModPS}{Modificar Planta de Selección}
% \IUref{IUEliPS}{Eliminar Planta de Selección}

%-------------------------------------- COMIENZA descripción del caso de uso.

%\begin{UseCase}[archivo de imágen]{UCX}{Nombre del Caso de uso}{
%--------------------------------------
\begin{UseCase}{CU-18}{Mostrar información de los ETS inscritos}{
		Este caso de uso permite al alumno visualizar la información detallada de los ETS que tiene inscritos.
	}
	\UCitem{Versión}{\color{Gray}1.0}
	\UCitem{Autor}{\color{Gray}De la cruz De la cruz Alejandra}
	\UCitem{Supervisa}{\color{Gray}Huertas Ramírez Daniel Martín}
	\UCitem{Actor}{\hyperlink{Alumno}{Alumno}}
	\UCitem{Propósito}{Permite al alumno visualizar la información detallada de cada ETS que tiene inscrito.}
	\UCitem{Entradas}{Pantalla táctil}
	\UCitem{Origen}{Seleccionar un ETS inscrito}
	\UCitem{Salidas}{Detalle de la información de los ETS inscritos}
	\UCitem{Destino}{\IUref{IU16}{Pantalla Información de ETS del alumno}}
	\UCitem{Precondiciones}{El alumno debe estar autenticado y tener ETS inscritos en el sistema.}
	\UCitem{Postcondiciones}{El alumno ha visualizado la información detallada de sus ETS inscritos.}
	\UCitem{Errores}{
			E1: El sistema no puede recuperar la información detallada de los ETS y se muestra {\bf MSG-9}{``Error al consultar la base de datos. Intente nuevamente más tarde.''}.
			
			E2: No hay ETS inscritos y se muestra el mensaje {\bf MSG-27}{``Información no disponible para el ETS seleccinado''}.

	}
	\UCitem{Tipo}{Extiende de CU-17 Consultar periodos de ETS inscritos del alumno}
	\UCitem{Observaciones}{Ninguna}
\end{UseCase}
%--------------------------------------
\begin{UCtrayectoria}
	\UCpaso[\UCactor] El alumno selecciona el ETS que desea visualizar desde la \IUref{IU15}{Pantalla Consultar ETS del alumno}.
	\UCpaso[\UCsist] El alumno verifica si existen detalles disponibles para el ETS seleccionado. \Trayref{A}
	\UCpaso El sistema despliega la información detallada de cada ETS asignado en la \IUref{IU16}{Pantalla Información de ETS del alumno}.
\end{UCtrayectoria}
%--------------------------------------        
\begin{UCtrayectoriaA}{A}{No hay detalles disponibles para el ETS seleccionado}
	\UCpaso El sistema muestra un mensaje: {\bf MSG-27}{``Información no disponible para el ETS seleccinado''}
	\UCpaso[\UCactor] El alumno presiona el botón \IUbutton{Regresar} para volver a la lista de ETS.
	\UCpaso Fin de la trayectoria alternativa.
\end{UCtrayectoriaA}

%--------------------------------------        
\begin{UCtrayectoriaA}{B}{Error en la conexión con la base de datos}
	\UCpaso El sistema muestra un mensaje de error {\bf MSG-9}{``Error al consultar la base de datos. Intente nuevamente más tarde.''}
	\UCpaso[\UCactor] El alumno presiona el botón \IUbutton{Aceptar} para cerrar el mensaje.
	\UCpaso[\UCactor] El alumno puede intentar la consulta nuevamente o presionar el botón \IUbutton{Regresar} para volver a la pantalla anterior.
	\UCpaso Fin de la trayectoria alternativa.
\end{UCtrayectoriaA}

%-------------------------------------- TERMINA descripción del caso de uso.

\newpage

%-------------------------------------- COMIENZA descripción del caso de uso.

\label{CU-19}
\begin{UseCase}{CU-19}{Probar reconocimiento facial}{
		Permitir que el alumno pruebe la funcionalidad de reconocimiento facial para asegurar que su rostro sea reconocido correctamente por el sistema.
	}
	\UCitem{Versión}{\color{Gray}2.0}
	\UCitem{Autor}{\color{Gray}De la cruz De la cruz Alejandra}
	\UCitem{Supervisa}{\color{Gray}Huertas Ramírez Daniel Martín}
	\UCitem{Actor}{\hyperlink{Alumno}{Alumno}}
	\UCitem{Propósito}{Permitir al alumno verificar que el sistema de reconocimiento facial funciona correctamente con su rostro.}
	\UCitem{Entradas}{Selección del botón \IUbutton{Probar reconocimiento facial} en la \IUref{IUE03}{Pantalla de saludo del alumno}.}
	\UCitem{Origen}{Interacción del alumno con la \IUref{IUE03}{Pantalla de saludo del alumno}.}
	\UCitem{Salidas}{Retroalimentación visual en la \IUref{IU19}{Pantalla Reconocimiento facial alumno} indicando la precisión del reconocimiento y una imagen del rostro capturado (si aplica). Mensaje de error si la cámara no se activa o falla el reconocimiento.}
	\UCitem{Destino}{\IUref{IU19}{Pantalla Reconocimiento facial alumno}.}
	\UCitem{Precondiciones}{El alumno ha iniciado sesión en la aplicación móvil (\textbf{\hyperref[CU-02]{CU-02 Iniciar sesión del alumno}}).}
	\UCitem{Postcondiciones}{El alumno ha visualizado el resultado de la prueba de reconocimiento facial.}
	\UCitem{Errores}{
		
			E1: Cuando no se puede capturar la fotografía del alumno se muestra el mensaje \textbf{ ´´Error al capturar la fotografía.´´}
			
			E2: Cuando hay un fallo en el reconocimiento facial se muestra el mensaje \textbf{ ´´Error al realizar el reconocimiento facial.´´}
			 
			E3: Cuando falla la conexión se muestra el mensaje \textbf{ ´´Error de conexión.´´}
		
	}
	\UCitem{Tipo}{Caso de uso primario}
	\UCitem{Observaciones}{La precisión del reconocimiento facial se mostrará al alumno.}
\end{UseCase}
%--------------------------------------
\begin{UCtrayectoria}
	\UCpaso[\UCactor] Selecciona el botón \IUbutton{Probar reconocimiento facial } desde la pantalla \IUref{IUE03}{Pantalla de saludo del alumno}.
	\UCpaso El sistema activa la cámara del dispositivo y muestra la \IUref{IU19}{Pantalla Reconocimiento facial alumno} con una vista previa de la cámara. \Trayref{A}, \Trayref{B}, \Trayref{C}
	\UCpaso[\UCactor] El alumno se posiciona frente a la cámara y presiona el botón \IUbutton{Probar}.
	\UCpaso El sistema captura la imagen del rostro del alumno y la envía para el reconocimiento facial.
	\UCpaso El sistema recibe el resultado del reconocimiento facial (precisión).
	\UCpaso El sistema muestra en la \IUref{IU19}{Pantalla Reconocimiento facial alumno} la precisión del reconocimiento:
	\begin{itemize}
		\item Si la precisión es alta (>= 80\%), se muestra un mensaje indicando que el reconocimiento fue exitoso y la precisión. 
		\item Si la precisión es media (>= 60\% y < 80\%), se muestra un mensaje indicando que la identidad podría ser dudosa y la precisión.
		\item Si la precisión es baja (< 60\%), se muestra un mensaje indicando que no se encontró una coincidencia y la precisión.
	\end{itemize}
	\UCpaso[\UCactor] El alumno revisa el resultado.
\end{UCtrayectoria}
%--------------------------------------
\begin{UCtrayectoriaA}{A}{Error al capturar la fotografía}
	\UCpaso El sistema intenta activar la cámara y capturar la fotografía, pero ocurre un error.
	\UCpaso El sistema muestra un mensaje de error: \textbf{``Error al capturar la fotografía: [detalle del error]''.}
	\UCpaso[\UCactor] El alumno puede intentar nuevamente.
	\UCpaso Fin de la trayectoria alternativa.
\end{UCtrayectoriaA}
%--------------------------------------
\begin{UCtrayectoriaA}{B}{Error al realizar el reconocimiento facial}
	\UCpaso El sistema captura la fotografía, pero ocurre un error al realizar el reconocimiento facial.
	\UCpaso El sistema muestra un mensaje de error: \textbf{ ``Error al realizar el reconocimiento facial: [detalle del error]''.}
	\UCpaso[\UCactor] El alumno puede intentar nuevamente.
	\UCpaso Fin de la trayectoria alternativa.
\end{UCtrayectoriaA}
%--------------------------------------
\begin{UCtrayectoriaA}{C}{Error de conexión}
	\UCpaso Ocurre un error al intentar conectar con el servidor para el reconocimiento facial.
	\UCpaso El sistema muestra un mensaje de error: \textbf{ ``Error de conexión.''}
	\UCpaso[\UCactor] El alumno verifica su conexión a internet o intenta nuevamente.
	\UCpaso Fin de la trayectoria alternativa.
\end{UCtrayectoriaA}
%--------------------------------------
\begin{UCtrayectoriaA}{C1}{Reconocimiento Facial Exitoso (>= 80\%)}
	\UCpaso El sistema muestra: ``Es casi seguro que el alumno es quien dice ser. Precisión del reconocimiento facial: [precisión]\%''
	\UCpaso Fin de la trayectoria alternativa.
\end{UCtrayectoriaA}
%--------------------------------------
\begin{UCtrayectoriaA}{C2}{Identidad Dudosa (>= 60\% y < 80\%)}
	\UCpaso El sistema muestra: ``Es dudosa la identidad del alumno. la precisión del reconocimiento facial: [precisión]\%''
	\UCpaso Fin de la trayectoria alternativa.
\end{UCtrayectoriaA}
%--------------------------------------
\begin{UCtrayectoriaA}{C3}{No Coincidencia (< 60\%)}
	\UCpaso El sistema muestra: ``El casi seguro que el alumno no es quien dice ser. Precisión del reconocimiento facial: menor al 60\%''
	\UCpaso Fin de la trayectoria alternativa.
\end{UCtrayectoriaA}
%--------------------------------------
\begin{UCtrayectoriaA}{C4}{Error en el Reconocimiento Facial (Detallado)}
	\UCpaso El sistema realiza el reconocimiento facial pero ocurre un error específico.
	\UCpaso El sistema muestra: \textbf{ ``Error al realizar el reconocimiento facial: [detalle del error]''.}
	\UCpaso Fin de la trayectoria alternativa.
\end{UCtrayectoriaA}
%-------------------------------------- TERMINA descripción del caso de uso.


\newpage

%-------------------------------------- COMIENZA descripción del caso de uso.

\label{CU-20}
\begin{UseCase}{CU-20}{Revisar información de acceso a los ETS}{
		Este caso de uso permite al alumno visualizar la información detallada del proceso para la presentación de los ETS.
	}
	\UCitem{Versión}{\color{Gray}2.0}
	\UCitem{Autor}{\color{Gray}De la cruz De la cruz Alejandra}
	\UCitem{Supervisa}{\color{Gray}Huertas Ramírez Daniel Martín}
	\UCitem{Actor}{\hyperlink{tAlumno}{Alumno}}
	\UCitem{Propósito}{Permitir al alumno consultar los detalles del proceso para presentar su ETS.}
	\UCitem{Entradas}{Selecciona el botón \IUbutton{Información de Acceso} en la \IUref{IUE03}{Pantalla saludo del alumno}.}
	\UCitem{Origen}{Pantalla táctil.}
	\UCitem{Salidas}{Detalles del proceso para la presentación de ETS.}
	\UCitem{Destino}{\IUref{IU18}{Pantalla Detalles del Proceso de ETS}.}
	\UCitem{Precondiciones}{El alumno debe estar autenticado en el sistema y tener ETS asignados.}
	\UCitem{Postcondiciones}{El alumno ha visualizado la información detallada del proceso para presentar su ETS.}
	\UCitem{Errores}{
		\begin{itemize}
		\item \textbf {\hyperref[msg:CU20-E1]{MSG-CU20-E1}}{``Error al recuperar la información del proceso. Intente nuevamente más tarde.''}.
		\end{itemize}
	}
	\UCitem{Tipo}{Caso de uso primario}
	\UCitem{Observaciones}{Entre la información mostrada estan documentos necesarios, horarios, ubicación y cualquier otra instrucción relevante para el día del ETS.}
\end{UseCase}
%--------------------------------------
\begin{UCtrayectoria}
	\UCpaso[\UCactor] El alumno selecciona el botón \IUbutton{Información de Acceso} desde la \IUref{IUE03}{Pantalla saludo del alumno}.
	\UCpaso El sistema verifica que exista información disponible sobre el proceso para presentar ETS. \Trayref{A}
	\UCpaso El sistema muestra la información detallada del proceso para presentar su ETS en la \IUref{IU18}{Pantalla Detalles del Proceso de ETS}.
	\UCpaso[\UCactor] El alumno visualiza la información.
\end{UCtrayectoria}
%--------------------------------------
\begin{UCtrayectoriaA}{A}{Error al querer mostrar la información}
	\UCpaso El sistema intenta recuperar la información del proceso.
	\UCpaso Ocurre un error al acceder a la información.
	\UCpaso El sistema muestra el mensaje de error: \textbf{\hyperref[msg:CU20-E1]{MSG-CU20-E1}}{``Error al recuperar la información del proceso. Intente nuevamente más tarde.''}
	\UCpaso[\UCactor] El alumno presiona el botón \IUbutton{Aceptar} para cerrar el mensaje.
	\UCpaso[\UCactor] El alumno puede intentar acceder a la información nuevamente o presionar el botón \IUbutton{Regresar} para volver a la pantalla principal.
	\UCpaso Fin de la trayectoria alternativa.
\end{UCtrayectoriaA}





\newpage

% !TeX root = ../ejemplo.tex

%--------------------------------------
\begin{UseCase}{CU-21}{Dar de alta un alumno}{

    Permitir al personal de la DAE dar de alta un alumno.
}
    \UCitem{Versión}{\color{Gray}1}
    \UCitem{Autor}{\color{Gray}Huertas Ramírez Daniel Martín}
    \UCitem{Supervisa}{\color{Gray}De la cruz De la cruz Alejandra.}
    \UCitem{Actor}{Personal de la DAE}
    \UCitem{Propósito}{ Permitir al personal de la DAE dar de alta un alumno.}
    \UCitem{Entradas}{Video, CURP, boleta, nombre, apellido paterno, apellido materno, sexo, escuela asignada, correo institucional y carrera.}
    \UCitem{Origen}{Teclado y cámara}
    \UCitem{Salidas}{Muestra mensaje {\bf MSG-31} ``Alumno dado de alta con éxito''.}
    \UCitem{Destino}{-}
    \UCitem{Precondiciones}{El Personal de la DAE debe de haber iniciado sesión.}
    \UCitem{Postcondiciones}{-}
    \UCitem{Errores}{ 

        E2: Cuando falte algún dato requerido entonces el sistema muestra el mensaje {\bf MSG-29}{``Los campos no están correctamente llenados.''}

        E3: Cuando la CURP o la boleta del alumno ya estén registradas en el sistema muestra el mensaje {\bf MSG-30}{``La CURP o la boleta ya han sido asociadas a este alumno con anterioridad u otro alumno.''}

    }
    \UCitem{Tipo}{ Extiende de CU41 Iniciar sesión de personal escolar web}
    \UCitem{Observaciones}{Ninguna}

\end{UseCase}
%-------------------------------------- 

\begin{UCtrayectoria}

    \UCpaso[\UCactor] El Personal de la DAE accede a la pantalla \IUref{IU21}{ Dar de alta un alumno }\label{CU21.introduceDatos} desde cualquiera de las pantallas del personal de la DAE (\IUref{IU22}{ Consultar alumnos }, \IUref{IUE04}{ Pantalla inicial de personal de la DAE}) apretando el botón \IUbutton{Dar de alta alumnos } e introduce los datos de alumno (Video, CURP, boleta, nombre, apellido paterno, apellido materno, sexo, escuela asignada, correo institucional y carrera) .
    
    \UCpaso[\UCactor] El Personal de la DAE oprime el botón \IUbutton{Guardar}.
    
    \UCpaso El sistema revisa que los datos del alumno sean válidos.
    \UCpaso El sistema verifica que la CURP o la boleta no hayan sido registrados con anterioridad.
    \UCpaso El sistema separa el video en 15 fotos y las guarda.
    \UCpaso El Sistema le muestra el mensaje {\bf MSG-31} {``Alumno dado de alta con éxito''}.

\end{UCtrayectoria}

\newpage

% !TeX root = ../ejemplo.tex

%--------------------------------------
\begin{UseCase}{CU-22}{Crear credencial}{

    Permitir al personal de la DAE previsualizar la credencial del alumno que acaba de dar de alta y si se da el caso corregir los datos.
}
    \UCitem{Versión}{\color{Gray}1}
    \UCitem{Autor}{\color{Gray}Huertas Ramírez Daniel Martín}
    \UCitem{Supervisa}{\color{Gray}De la cruz De la cruz Alejandra.}
    \UCitem{Actor}{Personal de la DAE}
    \UCitem{Propósito}{ Permitir al personal de la DAE previsualizar la credencial del alumno que acaba de dar de alta y si se da el caso corregir los datos..}
    \UCitem{Entradas}{ Boleta, Nombre, CURP, Sexo y Correo institucional}
    \UCitem{Origen}{Teclado}
    \UCitem{Salidas}{Ninguna}
    \UCitem{Destino}{Pantalla \IUref{IU23}{ Capturar fotografía estudiantil} }
    \UCitem{Precondiciones}{El Personal de la DAE debe de haber iniciado sesión y este debe de haber dado de alta un alumno con anterioridad}
    \UCitem{Postcondiciones}{El alumno es dado de alta por el Personal de la DAE y está esperando por su credencial escolar.}
    \UCitem{Errores}{

        E1: Cuando se pierde la conexión durante el proceso, los procesos se cancelan y se muestra el mensaje {\bf MSG-28}  ``El proceso no se pudo realizar por un falló de red.''

        E2: Cuando falte algún dato requerido entonces el sistema muestra el mensaje {\bf MSG-29-}{``Los campos no están correctamente llenados.''}

        E3: Cuando la CURP o la boleta del alumno ya estén registrados el sistema muestra el mensaje {\bf MSG-30-}{``La CURP o la boleta ya han sido asociadas a este alumno con anterioridad u otro alumno.''}
    }
    \UCitem{Tipo}{ Extiende de CU21 Dar de alta a alumno}
    \UCitem{Observaciones}{Ninguna}

\end{UseCase}
%-------------------------------------- 

\begin{UCtrayectoria}
    \UCpaso[\UCactor] El Personal de la DAE accede a la pantalla \IUref{IU22}{Crear credencial }\label{CU22.introduceDatos} apretando el botón \IUbutton{Dar de alta alumno } desde la pantalla \IUref{IU21}{ Dar de alta a alumno }
    \UCpaso El sistema muestra cómo se vería la credencial del alumno que el personal de la DAE dio de alta 
    \UCpaso[\UCactor] El Personal de la DAE revisa que los datos sean correctos \Trayref{A}.
    \UCpaso[\UCactor] El Personal de la DAE selecciona el botón \IUbutton{Subir foto }.
    \UCpaso El sistema revisa que los datos del alumno sean válidos.
    \UCpaso El sistema verifica que el CURP o la boleta no hayan sido registrados con anterioridad.
    \UCpaso El sistema mantiene los datos para usarlos en el proceso de crear credencial.
\UCpaso[\UCactor] El Personal de la DAE es redirigido a la pantalla \IUref{IU23}{ Capturar fotografía estudiantil }.
\end{UCtrayectoria}

\begin{UCtrayectoriaA}{A}{El personal de la DAE se da cuenta que se equivocó en un dato al momento de dar de alta un alumno }
    \UCpaso[\UCactor] El Personal de la DAE modifica alguno de los siguientes datos del alumno: (Boleta, Nombre, CURP, Sexo o Correo institucional) .
    \UCpaso[\UCactor] El Personal de la DAE selecciona el botón \IUbutton{Subir foto }.
    \UCpaso El sistema revisa que los datos del alumno sean válidos.
    \UCpaso El sistema verifica que el CURP o la boleta no hayan sido registrados con anterioridad.
    \UCpaso El sistema mantiene los datos para usarlos en el proceso de crear credencial.
    \UCpaso[\UCactor] El Personal de la DAE es redirigido a la pantalla \IUref{IU23}{ Capturar fotografía estudiantil }.
\end{UCtrayectoriaA}



\newpage

% !TeX root = ../ejemplo.tex

%--------------------------------------
\begin{UseCase}{CU-23}{Capturar fotografía estudiantil}{

    Permitir al personal de la DAE Capturar 5 fotos del alumno donde todas serán guardadas en la base de datos para alimentar el modelo de reconocimiento facial y la primera será usada para la credencial.
}
    \UCitem{Versión}{\color{Gray}1}
    \UCitem{Autor}{\color{Gray}Huertas Ramírez Daniel Martín}
    \UCitem{Supervisa}{\color{Gray}De la cruz De la cruz Alejandra.}
    \UCitem{Actor}{\hyperlink{Personal de la DAE}{Personal de la DAE}}
    \UCitem{Propósito}{Obtener una foto para la credencial y obtener fotos para el modelo de reconocimiento facial.}
    \UCitem{Entradas}{Ninguna}
    \UCitem{Origen}{Teclado}
    \UCitem{Salidas}{Ninguna}
    \UCitem{Destino}{Pantalla \IUref{IUE04}{saludo de personal de la DAE}}
    \UCitem{Precondiciones}{El Personal de la DAE debe de haber iniciado sesión y este debe de haber dado de alta un alumno con anterioridad}
    \UCitem{Postcondiciones}{El alumno es dado de alta por el Personal de la DAE y está esperando por su credencial escolar.}
    \UCitem{Errores}{

        E1: Cuando se pierde la conexión durante el proceso, los procesos se cancelan y se muestra el mensaje {\bf MSG-28}  ``El proceso no se pudo realizar por un fallo de red.''
    }
    \UCitem{Tipo}{ Extiende de CU22 Crear credencial}
    \UCitem{Observaciones}{Ninguna}

\end{UseCase}
%-------------------------------------- 

\begin{UCtrayectoria}
    \UCpaso[\UCactor] El Personal de la DAE accede a la pantalla \IUref{IU23}{Capturar fotografía estudiantil}\label{CU23.introduceDatos} apretando el botón \IUbutton{Subir foto} desde la pantalla \IUref{IU22}{Crear credencial}
    \UCpaso[\UCactor] El Personal de la DAE apunta la cámara hacia el alumno y usa el botón \IUbutton{Cámara}.
    \UCpaso El sistema toma 5 fotos cuando el alumno este mirando de frente
    \UCpaso El sistema da de alta al alumno y su credencial.
    \UCpaso El sistema guarda las fotos en la base de datos.
\UCpaso[\UCactor] El Personal de la DAE es redirigido a la pantalla \IUref{IUE04}{ Menú de personal de la DAE}.
\end{UCtrayectoria}


\newpage

% !TeX root = ../ejemplo.tex

%--------------------------------------
\begin{UseCase}{CU-24}{Consultar lista de periodo de ETS }{

    Permitir al Personal de gestión escolar consultar la lista de los periodos de ETS.
}
    \UCitem{Versión}{\color{Gray}1}
    \UCitem{Autor}{\color{Gray}Huertas Ramírez Daniel Martín}
    \UCitem{Supervisa}{\color{Gray}De la cruz De la cruz Alejandra.}
    \UCitem{Actor}{Personal de gestión escolar}
    \UCitem{Propósito}{Mostrar todos los periodos de ETS dados de alta con su respectiva información.}
    \UCitem{Entradas}{Ninguna}
    \UCitem{Origen}{Mouse}
    \UCitem{Salidas}{Ninguna}
    \UCitem{Destino}{Ninguna}
    \UCitem{Precondiciones}{ El Personal de gestión escolar debe de haber iniciado sesión}
    \UCitem{Postcondiciones}{Ninguna.}
    \UCitem{Errores}{
        E1: Cuando no hay ningún periodo de ETS dado de alta se muestra el mensaje  \bf ``Ningún periodo de ETS ha sido dado de alta.''
    }
    \UCitem{Tipo}{ Extiende de CU42 Iniciar sesión de personal escolar web}
    \UCitem{Observaciones}{Ninguna}

\end{UseCase}
%-------------------------------------- 

\begin{UCtrayectoria}
    \UCpaso[\UCactor] El Personal de gestión escolar accede a la pantalla \IUref{IU24}{Consultar lista de periodo de ETS}\label{CU24.introduceDatos} a través del menú lateral que tiene en todas las pantallas apretando el botón \IUbutton{Consultar lista de periodo de ETS}.
    \UCpaso El sistema muestra la información de todos los periodos de ETS.
    \UCpaso[\UCactor] El Personal de gestión escolar revisa los periodos de ETS.
\end{UCtrayectoria}

\newpage

% !TeX root = ../ejemplo2.tex

%--------------------------------------
\begin{UseCase}{CU-25}{Dar de alta de periodo de ETS}{

    Permitir al personal de gestión escolar dar de alta un periodo de ETS.
 }
     \UCitem{Versión}{\color{Gray}1}
     \UCitem{Autor}{\color{Gray}Huertas Ramírez Daniel Martín}
     \UCitem{Supervisa}{\color{Gray}De la cruz De la cruz Alejandra.}
     \UCitem{Actor}{Personal de gestión escolar}
     \UCitem{Propósito}{ Permitir al personal de gestión escolar dar de alta un nuevo periodo de ETS.}
     \UCitem{Entradas}{ Periodo, Tipo de periodo, Fecha de inicio y Fecha de fin. }
     \UCitem{Origen}{Teclado}
     \UCitem{Salidas}{Muestra mensaje \bf ``Periodo de ETS  dado de alta con éxito''.}
     \UCitem{Destino}{-}
     \UCitem{Precondiciones}{El Personal de gestión escolar debe de haber iniciado sesión.}
     \UCitem{Postcondiciones}{El periodo de ETS es dado de alta y guardado en la base de datos}
     \UCitem{Errores}{
 
        E1: Cuando falta algún dato requerido entonces el sistema muestra el mensaje \bf ``Los campos no están correctamente llenados.''

        E2: Cuando el Periodo, Fecha-de-inicio o Fecha-de-fin ya están registradas en el sistema, el proceso no se realiza y se muestra el mensaje \bf `` Periodo, Fecha-de-inicio o Fecha-de-fin ya han sido asociadas a un periodo de ETS.''
     }
     \UCitem{Tipo}{ Extiende de CU42 Iniciar sesión de personal escolar web. }
     \UCitem{Observaciones}{Ninguna}
 
 \end{UseCase}
 %-------------------------------------- 
 
 \begin{UCtrayectoria}
 
     \UCpaso[\UCactor] El Personal de gestión escolar accede a la pantalla \IUref{IU25}{ Dar de alta de periodo de ETS }\label{CU25.introduceDatos} presionando el botón \IUbutton{Dar de alta periodo} desde el menú lateral, (el cual estará disponible desde cualquier pantalla) e introduce los datos del periodo Periodo, Tipo, Fecha-de-inicio y Fecha-de-fin.
     \UCpaso[\UCactor] El Personal de gestión escolar oprime el botón \IUbutton{ Guardar }.
     \UCpaso El sistema revisa que los datos del periodo sean válidos.
     \UCpaso El sistema verifica que el Periodo, Fecha-de-inicio o Fecha-de-fin no hayan sido registrados con anterioridad.
     \UCpaso El periodo de ETS es dado de alta y guardado en la base de datos.
     \UCpaso El sistema le muestra el mensaje {\bf ``Periodo de ETS  dado de alta con éxito'' al personal de gestión escolar.}
 
 \end{UCtrayectoria}
 
\newpage

% !TeX root = ../ejemplo.tex

%--------------------------------------
\begin{UseCase}{CU-28}{Consultar lista de ETS}{

    Permitir al Personal de gestión escolar consultar la lista de los ETS del periodo de ETS actual.
}
    \UCitem{Versión}{\color{Gray}1}
    \UCitem{Autor}{\color{Gray}De la cruz De la cruz Alejandra}
    \UCitem{Supervisa}{\color{Gray}Huertas Ramírez Daniel Martín.}
    \UCitem{Actor}{Personal de gestión escolar}
    \UCitem{Propósito}{Mostrar todos los ETS dados de alta en el periodo actual y su información específica al personal de gestión escolar.}
    \UCitem{Entradas}{Ninguna}
    \UCitem{Origen}{Teclado}
    \UCitem{Salidas}{Ninguna}
    \UCitem{Destino}{ - }
    \UCitem{Precondiciones}{ El Personal de gestión escolar debe de haber iniciado sesión}
    \UCitem{Postcondiciones}{Ninguna.}
    \UCitem{Errores}{

        E1: Cuando no hay ningún ETS dado de alta se muestra el mensaje \bf   ``Ningún ETS ha sido dado da alta.''
    }
    \UCitem{Tipo}{ Extiende de CU41 Iniciar sesión de personal escolar web }
    \UCitem{Observaciones}{Ninguna}

\end{UseCase}
%-------------------------------------- 

\begin{UCtrayectoria}
\UCpaso[\UCactor] El Personal de gestión escolar accede a la pantalla \IUref{IU26}{Consultar lista de ETS}\label{CU28.introduceDatos}  a través del menú lateral que tiene en todas las pantallas apretando el botón \IUbutton{Consultar lista de ETS}.
\UCpaso El sistema muestra la información de todos los ETS del periodo actual.
\UCpaso[\UCactor] El Personal de gestión escolar revisa los  ETS registrados.
\end{UCtrayectoria}



\newpage

% !TeX root = ../ejemplo.tex

%--------------------------------------
\begin{UseCase}{CU-29}{Dar de alta ETS }{
    Permitir al personal de gestión escolar dar de alta un nuevo ETS.
    
     }
         \UCitem{Versión}{\color{Gray}1}
         \UCitem{Autor}{\color{Gray}Huertas Ramírez Daniel Martín}
         \UCitem{Supervisa}{\color{Gray}De la cruz De la cruz Alejandra.}
         \UCitem{Actor}{Personal de gestión escolar}
         \UCitem{Propósito}{Permitir que el personal de gestión escolar dar de alta un nuevo ETS relacionado con el periodo de ETS actual.}
         \UCitem{Entradas}{ETS, Periodo-de-ETS,  Fecha, Turno, Cupo , Unidad-de-aprendizaje,  Salon y Docente .}
         \UCitem{Origen}{Teclado}
         \UCitem{Salidas}{Muestra mensaje {\bf MSG-35} ``ETS  dado de alta con éxito''.}
         \UCitem{Destino}{ - }
         \UCitem{Precondiciones}{El Personal de gestión escolar debe de haber iniciado sesión.}
         \UCitem{Postcondiciones}{El ETS es dado de alta y guardado en la base de datos}
         \UCitem{Errores}{
            E1: Cuando falta algún dato requerido entonces el sistema muestra el mensaje {\bf MSG-29}{``Los campos no están correctamente llenados.''}
    
            E2: Cuando el dato ETS o salón ya están registradas en el sistema, el proceso no se realiza y se muestra el mensaje {\bf MSG-36}{`` ETS o salón  ya han sido asociadas a un ETS de ETS.''}
         }
         \UCitem{Tipo}{ Extiende de CU41 Iniciar sesión de personal escolar web}
         \UCitem{Observaciones}{Ninguna}
     
     \end{UseCase}
     %-------------------------------------- 
     
     \begin{UCtrayectoria}
     
         \UCpaso[\UCactor] El Personal de gestión escolar accede a la pantalla \IUref{IU27}{ Dar de alta ETS }\label{CU29.introduceDatos} apretando el botón \IUbutton{Dar de alta ETS} a través del menú lateral que tiene en todas las pantallas e introduce los datos del ETS: Periodo, Unidad-de-aprendizaje, Turno, Fecha de aplicación, Hora de aplicación, Cupo máximo, Duración del ETS, Docente a cargo y Salon de aplicación.
         \UCpaso[\UCactor] El Personal de gestión escolar oprime el botón \IUbutton{ Guardar }.
         \UCpaso El sistema revisa que los datos del ETS sean válidos.
         \UCpaso El sistema verifica que el ETS o el salón no hayan sido registrados con anterioridad.
         \UCpaso El ETS es dado de alta y guardado en la base de datos.
         \UCpaso El sistema muestra el mensaje  {\bf MSG-35} ``ETS  dado de alta con éxito''.
     
     \end{UCtrayectoria}
     
    
\newpage

% !TeX root = ../ejemplo.tex

%--------------------------------------
\begin{UseCase}{CU-32}{Consultar lista de personal de seguridad}{

    Permitir al Personal de gestión escolar consultar la lista con la información de las personas que esta registradas como personal de seguridad.
}
    \UCitem{Versión}{\color{Gray}1}
    \UCitem{Autor}{\color{Gray}Huertas Ramírez Daniel Martín}
    \UCitem{Supervisa}{\color{Gray}De la cruz De la cruz Alejandra.}
    \UCitem{Actor}{Personal de gestión escolar}
    \UCitem{Propósito}{Mostrar una lista con la información de las personas que esta registradas como personal de seguridad.}
    \UCitem{Entradas}{Ninguna}
    \UCitem{Origen}{Teclado}
    \UCitem{Salidas}{Ninguna}
    \UCitem{Destino}{Pantalla \IUref{IU28}{Consultar lista de personal de seguridad}}
    \UCitem{Precondiciones}{ El Personal de gestión escolar debe de haber iniciado sesión}
    \UCitem{Postcondiciones}{Ninguna.}
    \UCitem{Errores}{
        E1: Cuando se pierde la conexión durante el proceso, los procesos se cancelan y se muestra el mensaje {\bf MSG-28}  ``El proceso no se pudo realizar por un fallo de red.''

        E2: Cuando no hay ningún usuario personal de seguridad dado de alta se muestra el mensaje {\bf MSG-29}  ``No hay personal de seguridad dado de alta.''
    }
    \UCitem{Tipo}{ Extiende de CU42 Iniciar sesión de personal escolar web}
    \UCitem{Observaciones}{Ninguna}

\end{UseCase}
%-------------------------------------- 

\begin{UCtrayectoria}
\UCpaso[\UCactor] El Personal de gestión escolar accede a la pantalla \IUref{IU28}{ Consultar lista de personal de seguridad }\label{CU32.introduceDatos} desde la pantalla \IUref{IUE05}{de saludo de personal de gestión escolar } apretando el botón \IUbutton{Consultar lista de personal de seguridad}.
\UCpaso El sistema muestra la información de todo el personal de seguridad.
\UCpaso[\UCactor] El Personal de gestión escolar revisa la información del personal de seguridad dado de alta.
\UCpaso[\UCactor] El Personal de gestión escolar decide que quiere dar de alta a un usuario como personal de seguridad.
\UCpaso[\UCactor] El Personal de gestión escolar selecciona el botón \IUbutton{Dar de alta personal de seguridad }.
\UCpaso[\UCactor] El Personal de gestión escolar es redirigido a la pantalla \IUref{IU29}{Dar de alta personal de seguridad}.
\end{UCtrayectoria}



\newpage

% !TeX root = ../ejemplo.tex

%--------------------------------------
\begin{UseCase}{CU-33}{Dar de alta personal de seguridad}{

    Permitir al personal de gestión escolar dar de alta un personal de seguridad.
}
    \UCitem{Versión}{\color{Gray}1}
    \UCitem{Autor}{\color{Gray}Huertas Ramírez Daniel Martín}
    \UCitem{Supervisa}{\color{Gray}De la cruz De la cruz Alejandra.}
    \UCitem{Actor}{Personal de gestión escolar}
    \UCitem{Propósito}{ Permitir al personal de gestión escolar dar de alta a un personal de seguridad.}
    \UCitem{Entradas}{CURP, RFC, Nombre, Apellido paterno, Apellido materno, Sexo, Cargo y Turno}
    \UCitem{Origen}{Teclado}
    \UCitem{Salidas}{Muestra mensaje {\bf MSG-37} ``Personal de seguridad dado de alta con éxito''.}
    \UCitem{Destino}{-}
    \UCitem{Precondiciones}{El Personal de gestión escolar debe de haber iniciado sesión.}
    \UCitem{Postcondiciones}{El personal de seguridad es dado de alta y sus datos se guardan en la base de datos.}
    \UCitem{Errores}{

        E1: Cuando falta algún dato requerido entonces el sistema muestra el mensaje {\bf MSG-29}{``Los campos no están correctamente llenados.''}

        E2: Cuando el CURP del personal de seguridad ya está registrado en el sistema muestra el mensaje {\bf MSG-38}{``El CURP ya ha sido asociado a este personal de seguridad con anterioridad u otro personal de seguridad.''}

    }
    \UCitem{Tipo}{ Extiende de CU42 Iniciar sesión de personal escolar web}
    \UCitem{Observaciones}{Ninguna}

\end{UseCase}
%-------------------------------------- 

\begin{UCtrayectoria}

    \UCpaso[\UCactor] El Personal de gestión escolar accede a la pantalla \IUref{IU29}{ Dar de alta personal de seguridad}\label{CU33.introduceDatos} apretando el botón \IUbutton{Dar de alta p. seguridad} a través del menú lateral que tiene en todas las pantallas e introduce los datos del personal de seguridad CURP, RFC, Nombre, Apellido paterno, Apellido materno, Sexo, Cargo y Turno.
    \UCpaso[\UCactor] El Personal de gestión escolar oprime el botón \IUbutton{Dar de alta personal de seguridad}.
    \UCpaso El sistema revisa que los datos del personal de seguridad sean válidos.
    \UCpaso El sistema verifica que el CURP no haya sido registrado con anterioridad.
    \UCpaso El sistema muestra el mensaje {\bf MSG-37} ``Personal de seguridad dado de alta con éxito''.

\end{UCtrayectoria}

\newpage

% !TeX root = ../ejemplo.tex

%--------------------------------------
\begin{UseCase}{CU-36}{Consultar lista de docentes }{

    Permitir al Personal de gestión escolar consultar la lista con la información de docentes registrados.
}
    \UCitem{Versión}{\color{Gray}1}
    \UCitem{Autor}{\color{Gray}Huertas Ramírez Daniel Martín}
    \UCitem{Supervisa}{\color{Gray}De la cruz De la cruz Alejandra.}
    \UCitem{Actor}{Personal de gestión escolar}
    \UCitem{Propósito}{Mostrar una lista con todos los docentes registrados en el sistema.}
    \UCitem{Entradas}{Ninguna}
    \UCitem{Origen}{Teclado}
    \UCitem{Salidas}{Ninguna}
    \UCitem{Destino}{-}
    \UCitem{Precondiciones}{ El Personal de gestión escolar debe de haber iniciado sesión}
    \UCitem{Postcondiciones}{Ninguna.}
    \UCitem{Errores}{
        E1: Cuando no hay ningún docente dado de alta se muestra el mensaje {\bf MSG-39}  ``No hay docentes dados de alta.''
    }
    \UCitem{Tipo}{ Extiende de CU41 Iniciar sesión de personal escolar web}
    \UCitem{Observaciones}{Ninguna}

\end{UseCase}
%-------------------------------------- 

\begin{UCtrayectoria}
\UCpaso[\UCactor] El Personal de gestión escolar accede a la pantalla \IUref{IU30}{ Consultar lista de docentes}\label{CU36.introduceDatos} apretando el botón \IUbutton{Consultar docentes} a través del menú lateral que tiene en todas las pantallas.
\UCpaso El sistema muestra la información de todos los docentes.
\end{UCtrayectoria}



\newpage

% !TeX root = ../ejemplo.tex

%--------------------------------------
\begin{UseCase}{CU-37}{ Dar de alta docente}{

    Permitir al personal de gestión escolar dar de alta a un docente.
}
    \UCitem{Versión}{\color{Gray}1}
    \UCitem{Autor}{\color{Gray}Huertas Ramírez Daniel Martín}
    \UCitem{Supervisa}{\color{Gray}De la cruz De la cruz Alejandra.}
    \UCitem{Actor}{Personal de gestión escolar}
    \UCitem{Propósito}{ Permitir al personal de gestión escolar dar de alta a un docente.}
    \UCitem{Entradas}{CURP,RFC,Correo-institucional,Sexo,Nombre y Cargo}
    \UCitem{Origen}{Teclado}
    \UCitem{Salidas}{Muestra mensaje {\bf MSG-41} ``Docente dado de alta con éxito''.}
    \UCitem{Destino}{Pantalla \IUref{IU31}{Consultar lista de docentes}}
    \UCitem{Precondiciones}{El Personal de gestión escolar debe de haber iniciado sesión.}
    \UCitem{Postcondiciones}{El Docente es dado de alta y sus datos se guardan en la base de datos.}
    \UCitem{Errores}{

        E1: Cuando se pierde la conexión durante el proceso, los procesos se cancelan y se muestra el mensaje {\bf MSG-28}  ``El proceso no se pudo realizar por un fallo de red.''

        E2: Cuando falta algún dato requerido entonces el sistema muestra el mensaje {\bf MSG-29}{``Los campos no están correctamente llenados.''}

        E3: Cuando el CURP o el RFC del docente ya está registrado en el sistema muestra el mensaje {\bf MSG-40}{``El CURP o el RFC ya ha sido asociado a este docente con anterioridad u otro docente.''}

    }
    \UCitem{Tipo}{ Extiende de CU-36 Consultar lista de docentes}
    \UCitem{Observaciones}{Ninguna}

\end{UseCase}
%-------------------------------------- 

\begin{UCtrayectoria}

    \UCpaso[\UCactor] El Personal de gestión escolar accede a la pantalla \IUref{IU31}{Dar de alta docente}\label{CU37.introduceDatos} desde la pantalla \IUref{IU30}{Consultar lista de docentes} apretando el botón \IUbutton{Dar de alta docente} e introduce los datos del docente (CURP, Docente.RFC, Correo-institucional, Sexo, Nombre y Cargo.)
    \UCpaso[\UCactor] El Personal de gestión escolar oprime el botón \IUbutton{dar de alta docente}.
    \UCpaso El sistema revisa que los datos del personal de seguridad sean válidos.
    \UCpaso El sistema verifica que la CURP y el RFC no hayan sido registrados con anterioridad.
    \UCpaso[\UCactor] El Personal de gestión escolar es redirigido a la pantalla \IUref{IU30}{Consultar lista de docentes}.

\end{UCtrayectoria}

\newpage

% !TeX root = ../ejemplo.tex

%--------------------------------------
\begin{UseCase}{CU-40}{Solicitar desbloqueo de cuenta}{
    Permitir que el usuario haga una solicitud de reactivación de cuenta.
}
    \UCitem{Versión}{\color{Gray}1}
    \UCitem{Autor}{\color{Gray}Huertas Ramírez Daniel Martín}
    \UCitem{Supervisa}{\color{Gray}De la Cruz de la Cruz Alejandra.}
    \UCitem{Actor}{\hyperlink{Usuario}{Usuario}}
    \UCitem{Propósito}{Que el usuario haga una solicitud de reactivación de cuenta.}
    \UCitem{Entradas}{ Una justificación de la causa del bloqueo, para el alumno \hyperlink{Alumno.Boleta}{Boleta} y para el resto de usuarios \hyperlink{Empleado.RFC}{RFC}.}
    \UCitem{Origen}{Teclado}
    \UCitem{Salidas}{Manda un correo electrónico con una cuenta default con los datos requeridos para solicitar la reactivación de su cuenta.}
    \UCitem{Destino}{Ninguno}
    \UCitem{Precondiciones}{El usuario debe de haber iniciado sesión.}
    \UCitem{Postcondiciones}{Se envía el correo con la petición.}
    \UCitem{Errores}{
        E1: Cuando falta algún dato requerido entonces el sistema muestra el mensaje {\bf MSG-29}{``Los campos no están correctamente llenados.''}

        E2: Cuando se pierde la conexión durante el proceso, los procesos se cancelan y se muestra el mensaje {\bf MSG-28}``El proceso no se pudo realizar por un fallo de red.''

        E3: Cuando los datos no corresponden con algún registro en la base de datos se muestra el mensaje {\bf MSG-43}  ``Los datos no coinciden con ningún usuario''.
    }
    \UCitem{Tipo}{Caso de uso primario}
    \UCitem{Observaciones}{Ninguna}

\end{UseCase}
%-------------------------------------- 

\begin{UCtrayectoria}

    \UCpaso[\UCactor] El usuario introduce su Boleta y contraseña si es alumno y si es Empleado su RFC y contraseña en el sistema vía la \IUref{IU32}{Pantalla Solicitar desbloqueo de cuenta}\label{CU40.introduceDatos} para la app móvil y \IUref{IU32-2}{Pantalla Solicitar desbloqueo de cuenta} para el sistema web.
    \UCpaso[\UCactor] El usuario confirma la operación presionando el botón \IUbutton{Enviar}.
    \UCpaso El sistema verifica que todos los datos requeridos hayan sido capturados.
    \UCpaso El sistema verifica que el usuario este registrado en el sistema.
    \UCpaso El sistema manda correo con la solicitud de desbloqueo de cuenta.   

\end{UCtrayectoria}



\newpage

% !TeX root = ../ejemplo.tex

%--------------------------------------
\begin{UseCase}{CU-41}{Iniciar sesión de personal escolar web}{
		Permitir que solo el personal escolar pueda acceder al sistema, además de separar completamente las funciones del alumno y el personal escolar.
	}
	\UCitem{Versión}{\color{Gray}1}
	\UCitem{Autor}{\color{Gray}Huertas Ramírez Daniel Martín}
	\UCitem{Supervisa}{\color{Gray}De la Cruz de la Cruz Alejandra.}
	\UCitem{Actor}{\hyperlink{PersonalAcademico}{Empleado} (Personal de la DAE y Personal de gestión escolar)}
	\UCitem{Propósito}{Que el empelado pueda acceder al sistema web y sus funciones específicas. }
	\UCitem{Entradas}{RFC, Contraseña, Captcha}
	\UCitem{Origen}{Teclado}
	\UCitem{Salidas}{-}
	\UCitem{Destino}{Pantalla \IUref{IUE04}{Pantalla inicial de personal de la DAE} si es el usuario es un personal de la DAE, o, en su caso contrario a la pantalla \IUref{IUE05}{Pantalla de saludo de personal de gestión escolar} si es personal de gestión escolar.}
	\UCitem{Precondiciones}{El empleado debe estar registrado en el sistema.}
	\UCitem{Postcondiciones}{El empleado accede al sistema y podrá realizar las acciones pertinentes a su cargo.}
	\UCitem{Errores}{
		E1: Cuando falta algún dato requerido entonces el sistema muestra el mensaje {\bf``Los campos no están correctamente llenados.''}
		
		E3: Cuando la contraseña o el usuario ingresados sean incorrectos, el sistema no permite el acceso al empleado y se muestra el mensaje {\bf ``El usuario o la contraseña no corresponden con ningún empleado.''}
	}
	\UCitem{Tipo}{Caso de uso primario}
	\UCitem{Observaciones}{}
\end{UseCase}
%--------------------------------------

	\begin{UCtrayectoria}
	\UCpaso[\UCactor] El usuario introduce su usuario, contraseña y captcha requerido en el sistema por medio del teclado en la  \IUref{IU33}{Pantalla de Iniciar sesión de personal escolar web}\label{CU41.introduceDatos}.
	\UCpaso[\UCactor] El usuario confirma la operación presionando el botón \IUbutton{Iniciar Sesión}.
	\UCpaso El sistema verifica que todos los datos requeridos hayan sido capturados.
	\UCpaso El sistema verifica que el empleado este registrado en el sistema.
	\UCpaso El sistema verifica que la contraseña corresponda al usuario ingresado.
	\UCpaso El sistema verifica que tipo acceso tiene este empleado.
	\UCpaso La sesión es iniciada con éxito.
	\UCpaso El Empleado es redirigido a la pantalla \IUref{IUE04}{Pantalla Inicial de personal de la DAE} si es personal de la DAE o a la pantalla \IUref{IUE05}{Pantalla de saludo de personal de gestión escolar} si es personal de gestión escolar.
	
\end{UCtrayectoria}







\newpage

% !TeX root = ../ejemplo.tex

%--------------------------------------
\begin{UseCase}{CU-43}{Consultar lista de alumnos }{
		
		Permitir al Personal de la DAE consultar la lista de alumnos con su respectiva información.
	}
	\UCitem{Versión}{\color{Gray}1}
	\UCitem{Autor}{\color{Gray}Huertas Ramírez Daniel Martín}
	\UCitem{Supervisa}{\color{Gray}De la cruz De la cruz Alejandra.}
	\UCitem{Actor}{Personal de la DAE}
	\UCitem{Propósito}{Permitir al Personal de la DAE consultar la lista de alumnos con su respectiva información.}
	\UCitem{Entradas}{Ninguna}
	\UCitem{Origen}{Teclado}
	\UCitem{Salidas}{Ninguna}
	\UCitem{Destino}{-}
	\UCitem{Precondiciones}{ El Personal de la DAE debe de haber iniciado sesión}
	\UCitem{Postcondiciones}{Ninguna.}
	\UCitem{Errores}{
		E1: Cuando no hay ningún alumno dado de alta se muestra el mensaje {\bf MSG-39}  ``No hay alumnos dados de alta.''
	}
	\UCitem{Tipo}{ Extiende de CU41 Iniciar sesión de personal escolar web}
	\UCitem{Observaciones}{Ninguna}
	
\end{UseCase}
%-------------------------------------- 

\begin{UCtrayectoria}
	\UCpaso[\UCactor] El Personal de la DAE accede a la pantalla \IUref{IU46}{ Consultar lista de alumnos}\label{CU36.introduceDatos} apretando el botón \IUbutton{Consultar alumnos} a través del menú lateral que tiene en todas las pantallas.
	\UCpaso El sistema muestra la información de todos los docentes.
\end{UCtrayectoria}


