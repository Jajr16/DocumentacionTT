% !TeX root = ../ejemplo.tex

%--------------------------------------
\section{IU21} {Dar de alta a alumno}

\subsection{Objetivo}
	Permitir al personal de la DAE dar de alta a un alumno.
\subsection{Diseño}
    Esta pantalla \IUref{IU21}{ Dar de alta a alumno } (ver figura~\ref{IU21}) puede ser accedida desde la pantalla \IUref{IUE04}{de personal de la DAE}

\IUfig[1]{UI-CU21}{IU21}{ Dar de alta a alumno.}

\subsection{Salidas}
Muestra mensaje {\bf MSG-31} ``Alumno dado de alta con éxito''.
\subsection{Entradas}
    \hyperlink{Alumno.Boleta}{Boleta}, \hyperlink{Alumno.Nombre}{Nombre}, \hyperlink{Alumno.CURP}{CURP}, \hyperlink{Alumno.Sexo}{Sexo} y \hyperlink{Alumno.Correo institucional}{Correo institucional}
\subsection{Comandos}
\begin{itemize}
    \item \IUbutton{Dar de alta un alumno}  El sistema revisa que los datos del alumno sean válidos, verifica que el CURP o la boleta no hayan sido registrados con anterioridad, mantiene los datos para usarlos en el proceso de crear credencial. Y redirige a la pantalla \IUref{UI22}{Crear credencial}.
    \item \IUbutton{Calendario} Redirige a la pantalla \IUref{UI02}{Consultar calendario escolar}.
    \item \IUbutton{Campana} Redirige a la pantalla \IUref{UI03}{Consultar notificaciones }.
    \item \IUbutton{Home} Redirige a la pantalla de bienvenida correspondiente al tipo de usuario.
    
\end{itemize}

\subsection{Mensajes}

\begin{Citemize}
    \item {\bf MSG-31} ``Alumno dado de alta con éxito''.
    \item {\bf MSG-28}  ``El proceso no se pudo realizar por un falló de red.''
    \item {\bf MSG-29}{``Los campos no están correctamente llenados.''}
    \item {\bf MSG-30}{``La CURP o la boleta ya han sido asociadas a este alumno con anterioridad u otro alumno.''}
\end{Citemize}
