\section{Docentes}

En esta sección, se presenta el análisis exhaustivo de los resultados obtenidos de las 15 encuestas realizadas a docentes. Estas encuestas tuvieron como propósito principal evaluar tanto el funcionamiento como la experiencia de uso de la aplicación móvil en funcionalidades clave. La recolección de esta retroalimentación es crucial para comprender a fondo la percepción de los usuarios sobre aspectos críticos del sistema y su desempeño en un entorno real. 

Las preguntas de las encuestas se diseñaron para enfocarse en varios puntos específicos: la efectividad del reconocimiento facial, la facilidad y precisión para consultar el listado de ETS y la eficiencia en la comparación de credenciales mediante código QR. 

Los resultados preliminares de estas encuestas, en sintonía con la retroalimentación de otros grupos de usuarios, arrojan una mayoría de comentarios positivos respecto a la operatividad general de las funcionalidades. Sin embargo, el proceso de encuestado también permitió identificar puntos específicos de mejora y áreas de oportunidad significativas. Estas incluyen tanto aspectos relacionados con la usabilidad y la experiencia del usuario, como la detección de errores de funcionalidad específicos que se detallan en la sección de pruebas.

A continuación, se presenta el análisis completo de las encuestas aplicadas a los docentes, desglosando los hallazgos y las implicaciones para futuras mejoras del sistema. 
