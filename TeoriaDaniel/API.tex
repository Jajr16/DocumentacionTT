% !TeX root = ../ejemplo.tex

\section{API}

\begin{list}{}%
    {\setlength{\leftmargin}{1cm}\setlength{\rightmargin}{1cm}}
    \item\relax
    \small

Una API, o interfaz de programación de aplicaciones, es un conjunto de reglas o protocolos que permiten que las aplicaciones de software se comuniquen entre sí para intercambiar datos, características y funcionalidades \cite{CitaD15}.
Las API simplifican y aceleran el desarrollo de software y aplicaciones permitiendo a los desarrolladores integrar datos, servicios y capacidades de otras aplicaciones, en lugar de desarrollarlas desde cero. Las API también ofrecen a los propietarios de aplicaciones una forma sencilla y segura de poner los datos y las funciones de sus aplicaciones a disposición de los departamentos de su organización. Los propietarios de aplicaciones también pueden compartir o comercializar datos y funciones con asociados de negocios o terceros [24].

\end{list}

\subsection{OpenCV}

\begin{list}{}%
    {\setlength{\leftmargin}{1cm}\setlength{\rightmargin}{1cm}}
    \item\relax
    \small

OpenCV, o Open Source Computer Vision Library, es una colección de más de 2500 algoritmos optimizados que facilitan el procesamiento de imágenes y la realización de diversas operaciones de vision computacional. Fundada por Intel en 2000 y con colaboración de múltiples empresas y desarrolladores, esta biblioteca es de fácil acceso y cuenta con una licencia BSD [25].
Siendo compatible con varios lenguajes de programación, incluido Python, Java y C++, OpenCV permite trabajar en múltiples plataformas y sistemas operativos. Esta biblioteca también destaca por su papel en el impulso de tecnologías emergentes como el deep learning \cite{CitaD16}.

\end{list}

Por último esta cuenta con muchos procesos y funciones que apoyan al reconocimiento facial, además de ser compatible con Python, Kotlin y Java y poseer una alta cantidad de tutoriales que rondan desde las integraciones básicas hasta procesos avanzados.

\subsection{Licencia BSD}

\begin{list}{}%
    {\setlength{\leftmargin}{1cm}\setlength{\rightmargin}{1cm}}
    \item\relax
    \small

"La licencia BSD, también conocida como licencia de distribución de software de Berkeley, es una popular licencia de código abierto que permite el libre uso, modificación y distribución de software \cite{CitaD17}."

\end{list}

Esta licencia nos permite hacer uso de la biblioteca OpenCV de manera libre, lo cual implica que se puede implementar la biblioteca y darle uso al software de manera comercial, nos permite modificar y distribuir el software sin restricciones significativas y nos permite no publicar el código fuente si así lo deseamos.

Ya con el sistema operativo y los lenguajes de programación establecidos, tenemos que especificar las herramientas que nos permitirán escribir, depurar y gestionar el código de manera eficiente. Estas herramientas son los entornos de desarrollo Integrados (IDE, por sus siglas en inglés), que proporcionan un conjunto de funciones como editores de código, depuradores, y compiladores.

