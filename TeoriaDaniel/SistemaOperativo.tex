% !TeX root = ../ejemplo.tex

\section{Sistema operativo}

Ahora bien, para que una aplicación móvil funcione correctamente, es fundamental comprender el papel crucial que desempeña el sistema operativo en los dispositivos.
Los sistemas operativos son el software principal de cualquier dispositivo móvil o computadora, estos comienzan a funcionar desde el primer momento que el dispositivo es encendido y son los encargados de preparar e iniciar todos los procesos principales, estructuras importantes y todas las herramientas que hacen que el dispositivo pueda comunicarse e interactuar con quien lo utiliza, y viceversa \cite{CitaD1},\cite{CitaD2}.
En otras palabras, un sistema operativo es el software que se encarga de ofrecer una interfaz para interactuar al usuario, gestionar la seguridad del dispositivo, el acceso a los periféricos conectados y administrar la memoria asignada, los procesos, los recursos, las tareas y las actualizaciones y versiones de sí mismo o de otro software \cite{CitaD1},\cite{CitaD2}. 
Los sistemas operativos se dividen según su entorno (móviles o de computadora):

\subsection{Sistemas operativos móviles}

“Son los que se han creado y desarrollado para dispositivos móviles, fundamentalmente móviles y tablets, pero también relojes inteligentes. Los más conocidos son Android y iOS” \cite{CitaD2}.

\subsection{Sistema operativo android}

\begin{list}{}%
    {\setlength{\leftmargin}{1cm}\setlength{\rightmargin}{1cm}}
    \item\relax
    \small
    Android es reconocido como el sistema operativo más utilizado en el mundo debido a su versatilidad. Su diseño de código abierto permite a los desarrolladores crear aplicaciones personalizadas y a los fabricantes de hardware adaptarlo para sus dispositivos específicos. Además, su constante evolución garantiza compatibilidad con las últimas tecnologías y características \cite{CitaD3}.
    \end{list}

\subsection{Principales características Android}

    \begin{list}{}%
    {\setlength{\leftmargin}{1cm}%
     \setlength{\rightmargin}{1cm}%
     \setlength{\itemsep}{0.5\baselineskip}%
     \setlength{\parsep}{0pt}}
     
    \item\relax
    \small
    \textbf{Interfaz personalizable:} Android permite a los fabricantes y usuarios personalizar la apariencia y funcionalidad del sistema (pantalla de inicio, iconos y estructura gráfica) \cite{CitaD4}. 
    
    \item\relax
    \small
    \textbf{Google Play y su ecosistema:} Android ofrece acceso a la Google Play Store, la cual es una tienda virtual de Google que permite a los usuarios descargar y subir aplicaciones de todo tipo, siempre y cuando cumplan con las normas de seguridad y calidad. Además, esta tienda facilita la integración con los servicios de Google, los cuales son usados con extrema frecuencia \cite{CitaD4}.
    
    \item\relax
    \small
    \textbf{Seguridad:} Android incluye Google Play Protect, que es una aplicación encargada de analizar otras aplicaciones y proteger el dispositivo. Además, esta ofrece actualizaciones frecuentes y permite al usuario decidir qué datos compartir con el desarrollador mediante el uso de permisos \cite{CitaD4}.
    
    \end{list}

\subsection{Sistemas operativo iOS}

\begin{list}{}%
    {\setlength{\leftmargin}{1cm}\setlength{\rightmargin}{1cm}}
    \item\relax
    \small
    iOS es un sistema operativo lanzado y utilizado por Apple. Su nombre proviene de iPhone OS. Es decir, iPhone Operative System o Sistema Operativo de iPhone. Utilizando las siglas, iOS. Se lanzó originalmente para el teléfono de la marca, aunque también se ha utilizado durante años en otros dispositivos de la compañía como en algunos de los reproductores de música iPod o en las tabletas iPad (hasta la llegada de iPadOS) \cite{CitaD5}.

    Se trata de un sistema cerrado que no puedes utilizar salvo en dispositivos de marca Apple. La gran diferencia con Android es esta: el sistema operativo de Google puede instalarse en infinidad de teléfonos de todas las marcas, pero iOS es un sistema cerrado y exclusivo para los aparatos de la marca de Cupertino. No para los demás. Al igual que otros sistemas operativos móviles, iOS nos permite instalar aplicaciones para añadir funciones a las que vienen por defecto en el smartphone \cite{CitaD5}.

\end{list}
    
\subsection{Principales características iOS}


    \begin{list}{}%
        {\setlength{\leftmargin}{1cm}%
         \setlength{\rightmargin}{1cm}%
         \setlength{\itemsep}{0.5\baselineskip}%
         \setlength{\parsep}{0pt}}
         
        \item\relax
        \small
        \textbf{Interfaz Intuitiva y Optimización:} iOS ofrece una interfaz amigable y visualmente atractiva, con navegación mediante gestos simples y optimiza el rendimiento para el hardware específico de Apple \cite{CitaD6}.
 
        
        \item\relax
        \small
        \textbf{Seguridad y Privacidad de Alto Nivel:} Apple prioriza la seguridad a través de características como encriptación de extremo a extremo, autenticación biométrica y su sistema de App Sandbox, que aísla cada aplicación para evitar el acceso no autorizado a información sensible \cite{CitaD6}.

        
        \item\relax
        \small
        \textbf{App Store Segura:} Las aplicaciones en iOS deben cumplir con estrictos estándares de seguridad para ser aprobadas en la App Store. Apple realiza auditorías regulares para asegurar que las aplicaciones estén libres de malware y respeten la privacidad de los usuarios \cite{CitaD6}.
        
        \end{list}

\subsection{Comparando Android e iOS}

\begin{longtable}{|p{5cm}|p{5cm}|p{5cm}|}
    \hline
    \textbf{Criterio} & \textbf{Android} & \textbf{iOS} \\ 
    \endfirsthead
    \hline
    \textbf{Criterio} & \textbf{Android} & \textbf{iOS} \\  
    \hline
    \endhead
    \hline
    \endfoot
    \endlastfoot
    \hline
    Lenguajes de programación & Kotlin & Swift \\ \hline
    IDE (Entorno de desarrollo) & Android Studio & Xcode \\ \hline
    Plataforma de distribución & Google Play Store & App Store \\ \hline
    Documentación y comunidad & Acceso más flexible, pero con más variabilidad entre dispositivos. & Mejor integración con el hardware de Apple (optimización). \\ \hline
    Distribución de aplicaciones & Publicación rápida, sin aprobación previa. & Necesita aprobación antes de publicación (más controlado). \\ \hline
    Costos de desarrollo & Sin costos de registro anual. & Licencia de desarrollador de 99 dólares al año. \\ \hline
    Mercado objetivo & Mercado más amplio, con una variedad de dispositivos (bajos, medios y altos). & Usuarios de Apple, con dispositivos limitados. \\ \hline
    Desarrollo multiplataforma & Ecosistema más diverso con diferentes marcas y modelos de dispositivos. & Integración total entre dispositivos Apple (iPhone, iPad, Mac, etc.). \\ \hline
    Actualizaciones & Las actualizaciones son más lentas y dependen del fabricante. & Actualizaciones frecuentes y rápidas para todos los usuarios. \\ \hline
    Comunidad y recursos & Comunidad más grande con más recursos, tutoriales y soporte. & Comunidad muy activa pero más pequeña comparada con Android, además de que los tutoriales son más avanzados y poco amigables a los principiantes. \\
    \hline
    
    \caption{Comparación entre Android y iOS. Elaboración propia}
    \label{tab:Tabla1}
\end{longtable}


    Analizando las características de estos 2 sistemas operativos, decidimos enfocar nuestros esfuerzos en el desarrollo de una aplicación para Android, ya que, nuestro presupuesto es limitado y el tiempo de desarrollo es ajustado, esta plataforma nos ofrece varias ventajas. En primer lugar, Android tiene una cuota de mercado mucho más amplia, lo que nos permitirá llegar a un público más amplio sin tener que depender de un grupo cerrado como es el de iOS, esto es de suma importancia porque necesitamos maximizar el impacto de nuestra aplicación alcanzando la mayor cantidad de usuarios posible para que la mayor parte de los alumnos y profesores puedan usar la aplicación.
    Otro factor importante es el costo. Publicar una aplicación en la App Store de iOS requiere una tarifa anual de 99 dólares, algo que no podemos permitirnos con el presupuesto actual, por otro lado, Android solo requiere un pago único de 25 dolares para registrar nuestra cuenta de desarrollador, lo cual nos deja más recursos para otras áreas del proyecto, además, la distribución de la aplicación en Google Play es mucho más rápida y sencilla, ya que no tenemos que pasar por un proceso de revisión tan estricto como el de iOS, esto nos permitirá lanzar nuestra aplicación en menos tiempo, algo que es crucial dado el tiempo limitado con el que contamos.
    Por ultimo Android es compatible con Python esto es un punto clave en nuestra decisión porque ya que planeamos usar Python para implementar el servidor de la aplicación y este nos facilita la implementación de funciones como; procesamiento de datos e imágenes, algoritmos de aprendizaje automático y técnicas de reconocimiento facial y redes neuronales (cabe recalcar que Android es compatible con OpenCV).
    

\subsection{Sistemas operativos para computadoras}

Son los que se han creado y desarrollado para que funcionen en computadoras de escritorio y portátiles. Los más conocidos son Windows, macOS y Unix.

\subsection{Sistema operativo Windows}

\begin{list}{}%
    {\setlength{\leftmargin}{1cm}\setlength{\rightmargin}{1cm}}
    \item\relax
    \small

El sistema operativo Windows, desarrollado por Microsoft, es uno de los sistemas operativos más populares y ampliamente utilizados en el mundo. Ofrece una interfaz gráfica intuitiva que permite a los usuarios interactuar con el hardware y software de sus computadoras de manera eficiente \cite{CitaD7}.
La estructura de Windows se basa en capas, donde cada capa cumple una función específica. La capa más baja es el kernel, que se encarga de gestionar los recursos de hardware y proporcionar servicios básicos al sistema operativo. Por encima del kernel se encuentran los controladores de dispositivo, que permiten la comunicación entre el hardware y el sistema operativo \cite{CitaD7}.
Además de estas capas fundamentales, Windows cuenta con una amplia gama de servicios y aplicaciones que ofrecen funcionalidades adicionales a los usuarios. Estas aplicaciones incluyen el Explorador de Windows, que permite a los usuarios navegar por los archivos y carpetas de su computadora, y el Panel de Control, que proporciona acceso a la configuración del sistema \cite{CitaD7}. 

\end{list}


\subsection{CSistemas operativo macOS}

\begin{list}{}%
    {\setlength{\leftmargin}{1cm}\setlength{\rightmargin}{1cm}}
    \item\relax
    \small

Mac o MacOS, es el nombre del sistema operativo desarrollado por Apple para la línea de computadoras creados por la misma empresa. MacOS en sí mismo no es un sistema operativo, sino una familia de sistemas que han ido evolucionando a través de diferentes versiones con el paso del tiempo. La principal característica del sistema operativo Mac, es que éste está diseñado para funcionar de manera optimizada en equipos de Apple \cite{CitaD8}.
El sistema operativo Mac sigue la línea de compatibilidad de Apple, por lo que permite la comunicación entre MacOS y otros dispositivos creados por Apple como el iPad, appleTV, iPhone, etc \cite{CitaD8}.
Debido a su arquitectura, MacOS cuenta con un sistema de archivos propio, por lo que no puede procesar de forma nativa los archivos o programas con formatos usualmente empleados para Windows o Linux \cite{CitaD8}.

\end{list}


\subsection{Sistemas operativo Unix}

\begin{list}{}%
    {\setlength{\leftmargin}{1cm}\setlength{\rightmargin}{1cm}}
    \item\relax
    \small

Unix es un sistema operativo que nace a principios de los años 70, creado principalmente por Dennis Ritchie y Ken Thompson. Sus características técnicas principales son: su portabilidad, su capacidad multiusuario y multitarea, su eficiencia; su alta seguridad y su buen desempeño en tareas de red. Pero Unix, más que una marca, también es una filosofía que tiene por principios el minimalismo y el modularidad: hacer programas que hagan una sola cosa bien hecha y que, al comunicarse entre sí, ejecuten tareas más complejas.
Un sistema Unix puede dividirse en tres áreas básicas: 

        El núcleo del sistema operativo.

        El intérprete de comandos y algunos programas utilitarios.

        Lo demás que necesitas, como las aplicaciones de usuario o la interfaz gráfica, son paquetes adicionales.

Por sus características técnicas y su filosofía abierta, existen diversos sistemas operativos que se conocen como derivados de Unix o sistemas de la familia Unix \cite{CitaD9}.

\end{list}



\subsection{Comparando Windows, Linux y macOS}


    
    \begin{longtable}{|p{3cm}|p{4.5cm}|p{4.5cm}|p{5cm}|}
        \hline
		\textbf{Característica} & \textbf{Windows} & \textbf{Linux} & \textbf{macOS}\\ 
		\endfirsthead
		\hline
        \textbf{Característica} & \textbf{Windows} & \textbf{Linux} & \textbf{macOS}\\ 
		\hline
		\endhead
		\hline
		\endfoot
		\endlastfoot
        \hline

    Accesibilidad & Muy accesible, ampliamente disponible en PCs de todos los rangos de precios. & Requiere conocimientos técnicos, pero es gratis y disponible en una amplia gama de hardware. & Accesible solo en hardware Apple (más costoso). \\ \hline
    Herramientas de Desarrollo & Soporta una amplia gama de IDEs y plataformas de desarrollo. & Soporta herramientas como Visual Studio Code, PyCharm, Eclipse; excelente para desarrollo web, sistemas y servidores. & Excelente soporte para Xcode y terminal poderoso. \\ \hline
    Compatibilidad con Lenguajes & Amplia compatibilidad con lenguajes como Java, Python, JavaScript, C++, entre otros. & Compatible con casi todos los lenguajes, ideal para Python, C, C++, JavaScript, etc. & Compatible con todos los lenguajes principales, especialmente para desarrollo iOS (Swift, Objective-C). \\ \hline
    Soporte de Software & Compatible con la mayoría de las aplicaciones comerciales. & Menos compatibilidad con software comercial, pero excelente soporte para software libre y de código abierto. & Mejor compatibilidad con software de diseño y desarrollo exclusivo de Apple. \\ \hline
    Rendimiento & Rinde bien en la mayoría de las configuraciones, pero puede ser más pesado que Linux o macOS. & Suele ser más ligero y rápido en hardware modesto. & Gran rendimiento en hardware Apple, optimizado para su arquitectura. \\ \hline
    Costo & Varía dependiendo de la edición (Windows Home o Pro). & Totalmente gratuito. & Alto costo de hardware y software. \\ \hline
    Soporte de Python & Soporta bien Python, con muchas herramientas. & Ideal para Python, con herramientas avanzadas y administración de entornos virtuales. & Buen soporte para Python, con herramientas como Anaconda y entornos virtuales. \\ \hline
    Facilidad de Uso & Interfaz fácil de usar para la mayoría de usuarios, especialmente principiantes. & Requiere algo de conocimiento técnico para instalar y configurar, pero es muy flexible. &Intuitivo y fácil de usar, especialmente para usuarios de Apple. \\ \hline
    Desarrollo Móvil & Soporta desarrollo Android mediante Android Studio y otras herramientas. & Soporta Android Studio y desarrollo móvil para otras plataformas, pero no es ideal para iOS. & Ideal para desarrollo iOS (Xcode) y aplicaciones macOS. \\



    \hline
    
    \caption{Comparación entre Android y iOS. Elaboración propia}
    \label{tab:Tabla2}
    \end{longtable}

Conociendo las características de estos 3 sistemas operativos, decidimos optar por Windows como el sistema operativo principal para el desarrollo de nuestro proyecto, ya que, teniendo en cuenta nuestras limitaciones de presupuesto y tiempo, este sistema operativo nos ofrece ventajas importantes. En primer lugar, Windows es el sistema operativo más accesible para nosotros porque todo el equipo posee una computadora con este sistema operativo, por lo cual se nos facilita que nuestro equipo trabaje con estos equipos o si se requiere, se nos facilitaría adquirir nuevas computadoras sin que el costo sea una preocupación mayor.
Otro punto a favor que consideramos de Windows es la compatibilidad con una gran cantidad de herramientas de desarrollo ya que este sistema operativo ofrece soporte para una variedad de lenguajes de programación y es compatible con muchos de los entornos de desarrollo integrados (IDE). Esto nos permite utilizar las herramientas que ya conocemos o las que se ajustan mejor a nuestras necesidades sin tener que invertir tiempo en aprender nuevas plataformas, además, Windows es compatible con una gran variedad de software comercial y aplicaciones específicas, como las herramientas de diseño gráfico y edición, que son necesarias para el diseño de la aplicación y sus componentes gráficos.
Cabe mencionar que Linux es una excelente opción por su flexibilidad y la capacidad extra de los servidores, sin embargo, la escala del proyecto no es tan masiva, por lo que, decidimos que no era la mejor opción para nuestro proyecto debido a que requiere un conocimiento más avanzado y específico para configurar y administrar, también la curva de aprendizaje retrasaría el progreso porque poseemos una nula experiencia trabajando con este sistema operativo. Además, aunque Linux es gratuito y ligero, su compatibilidad con software comercial y aplicaciones de terceros no es tan robusta como la de Windows, lo que nos presenta problemas para aspectos del desarrollo.
Por otro lado, macOS tiene ventajas notables, como su optimización y rendimiento en hardware Apple, pero el costo del hardware y la limitación de solo estar disponible para dispositivos Apple lo hacen menos accesible para nuestro presupuesto. Además, aunque macOS es excelente para el desarrollo de aplicaciones iOS/macOS, no es tan versátil como Windows en cuanto a compatibilidad con diversas herramientas de desarrollo y software de terceros.

