% !TeX root = ../ejemplo.tex

\section{Elección de tecnología}
La elección de Kotlin como lenguaje principal y Android como plataforma se basa en criterios técnicos, prácticos y económicos que se alinean directamente con los requisitos del proyecto.

\subsection{Kotlin}
Kotlin es un lenguaje de alto nivel debido a su abstracción respecto al hardware y la plataforma subyacente. Esto significa que permite a los desarrolladores escribir código sin preocuparse por la gestión de memoria o las interacciones directas con el sistema operativo o el hardware, lo que mejora la productividad y la legibilidad del código, además, este es estáticamente tipado, lo que implica que los tipos de las variables se definen en tiempo de compilación, lo que permite detectar errores antes de ejecutar el programa. Este se usa principalmente en el desarrollo de aplicaciones Android, donde es el lenguaje recomendado por Google \cite{CitaD14}.

Además, Kotlin permite el uso de Jetpack Compose, un framework declarativo que moderniza la creación de interfaces de usuario en Android. También es compatible con bibliotecas clave para este proyecto como OpenCV, empleada en el módulo de reconocimiento facial.

\subsection{Android}
La decisión de desarrollar exclusivamente para Android se basa en:

\begin{itemize}
	\item La mayoría de los usuarios objetivo (alumnos, docentes y personal de seguridad) utilizan dispositivos Android.
	
	\item Android permite mayor flexibilidad y personalización, crucial para funciones como escaneo de códigos QR, acceso a la cámara, integración con Python y OpenCV.
	
	\item El entorno Android es ideal para pruebas, ya que se puede emular fácilmente en computadoras con Windows.
\end{itemize}

La compatibilidad de Android con Python facilita la integración del frontend (Android) con el backend, donde Python se encarga del procesamiento de imágenes y entrenamiento de modelos de reconocimiento facial, mientras que Spring Boot gestiona la lógica del sistema. Esta arquitectura garantiza un funcionamiento sincronizado, eficiente y seguro.


\section{Herramientas de Desarrollo}
El desarrollo de la aplicación se realizó con herramientas modernas que forman parte del ecosistema oficial de Android, permitiendo una implementación robusta y escalable.

\subsection{Android Studio}
Android Studio es el entorno de desarrollo integrado (IDE) oficial para el desarrollo de aplicaciones del mismo OS. Se basa en IntelliJ IDEA, un entorno de desarrollo integrado de Java para software, e incorpora sus herramientas de desarrollo y edición de código \cite{CitaD19}.

Para respaldar el desarrollo de aplicaciones dentro del sistema operativo, Android Studio utiliza un sistema de compilación basado en Gradle, un emulador de Android, plantillas de código e integración de GitHub. Cada proyecto en Android Studio tiene una o más modalidades con código fuente y archivos de recursos \cite{CitaD19}.

\subsection{Jetpack Compose}
Para la implementación de la aplicación móvil que forma parte de nuestro sistema de identificación y control de acceso, hemos decidido utilizar Jetpack Compose. La elección de la tecnología adecuada es importante para ofrecer una mejor experiencia de usuario y cumplir nuestros objetivos de diseño y funcional. 

\subsection*{¿Por que Jetpack compose?}
Jetpack compose es un framework (estructura o marco de trabajo que, bajo parámetros estandarizados, ejecutan tareas específicas en el desarrollo de un software) con la particularidad de ejecutar prácticas modernas en los desarrolladores de software a partir de la reutilización de componentes, así como también contando con la oportunidad de crear animaciones y temas oscuros. En este sentido, Jetpack Compose es el conjunto de herramientas ofrecidas por Android para el desarrollo de aplicaciones con un objetivo específico: simplificar y optimizar los códigos en la IU nativas \cite{CitaA01}. 


\subsection*{Ventajas}

\begin{itemize}
	\item \textbf{Menos código:} Simplifica el proceso de desarrollo haciendo menos código, todo se basa en funciones de modo que el código será simple y fácil de mantener.
	\item \textbf{Intuitiva:} Tan solo describe tu IU con un enfoque declarativo haciendo “qué hay que hacer” en vez de “cómo se debe hacer”.
	\item \textbf{Potente:} Tiene integrado Material Design con el cual puede crear apps atractivas al usuario con animaciones y mucho más.
	\item \textbf{Acelera el desarrollo:} Es compatible con proyectos existentes, puedes empezar a integrarlo por partes cuando quieras y donde quieras.
	\item \textbf{Kotlin:} Está escrito 100\% en Kotlin, lo cual nos permitirá usar sus herramientas potentes y API’s intuitivas.
\end{itemize}

\subsection*{Arquitectura Jetpack}
La arquitectura o estructura sobre la que se basa Jetpack Compose es una estructura Jetpack que se encarga principalmente de seguir ejecutando y beneficiándose de aquellos componentes de Android según la funcionalidad disponible. Por lo tanto, Jetpack Compose desarrolla herramientas denominadas «composables» a partir de elementos como botones o listados de objetos. A partir de fuentes de datos, pueden ser reutilizadas y ejecutadas en distintas fuentes sin la necesidad de programar varios códigos repetitivos.

Desde un aspecto comparativo, la creación de UI con Compose se realiza de forma similar a los de React Native. Es decir, mediante componentes reutilizables evitan maximizar la cantidad de códigos necesarios y repetitivos, tal como ocurre con HTML en la forma en la que se conjugan uno con otro. Sin embargo, es importante destacar que dicha compilación de componentes que logran la disminución en cantidad de códigos se realiza gracias a los plugins de compiladores de Kotlin, ejecutando así los componentes mediante la estructura de archivos de Kotlin \cite{CitaA19}. 

\subsection*{Estructura de Jetpack Compose}
Para entender de forma más sencilla la estructura de esta herramienta, conviene tener en cuenta componentes como:
\begin{itemize}
	\item \textbf{Compiler:} Se encarga mediante una estructura de plugins en Gradle, donde logra la interpretación y simplificación de códigos.
	\item \textbf{El entorno de ejecución:} Es el ámbito “depurador”, si se puede establecer de este modo, en el que se establecen y diferencian los componentes que necesitan actualización y cuáles no, así como también se crean listados de mantenimiento y composición.
	\item \textbf{UI:} Es el componente que interpreta el lenguaje establecido en el entorno de ejecución y lo refleja en pantalla.
\end{itemize}

\subsection{Gradle}

\begin{list}{}%
	{\setlength{\leftmargin}{1cm}\setlength{\rightmargin}{1cm}}
	\item\relax
	\small
	
	Gradle es el sistema de automatización de compilación utilizado por Android Studio, y fue esencial para el desarrollo del presente proyecto, permitiendo una integración fluida entre los módulos móviles y el backend. Esta herramienta de código abierto destaca por su rendimiento y flexibilidad, utilizando Kotlin DSL (Domain Specific Language) para definir scripts de compilación \cite{CitaD20}. 
	
	Gradle facilita la gestión eficiente de dependencias externas como Jetpack Compose, Retrofit y OpenCV, utilizadas en el desarrollo de la aplicación. Además, su capacidad para realizar compilaciones optimizadas en paralelo y reutilizar salidas anteriores permite reducir considerablemente los tiempos de construcción. También permite configurar distintos entornos de ejecución (debug y release), lo que fue útil para probar características específicas antes de su despliegue oficial. 
	
	En este proyecto, Gradle también permitió la integración con servicios backend mediante llamadas a APIs REST, conectando la aplicación con el servidor desarrollado en Spring Boot, fortaleciendo así la arquitectura cliente-servidor de la solución.
	
\end{list}

\subsection{Git y GitHub}

\begin{list}{}%
	{\setlength{\leftmargin}{1cm}\setlength{\rightmargin}{1cm}}
	\item\relax
	\small
	
	Durante el desarrollo del proyecto, se utilizó Git como sistema de control de versiones y GitHub como plataforma de colaboración y almacenamiento en la nube. Git, diseñado por Linus Torvalds, permite llevar un control detallado del historial de cambios en el código, lo cual es esencial en proyectos donde trabajan varios desarrolladores de forma simultánea \cite{CitaD20}.
	
	Cada miembro del equipo pudo trabajar en ramas independientes, realizar modificaciones, proponer mejoras y fusionar los cambios mediante revisiones de código, garantizando la integridad del repositorio principal. Esta metodología permitió detectar errores, revertir versiones cuando fue necesario, y mantener una trazabilidad clara del desarrollo de cada módulo.
	
	GitHub, además de servir como repositorio central, funcionó como herramienta de comunicación y documentación del avance del proyecto. Las funcionalidades de issues, pull requests y proyectos ayudaron a distribuir tareas, dar seguimiento a problemas y mantener una organización clara durante todo el ciclo de desarrollo \cite{CitaD21}.
	
\end{list}


