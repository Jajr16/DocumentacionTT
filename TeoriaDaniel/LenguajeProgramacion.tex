% !TeX root = ../ejemplo.tex

\section{Lenguaje de programación}

Una vez establecido el sistema operativo que vamos a usar, nos interesa establecer que es un lenguaje de programación.
\begin{list}{}%
    {\setlength{\leftmargin}{1cm}\setlength{\rightmargin}{1cm}}
    \item\relax
    \small
En términos generales, un lenguaje de programación es una herramienta que permite desarrollar software o programas para computadora. Los lenguajes de programación son empleados para diseñar e implementar programas encargados de definir y administrar el comportamiento de los dispositivos físicos y lógicos de una computadora. Lo anterior se logra mediante la creación e implementación de algoritmos de precisión que se utilizan como una forma de comunicación humana con la computadora \cite{CitaD10}.

A grandes rasgos, un lenguaje de programación se conforma de una serie de símbolos y reglas de sintaxis y semántica que definen la estructura principal del lenguaje y le dan un significado a sus elementos y expresiones \cite{CitaD10}.
\end{list}
Estos se dividen en 3 tipos: lenguajes máquina, lenguajes de bajo nivel y lenguajes de alto nivel.


\subsection{Lenguajes máquina}

\begin{list}{}%
    {\setlength{\leftmargin}{1cm}\setlength{\rightmargin}{1cm}}
    \item\relax
    \small

Es el sistema de códigos interpretable directamente por un circuito microprogramable, como el microprocesador de una computadora. Este lenguaje se compone de un conjunto de instrucciones que determinan acciones que serán realizadas por la máquina. Y un programa de computadora consiste en una cadena de estas instrucciones de lenguaje de máquina (más los datos). Normalmente estas instrucciones son ejecutadas en secuencia, con eventuales cambios de flujo causados por el propio programa o eventos externos. El lenguaje máquina es específico de cada máquina o arquitectura de la máquina, aunque el conjunto de instrucciones disponibles pueda ser similar entre ellas \cite{CitaD10}.

\end{list}

\subsection{Lenguajes de alto nivel}

\begin{list}{}%
    {\setlength{\leftmargin}{1cm}\setlength{\rightmargin}{1cm}}
    \item\relax
    \small

Un lenguaje de programación de bajo nivel es el que proporciona poca o ninguna abstracción del microprocesador de una computadora. Consecuentemente, su trasladado al lenguaje máquina es fácil. El término ensamblador (del inglés assembler) se refiere a un tipo de programa informático encargado de traducir un archivo fuente, escrito en un lenguaje ensamblador, a un archivo objeto que contiene código máquina ejecutable directamente por la máquina para la que se ha generado \cite{CitaD10}. 

\end{list}

\subsection{Lenguajes de bajo nivel}

\begin{list}{}%
    {\setlength{\leftmargin}{1cm}\setlength{\rightmargin}{1cm}}
    \item\relax
    \small

Los lenguajes de programación de alto nivel se caracterizan porque su estructura semántica es muy similar a la forma como escriben los humanos, lo que permite codificar los algoritmos de manera más natural, en lugar de codificarlos en el lenguaje binario de las máquinas, o a nivel de lenguaje ensamblador \cite{CitaD10}. 

\end{list}

Entre estos tipos de lenguajes nos interesa los lenguajes de alto nivel ya que para el proyecto estamos considerando el uso de Python, Java y Kotlin, que son considerados lenguajes de alto nivel. Concretamente Python es un lenguaje multiparadigma que puede ser estructurado como si se usara la programación orientada a objetos, Java usa la programación orientada a objetos y Kotlin puede trabajar con esta orientación por lo que antes de definir qué son estos dos lenguajes es de vital importancia definir qué es la “programación orientada a objetos”.


\subsection{Programación orientada a objetos}

\begin{list}{}%
    {\setlength{\leftmargin}{1cm}\setlength{\rightmargin}{1cm}}
    \item\relax
    \small

La programación orientada a objetos (POO) es un modelo de programación informática que organiza el diseño del software en torno a datos u objetos, en lugar de funciones y lógica. Un objeto puede definirse como un campo de datos que tiene unos atributos y comportamientos únicos\cite{CitaD11}.
Los ejemplos de un objeto pueden ir desde entidades físicas, como un ser humano que se describe por propiedades como el nombre y la dirección, hasta pequeños programas informáticos, como los widgets\cite{CitaD11}.
La POO se centra en los objetos que los desarrolladores quieren manejar, en lugar de la lógica necesaria para hacerlo. Este enfoque de la programación es muy adecuado para programas grandes, complejos y que se actualizan o mantienen activamente\cite{CitaD11}.
Esto incluye programas para la producción y el diseño web, así como aplicaciones móviles\cite{CitaD11}.

\end{list}

\subsection{Python}

\begin{list}{}%
    {\setlength{\leftmargin}{1cm}\setlength{\rightmargin}{1cm}}
    \item\relax
    \small

En términos técnicos, Python es un lenguaje de programación de alto nivel, orientado a objetos, con una semántica dinámica integrada, principalmente para el desarrollo web y de aplicaciones informáticas, o a nivel de lenguaje ensamblador\cite{CitaD12}.
Python es relativamente simple, por lo que es fácil de aprender, ya que requiere una sintaxis única que se centra en la legibilidad. Los desarrolladores pueden leer y traducir el código Python mucho más fácilmente que otros lenguajes\cite{CitaD12}.
Por tanto, esto reduce el costo de mantenimiento y de desarrollo del programa porque permite que los equipos trabajen en colaboración sin barreras significativas de lenguaje y experimentación\cite{CitaD12}.
Además, soporta el uso de módulos y paquetes, lo que significa que los programas pueden ser diseñados en un estilo modular y el código puede ser reutilizado en varios proyectos. Una vez se ha desarrollado un módulo o paquete, se puede escalar para su uso en otros proyectos, y es fácil de importar o exportar\cite{CitaD12}.
Por otro lado, uno de los beneficios más importantes de Python es que tanto la librería estándar como el intérprete están disponibles gratuitamente, tanto en forma binaria como en forma de fuente\cite{CitaD12}.
Tampoco hay exclusividad, ya que Python y todas las herramientas necesarias están disponibles en todas las plataformas principales \cite{CitaD12}.

\end{list}


\subsection{Java}

\begin{list}{}%
    {\setlength{\leftmargin}{1cm}\setlength{\rightmargin}{1cm}}
    \item\relax
    \small

Java es un lenguaje orientado a objetos; es decir que organiza el trabajo en torno a objetos en lugar de funciones y lógica. Las aplicaciones móviles y herramientas de software, incluidos los mundialmente famosos Amazon, Spotify y Minecraft, se basan en Java.
Java tiene dos propiedades que determinan qué tareas se pueden resolver con él.
Java es un lenguaje de programación orientado a objetos (POO). Toda interacción en él se produce a través de objetos. Estas entidades se describen en código y se les enseña a interactuar entre sí. Como resultado, un programa de estilo orientado a objetos consta de bloques separados que son fácilmente extensibles y escalables. \cite{CitaD13}

\end{list}

Y java tiene la capacidad de interpretar y compilar, es decir tiene un intérprete que se dedica a leer cada línea de código una por una y ejecutarla sin traducirla a lenguaje máquina y además puede transformar su código  a lenguaje máquina y hacer que el procesador lo compile traducido \cite{CitaD13}. 


\subsection{Kotlin}

Kotlin es un lenguaje de alto nivel debido a su abstracción respecto al hardware y la plataforma subyacente. Esto significa que permite a los desarrolladores escribir código sin preocuparse por la gestión de memoria o las interacciones directas con el sistema operativo o el hardware, lo que mejora la productividad y la legibilidad del código, además, este es estáticamente tipado, lo que implica que los tipos de las variables se definen en tiempo de compilación, lo que permite detectar errores antes de ejecutar el programa. Este se usa principalmente en el desarrollo de aplicaciones Android, donde es el lenguaje recomendado por Google \cite{CitaD14}.

Con la información anterior se puede explicar porque consideramos estos tres lenguajes de programación; Primero estos tres lenguajes poseen una misma orientación lo cual nos ayuda a que el desarrollo del proyecto sea más conciso y fácil, después estos son lenguajes de alto nivel que tienen una buena legibilidad lo que le permite al equipo entender de manera más sencilla el trabajo de los demás integrantes, también pueden ser conectados entre ellos de manera relativamente fácil y sobretodo su alta cantidad de bibliotecas de apoyo y su características de ser  estándares de desarrollo; en android (Kotlin), de desarrollo para servidores (Python) y Java para desarrollo web.
Otra razón por la que decidimos usar estos tres lenguajes de programación es por su compatibilidad con la API OpenCV la cual será una herramienta clave para el desarrollo de nuestro proyecto, por lo que es necesario explicar que es una API y qué es OpenCV.
