%--------------------------------------
\subsection{IU10: Pantalla Código QR}

\IUfig[.40]{cu12}{IU10}{Pantalla Código QR.}
\newpage

\subsubsection{Objetivo}
Permitir al docente escanear el código QR de la credencial del alumno para verificar su identidad.

\subsubsection{Diseño}
Esta pantalla \IUref{IU10}{Pantalla Código QR} (ver figura~\ref{IU10}) se muestra al presionar el botón "Verificar con QR"  desde la \IUref{IUE07}{Pantalla Crear Reporte}. Presenta una interfaz para que el docente pueda escanear el código QR de la credencial del alumno utilizando la cámara del dispositivo.



La pantalla contiene los siguientes elementos:
\begin{itemize}
	\item \textbf{Barra de navegación superior:}
	\begin{itemize}
		\item \textbf{Icono de flecha hacia la izquierda:} Para regresar a la pantalla anterior (\IUref{IUE07}{Pantalla Crear Reporte}).
	\end{itemize}
	\item \textbf{Vista de la cámara:} Un recuadro o área donde se muestra la imagen capturada por la cámara del dispositivo, permitiendo al docente enfocar el código QR.
	\item \textbf{Barra de navegación inferior:} Contiene iconos para:
	\begin{itemize}
		\item \textbf{Icono de casa:} Redirección a la pantalla de saludo correspondiente al tipo de usuario.
		\item \textbf{Icono de campana:} Redirección a la pantalla de notificaciones.
		\item \textbf{Icono de calendario con marcas:} Redirección a la pantalla \IUref{IU02}{Consultar calendario escolar}.
		\item \textbf{Icono de flecha apuntando hacia la derecha saliendo de un recuadro:} Cierra la sesión del usuario y lo regresa a la pantalla de inicio de sesión (\IUref{IU01}{Iniciar sesión del sistema móvil}).
	\end{itemize}
\end{itemize}

\subsubsection{Salidas}
Al escanear un código QR válido, el sistema recupera la información del alumno y redirige a la \IUref{IU11}{Pantalla Credencial del alumno}. En caso de error, se muestran mensajes.

\subsubsection{Entradas}
La imagen del código QR capturada por la cámara del dispositivo.

\subsubsection{Comandos}
\begin{itemize}
	\item \textbf{Escanear (automático al detectar el QR):} Al detectar un código QR, el sistema intenta leerlo y procesarlo.
	\item \textbf{Icono de calendario (barra de navegación inferior):} Redirige a la pantalla \IUref{IU02}{Consultar calendario escolar}.
	\item \textbf{Icono de campana (barra de navegación inferior):} Redirige a la pantalla de notificaciones.
	\item \textbf{Icono de casa (barra de navegación inferior):} Redirección a la pantalla de saludo correspondiente al tipo de usuario.
	\item \textbf{Icono de flecha saliendo (barra de navegación inferior):} Cierra la sesión del usuario y lo regresa a la pantalla de inicio de sesión (\IUref{IU01}{Iniciar sesión del sistema móvil}).
\end{itemize}

\subsubsection{Mensajes}
\begin{itemize}
	\item \textbf{\hyperref[msg:CU09-E2]{MSG-CU09-E2}} Código QR no válido.
\end{itemize}


