% !TeX root = ../ejemplo.tex

%--------------------------------------
\subsection{IU21 Dar de alta a alumno}

\subsubsection{Objetivo}
	Permitir al personal de la DAE dar de alta a un alumno.
\subsubsection{Diseño}
    Esta pantalla \IUref{IU21}{ Dar de alta a alumno } (ver figura~\ref{IU21}) puede ser accedida desde cualquiera de las pantallas del personal de la DAE (\IUref{IU21\_2}{ Consultar alumnos}, \IUref{IUE04}{ Pantalla inicial de personal de la DAE}) apretando el botón \IUbutton{Dar de alta alumnos }.

\IUfig[1]{UI-CU21}{IU21}{ Dar de alta a alumno.}

\subsubsection{Salidas}
Muestra mensaje {\bf MSG-31} ``Alumno dado de alta con éxito''.
\subsubsection{Entradas}
    Video, CURP, boleta, nombre, apellido paterno, apellido materno, sexo, escuela asignada, correo institucional y carrera en la que se encuentra el alumno.
\subsubsection{Comandos}
\begin{itemize}
	\item \IUbutton{Inicio}: Redirige al personal de la DAE a la pantalla \IUref{IUE04}{Pantalla Inicial de personal de la DAE}
	\item \IUbutton{Dar de alta alumnos}: Redirige al personal de la DAE a la pantalla \IUref{IU21}{Dar de alta a los estudiantes}
	\item \IUbutton{Consultar alumnos}: Redirige al personal de la DAE a la pantalla \IUref{IU46}{Consultar listado de alumnos}
	
    \item \IUbutton{Guardar}  El sistema revisa que los datos del alumno sean válidos, verifica que el CURP o la boleta no hayan sido registrados con anterioridad y en caso de que no, se registra al alumno.
    
\end{itemize}

\subsubsection{Mensajes}

\begin{Citemize}
    \item {\bf MSG-31} ``Alumno dado de alta con éxito''.
    \item {\bf MSG-29}{``Los campos no están correctamente llenados.''}
    \item {\bf MSG-30}{``La CURP o la boleta ya han sido asociadas a este alumno con anterioridad u otro alumno.''}
\end{Citemize}
