% !TeX root = ../ejemplo2.tex

%--------------------------------------
\subsection{IU01: Pantalla Iniciar sesión del sistema móvil}

\IUfig[.40]{UI-CU01}{IU01}{Pantalla Iniciar sesión del sistema móvil.}

\newpage

\subsubsection{Objetivo}
Controlar el acceso al sistema móvil, permitiendo a los usuarios registrados (docentes, personal de seguridad, alumnos, presidentes de academia y jefes de departamento) autenticarse mediante sus credenciales correspondientes.

\subsubsection{Diseño}
Esta pantalla \IUref{IU01}{Pantalla Iniciar sesión del sistema móvil} (ver figura~\ref{IU01}) es la primera que se muestra al iniciar la aplicación. Permite a los diferentes tipos de usuarios ingresar sus datos de autenticación.



La pantalla contiene los siguientes elementos:
\begin{itemize}
	\item \textbf{Título:} "Iniciar sesión".
	\item \textbf{Campo "Boleta":} Para que los alumnos ingresen su número de boleta. Tiene un icono de usuario asociado.
	\item \textbf{Campo "Contraseña":} Para que todos los usuarios ingresen su contraseña. Tiene un icono de candado asociado.
	\item \textbf{Botón \IUbutton{Entrar}:} Permite iniciar la sesión una vez que se han introducido las credenciales.
\end{itemize}

\subsubsection{Salidas}
Redirección a la pantalla de saludo correspondiente al rol del usuario autenticado, mostrando un saludo y su nombre. Las posibles pantallas de saludo son:
\begin{itemize}
	\item \IUref{IUE01}{Pantalla de saludo del docente} (para docentes, presidentes de academia y jefes de departamento).
	\item \IUref{IUE02}{Pantalla de saludo del personal de seguridad}.
	\item \IUref{IUE03}{Pantalla de saludo del alumno}.
	\item \IUref{IUE06}{Pantalla de saludo del presidente de academia/jefe de departamento} (también para presidentes de academia y jefes de departamento).
\end{itemize}

\subsubsection{Entradas}
Dependiendo del rol del usuario:
\begin{itemize}
	\item \textbf{Alumno:} Número de boleta y Contraseña.
	\item \textbf{Personal de seguridad:} CURP y Contraseña.
	\item \textbf{Docente, Presidente de academia, Jefe de departamento:} RFC y Contraseña.
\end{itemize}
	
	\subsubsection{Comandos}
	\begin{itemize}
		\item \IUbutton{Entrar}:
		\begin{enumerate}
			\item Verifica que se hayan llenado todos los campos (Boleta y Contraseña). Si falta algún campo, muestra el mensaje \textbf{``Por favor, completa todos los campos.''}.
			\item Verifica que las credenciales (Boleta y Contraseña) coincidan con un usuario registrado en el sistema. Si no coinciden, muestra el mensaje \textbf{ ``Datos incorrectos.''.}
			\item En caso de pérdida de conexión durante la verificación, muestra el mensaje \textbf{ ``Conexión perdida.''.}
			\item Si la autenticación es exitosa, determina el rol del usuario y lo redirige a la pantalla de saludo correspondiente (\IUref{IUE01}{Pantalla de saludo del docente}, \IUref{IUE02}{Pantalla de saludo del personal de seguridad}, \IUref{IUE03}{Pantalla de saludo del alumno} o \IUref{IUE06}{Pantalla de saludo del presidente de academia/jefe de departamento}).
		\end{enumerate}
		
	\end{itemize}
	
	\subsubsection{Mensajes}
	\begin{itemize}
		\item Por favor, completa todos los campos.
		\item Datos incorrectos.
		\item Conexión perdida.
	\end{itemize}


