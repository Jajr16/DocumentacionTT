% !TeX root = ../ejemplo.tex

%--------------------------------------
\subsection{IU01 Pantalla Iniciar sesión de personal escolar móvil} 

\subsubsection{Objetivo}
	Controlar el acceso al sistema mediante una contraseña a fin de que cada usuario acceda solo a las operaciones permitidas para su perfil.

\subsubsection{Diseño}
	Esta pantalla \IUref{IU01}{Pantalla Iniciar sesión de personal escolar móvil} (ver figura~\ref{IU01}) aparece al iniciar el sistema para los empleados. Para ingresar al mismo se debe escribir el RFC del empleado y la contraseña. 

\IUfig[.35]{UI-CU01}{IU01}{Pantalla Iniciar sesión de personal escolar móvil.}

\subsubsection{Salidas}

Saludo del sistema y mención de su nombre.

\subsubsection{Entradas}
En caso del empleado RFC, Contraseña y en caso del alumno CURP, Contraseña

\subsubsection{Comandos}
\begin{itemize}
	\item \IUbutton{Entrar}: Verifica que el empleado se encuentre registrado y la contraseña sea la correcta. Si la verificación es correcta, se verifica que tipo de empleado y se muestra la pantalla \IUref{IUE01}{Pantalla de Menús de docente} si es docente o \IUref{IUE02}{Pantalla de Menús de personal de seguridad} si es personal de seguridad.
	
	\item \IUbutton{Presiona aquí para pedir su activación}: Redirige a la pantalla \IUref{UI32}{Pantalla de Solicitar desbloqueo de cuenta}
	
\end{itemize}

\subsubsection{Mensajes}

\begin{Citemize}
	\item MSG-1 Los campos no están correctamente llenados. 
	\item MSG-2 Su cuenta esta bloqueada. 
	\item MSG-3 El RFC o la contraseña no corresponden con ningún empleado. 
	\item MSG-4 El proceso no se pudo realizar por un fallo de red. 
	\item MSG-5 Su cuenta ha sido bloqueada por la gran cantidad de intentos de inicio sesión fallidos.
\end{Citemize}

