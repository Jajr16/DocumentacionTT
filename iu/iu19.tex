%--------------------------------------
\subsection{IU18 Pantalla de Detalles del proceso de ETS}

\IUfig[.35]{CU20}{IU18}{Pantalla de Detalles del proceso de ETS}
\label{IU18}
\newpage

\subsubsection{Objetivo}
Permitir al alumno visualizar la información detallada sobre los pasos a seguir para la inscripción al ETS.

\subsubsection{Diseño}
Esta pantalla \IUref{IU18}{Pantalla de Detalles del proceso de ETS} (ver figura~\ref{IU18}) aparece luego de seleccionar el botón \IUbutton{Información de Acceso} en la \IUref{IUE03}{Pantalla saludo del alumno}.

La pantalla contiene los siguientes elementos:
\begin{itemize}
	\item \textbf{Barra de navegación superior:}
	\begin{itemize}
		\item \textbf{Icono de flecha hacia la izquierda:} Para regresar a la pantalla anterior (\IUref{IUE03}{Pantalla saludo del alumno}).
		\item \textbf{Título:} "Guía de Inscripción al ETS".
		\item \textbf{Icono de tres puntos horizontales con una burbuja de diálogo:} Accede a la funcionalidad de mensajes dentro de la aplicación.
	\end{itemize}
	\item \textbf{Lista de pasos para la inscripción al ETS:} Muestra los pasos numerados para la inscripción, cada uno dentro de un recuadro con un fondo claro:
	\begin{itemize}
		\item \textbf{1.} Pagar en caja y verificar que estén correctos los siguientes datos: Nombre, Boleta, Carrera y Número de unidades de aprendizaje.
		\item \textbf{2.} Acudir a ventanilla de gestión escolar para generar créditos en el "SAES".
		\item \textbf{3.} Una vez generados los créditos, inscribir las unidades de aprendizaje en la página del "SAES".
		\item \textbf{4.} Entregar en ventanilla de gestión escolar el comprobante de inscripción de ETS generado por SAES y el recibo de pago para finalizar la inscripción al ETS.
		\item \textbf{5.} Acudir el día y la hora establecida en el calendario.
	\end{itemize}
	\item \textbf{Barra de navegación inferior:} Contiene iconos para:
	\begin{itemize}
		\item \textbf{Icono de casa:} Redirección a la (\IUref{IUE03}{Pantalla de saludo del alumno}.
		\item \textbf{Icono de campana:} Redirección a la pantalla de notificaciones.
		\item \textbf{Icono de calendario con marcas:} Redirección a la pantalla \IUref{IU02}{Consultar calendario escolar}.
		\item \textbf{Icono de flecha apuntando hacia la derecha saliendo de un recuadro:} Cierra la sesión del usuario y lo regresa a la pantalla de inicio de sesión (\IUref{IU01}{Iniciar sesión del sistema móvil}).
	\end{itemize}
\end{itemize}

\subsubsection{Salidas}
Información detallada de los pasos para la inscripción al ETS.

\subsubsection{Entradas}
Ninguna.

\subsubsection{Comandos}
\begin{itemize}
	\item \textbf{Icono de flecha izquierda (barra superior):} Regresa a la pantalla anterior (\IUref{IUE03}{Pantalla saludo del alumno}).
	\item \textbf{Icono de tres puntos (barra superior):} Accede a la funcionalidad de mensajes dentro de la aplicación.
	\item \textbf{Icono de calendario (barra de navegación inferior):} Redirige a la pantalla \IUref{IU02}{Consultar calendario escolar}.
	\item \textbf{Icono de campana (barra de navegación inferior):} Redirige a la pantalla de notificaciones.
	\item \textbf{Icono de casa (barra de navegación inferior):} Redirección a la pantalla de saludo correspondiente al tipo de usuario.
	\item \textbf{Icono de flecha saliendo (barra de navegación inferior):} Cierra la sesión del usuario y lo regresa a la pantalla de inicio de sesión (\IUref{IU01}{Iniciar sesión del sistema móvil}).
\end{itemize}

\subsubsection{Mensajes}
\begin{itemize}
	\item \textbf{\hyperref[msg:CU20-E1]{MSG-CU20-E1}} Error al recuperar la información del proceso. Intente nuevamente más tarde.
\end{itemize}

