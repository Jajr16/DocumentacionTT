%--------------------------------------
\subsection{IU08: Pantalla Lista de asistencia de ETS}

\IUfig[.40]{cu08}{IU08}{Pantalla Lista de asistencia de ETS}
\newpage

\subsubsection{Objetivo}
Permitir al docente visualizar la lista de los alumnos inscritos en un ETS asignado, junto con su estado de asistencia, y acceder a las funciones de reporte dentro del periodo permitido.

\subsubsection{Diseño}
Esta pantalla \IUref{IU08}{Pantalla Lista de asistencia de ETS} (ver figura~\ref{IU08}) muestra la lista de alumnos inscritos en el ETS seleccionado. Se accede a ella al presionar el botón "Ir a la lista de alumnos" desde la \IUref{IU06}{Pantalla Información de ETS}.



La pantalla contiene los siguientes elementos:
\begin{itemize}
	\item \textbf{Barra de navegación superior:}
	\begin{itemize}
		\item \textbf{Icono de flecha hacia la izquierda:} Para regresar a la pantalla anterior (\IUref{IU06}{Pantalla Información de ETS}).
		\item \textbf{Título:} "ETS de Bases de Datos\newline 25/2\newline Lista de alumnos inscritos".
		\item \textbf{Icono de tres puntos horizontales con una burbuja de diálogo:} Accede a la funcionalidad de mensajes dentro de la aplicación.
	\end{itemize}
	\item \textbf{Lista de alumnos inscritos:} Muestra a los alumnos inscritos en el ETS en forma de tarjetas o filas. Para cada alumno se muestra:
	\begin{itemize}
		\item \textbf{Boleta y Nombre:} Presentados como texto. Este elemento, al ser presionado (si está habilitado), redirige a la \textbf{Pantalla Reporte} (\IUref{IUE07}{Creación del reporte}).
		\item \textbf{Icono de estado de asistencia:} Un icono visual que representa el estado de asistencia del alumno (ej., una "X" roja o una paloma verde). Al presionarlo, redirige a la \IUref{IU20}{Pantalla mostrar la foto e información del alumno} (solo dentro del periodo de tiempo especificado).
	\end{itemize}
	\item \textbf{Barra de navegación inferior:} Contiene iconos para:
	\begin{itemize}
		\item \textbf{Icono de casa:} Redirección a la pantalla de saludo correspondiente al tipo de usuario.
		\item \textbf{Icono de campana:} Redirección a la pantalla de notificaciones.
		\item \textbf{Icono de calendario con marcas:} Redirección a la pantalla \IUref{IU02}{Consultar calendario escolar}.
		\item \textbf{Icono de flecha apuntando hacia la derecha saliendo de un recuadro:} Cierra la sesión del usuario y lo regresa a la pantalla de inicio de sesión (\IUref{IU01}{Iniciar sesión del sistema móvil}).
	\end{itemize}
\end{itemize}

\subsubsection{Salidas}
Muestra la lista de los alumnos inscritos en el ETS, incluyendo su boleta, nombre y un icono de estado de asistencia. Puede mostrar mensajes informativos sobre el periodo de reporte y los permisos del docente.

\subsubsection{Entradas}
Ninguna directa por parte del usuario en esta pantalla, ya que la lista se muestra al acceder a ella.

\subsubsection{Comandos}
\begin{itemize}
	\item \textbf{Boleta y Nombre del alumno (al ser presionado):} Si está habilitado (dentro del periodo de reporte y si el docente tiene permisos), redirige a la \textbf{Pantalla Reporte} (\IUref{IUE07}{Creación del reporte}).
	\item \textbf{Icono de estado de asistencia (al ser presionado):} Redirige a la \IUref{IU20}{Pantalla mostrar la foto e información del alumno} (solo dentro del periodo de tiempo especificado).
	\item \textbf{Icono de calendario (barra de navegación inferior):} Redirige a la pantalla \IUref{IU02}{Consultar calendario escolar}.
	\item \textbf{Icono de campana (barra de navegación inferior):} Redirige a la pantalla de notificaciones.
	\item \textbf{Icono de casa (barra de navegación inferior):} Redirección a la pantalla de saludo correspondiente al tipo de usuario.
	\item \textbf{Icono de flecha saliendo (barra de navegación inferior):} Cierra la sesión del usuario y lo regresa a la pantalla de inicio de sesión (\IUref{IU01}{Iniciar sesión del sistema móvil}).
	\item \textbf{Icono de flecha izquierda (barra superior):} Regresa a la pantalla anterior (\IUref{IU06}{Pantalla Información de ETS}).
	\item \textbf{Icono de tres puntos (barra superior):} Accede a la funcionalidad de mensajes dentro de la aplicación.
\end{itemize}

\subsubsection{Mensajes}
Los mensajes que se pueden mostrar en esta pantalla, según el caso de uso CU-08, son:
\begin{itemize}
	\item \textbf{\hyperref[msg:CU08-E2]{MSG-CU08-E2}} Error al consultar la base de datos. Intente nuevamente más tarde.
	\item \textbf{\hyperref[msg:CU08-A1]{MSG-CU08-A1}} No hay alumnos inscritos al ETS.
	\item \textbf{\hyperref[msg:CU08-C1]{MSG-CU08-C1}} El ETS seleccionado no es válido.
	\item \textbf{\hyperref[msg:CU08-D1]{MSG-CU08-D1}} Aún no es periodo para crear los reportes. Faltan (tiempo).
	\item \textbf{\hyperref[msg:CU08-ET2]{MSG-CU08-ET2}} El periodo para registrar los reportes ha concluido.
	\item \textbf{\hyperref[msg:CU08-F1]{MSG-CU08-F1}} Usted no está autorizado para crear el reporte de este alumno.
\end{itemize}