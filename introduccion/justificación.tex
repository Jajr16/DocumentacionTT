% !TeX root = ../ejemplo.tex

\section{Justificación}

La implementación de un sistema de control de acceso basado en técnicas biométricas, como el reconocimiento facial, en conjunto con el uso de métodos de autenticación convencionales y ya establecidos como lo es el uso de las credenciales escolares, representa una innovación significativa en el ámbito educativo, particularmente en el contexto de los Exámenes a Título de Suficiencia (ETS). Este proyecto se justifica no solo por su capacidad de abordar retos específicos, sino también por la oportunidad que ofrece para explorar y aplicar tecnologías emergentes en el campo de la Inteligencia Artificial (IA) y la Visión por Computadora.\\

El uso de reconocimiento facial como método biométrico es especialmente relevante en el contexto educativo porque aporta precisión y fiabilidad en la verificación de identidad. Además, es un método de autenticación no invasivo y difícil de vulnerar en tiempo real ante la presencia del personal humano. Según estudios recientes, las tecnologías de autenticación biométrica presentan tasas de error mucho menores comparadas con métodos tradicionales, como contraseñas o credenciales físicas, las cuales pueden ser manipuladas fácilmente \cite{L06}.\\

Desde una perspectiva educativa y tecnológica, este proyecto también tiene un valor en la formación de profesionales en el área de la inteligencia artificial. Implementar un sistema basado en Visión por Computadora y aprendizaje profundo permite a los desarrolladores ampliar sus conocimientos en tecnologías avanzadas. Estas habilidades son esenciales en un entorno global donde la IA está transformando rápidamente múltiples sectores, incluyendo el educativo. Incluso en algunos países como Estados Unidos y Australia este tipo de tecnologías ya han sido implementadas para cuestiones relacionadas con la seguridad dentro de los planteles \cite{L04}.\\

En términos de beneficios para la comunidad que integra a la ESCOM, este proyecto no solo refuerza la seguridad y transparencia de los procesos evaluativos, sino que también genera confianza en la comunidad educativa. Profesores, estudiantes y personal administrativo podrán realizar sus actividades en un entorno más seguro, minimizando riesgos asociados con el acceso no autorizado \cite{L05}. \\

Finalmente, la integración de estas tecnologías en dispositivos tan accesibles como los son los teléfonos celulares permite su escalabilidad a otros procesos académicos que requieran de la validación de la identidad de los alumnos o de llevar un control sobre las personas que entran y salen de las instalaciones, lo que a su vez puede influir en la adopción de este tipo de tecnologías en otros sectores públicos.
