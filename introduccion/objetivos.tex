% !TeX root = ../ejemplo.tex
\section{Objetivo general}

Desarrollar un sistema de autenticación y control de acceso de alumnos con el motivo de evitar posibles casos de suplantación de identidad durante la aplicación de Exámenes a Título de Suficiencia (ETS) en la Escuela Superior de Cómputo (ESCOM), mediante el uso de tecnologías de reconocimiento facial, dispositivos móviles y credenciales escolares.


\subsection{Objetivos específicos}

\begin{itemize}
	\item Explorar el estado del arte en cuanto a técnicas para el reconocimiento facial y recuperación de datos (rostros).
	\item Evaluar distintos algoritmos y técnicas de reconocimiento facial para determinar aquellas que se ajusten mejor a las necesidades y recursos del proyecto.
	\item Desarrollar una aplicación móvil para estudiantes y docentes que integre la tecnología de reconocimiento facial y proporcione un acceso seguro a los servicios relacionados con los exámenes (consulta de horarios), incluida la verificación de la identidad.
	\item Diseñar e implementar un módulo de reconocimiento facial capaz de verificar la identidad de los estudiantes durante las fechas de aplicación de los ETS, garantizando que solo los estudiantes inscritos puedan presentar el examen.
	\item Implementar una sección dentro de la aplicación móvil que permita reforzar el control de acceso a las instalaciones durante los días de aplicación de ETS con la ayuda de códigos QR.
	\item Desarrollar un sistema web básico para la gestión de ETS y que permita la integración de una API para simular escenarios reales con el fin de demostrar y evaluar la funcionalidad del sistema.
	\item Evaluar el desempeño del prototipo de sistema de control de acceso en cuanto a precisión, confiabilidad y seguridad.
	
\end{itemize}

