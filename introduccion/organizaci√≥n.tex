% !TeX root = ../ejemplo.tex

\section{Organización del documento}

El presente documento se encuentra estructurado por los siguientes capítulos:

En el capítulo 1 se encuentran los antecedentes que permitieron identificar la problemática que rige este proyecto, lo que a su vez nos guío al análisis de esta misma para sustentar la propuesta de solución planteada a lo largo de este documento y su justificación desde un contexto social, personal y académico.\\
 
En el capítulo 2 se encuentra el estado del arte, en donde se revisaron y analizaron trabajos anteriores con similitudes al proyecto a realizar, esto con el fin de realizar una comparación de las principales características e identificar posibles áreas de mejora.\\

En el capítulo 3 se explican los términos, descripciones y demás elementos importantes que se relacionan en menor o mayor medida con este trabajo a fin de tener un panorama general sobre la terminología utilizada.\\

En el capítulo 4, se sustenta la metodología de trabajo empleada, se presentan los resultados del estudio de factibilidad realizado y el análisis del sistema que compone la propuesta de solución a fin de delimitar el alcance de este y los elementos a desarrollar, asimismo, se presenta el análisis de riesgos correspondiente.\\

En el capítulo 5, se muestran las decisiones de diseño tomadas para este proyecto, esto se encuentra representado a través de diagramas para ofrecer una visualización y explicación del funcionamiento del sistema y las interacciones que existen entre los componentes que lo conforman.\\


Finalmente en el capítulo 6 se encuentra en trabajo futuro identificado para este trabajo terminal 1, destacando la implementación de la aplicación móvil, el sistema para la gestión de alumnos, personal y exámenes ETS así como la implementación del sistema de reconocimiento facial en tiempo real. Se espera que este trabajo se desarrolle y culmine durante la segunda etapa del trabajo terminal.