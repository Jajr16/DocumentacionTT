\begin{UseCase}{CU-15}{Registrar ingreso a la instalación}{
	Este caso de uso permite al personal de seguridad registrar la entrada de los alumnos mediante el escaneo de credenciales y la búsqueda por nombre o boleta.
}
\UCitem{Versión}{1.0}
\UCitem{Autor}{\color{Gray}De la cruz De la cruz Alejandra}
\UCitem{Supervisa}{\color{Gray}Huertas Ramírez Daniel Martín}
\UCitem{Actor}{\hyperlink{PS}{Personal de Seguridad}}
\UCitem{Propósito}{Facilitar el registro de entrada de los alumnos a las instalaciones.}
\UCitem{Entradas}{Nombre del alumno o número de boleta y Escaneo de la credencial del alumno.}
\UCitem{Origen}{Pantalla táctil y Cámara con lector de códigos QR}
\UCitem{Salidas}{Confirmación de entrada registrada.}
\UCitem{Destino}{Pantalla del sistema.}
\UCitem{Precondiciones}{
		El sistema debe tener acceso a la base de datos de alumnos.
		
		El personal de seguridad debe estar autenticado en el sistema.

}
\UCitem{Postcondiciones}{
		Se registra la entrada del alumno en el sistema.
}
\UCitem{Errores}{
		E1: Si la información de la credencial no coincide con la información del sistema se mostrara el mensaje  {\bf ``Los datos no coinciden. No se puede registrar la asistencia con datos inconsistentes''}.


		E2: Si el alumno no tiene un ETS registrado para esa fecha o su ETS esta programada para otra fecha se mostrara el mensaje {\bf ``Error el alumno no cuenta con ETS inscritos ó su ETS está programada en otra fecha''}.
		
		
		E2: Si ya se cuenta con un registro previamente se mostrara el mensaje {\bf ``El alumno ya tiene registrada su asistencia el día de hoy''}.

}
\UCitem{Tipo}{Se entiende del CU-12 Consultar alumno mediante código QR de la credencial, CU-13 Buscar alumno por boleta y del CU-14 Buscar alumno por nombre }
\UCitem{Observaciones}{Ninguno}
\end{UseCase}
%--------------------------------------
\begin{UCtrayectoria}
	\UCpaso[\UCactor] El personal de seguridad selecciona al alumno desde la \IUref{IU12}{Pantalla Buscar alumno} o escanea la codigo QR de la credencial del alumno desde la \IUref{IU10}{Pantalla Código QR}.
	\UCpaso[\UCactor] El personal de seguridad escanea el código QR de la credencial del alumno.
	\UCpaso El sistema verifica la existencia del alumno en la base de datos.
	\UCpaso El sistema compara los datos del QR con los registros almacenados en la base de datos \Trayref{A}.
	\UCpaso El sistema verifica que el alumno tenga un ETS vigente \Trayref{B}.
	\UCpaso El sistema comprueba que no tenga asistencia registrada previamente \Trayref{C}.
	\UCpaso El sistema muestra:
	\begin{itemize}
		\item Datos del alumno 
		\item Información del ETS inscrito
		\item Hora actual del registro
	\end{itemize}
	\UCpaso[\UCactor] El personal de seguridad verifica la identidad del alumno.
	\UCpaso El sistema muestra diálogo de confirmación:
	\begin{itemize}
		\item Registrar asistencia 
	\end{itemize}
	\UCpaso[\UCactor] El personal de seguridad selecciona \IUbutton{Registrar asistencia}.
	\UCpaso El sistema muestra confirmación: {\bf ``Entrada registrada exitosamente.''}
\end{UCtrayectoria}
%--------------------------------------
\begin{UCtrayectoriaA}{A}{Datos inconsistentes}
	\UCpaso El sistema detecta inconsistencias entre los datos del QR y el sistema.
	\UCpaso El sistema muestra mensaje: {\bf ``Los datos no coinciden. No se puede registrar la asistencia con datos inconsistentes''}.
	\UCpaso[\UCactor] El personal de seguridad informa al alumno.
\end{UCtrayectoriaA}
%--------------------------------------
\begin{UCtrayectoriaA}{B}{Alumno sin ETS vigente}
	\UCpaso El sistema detecta que el alumno no tiene ETS vigente.
	\UCpaso El sistema muestra mensaje: {\bf ``Error el alumno no cuenta con ETS inscritos ó su ETS está programada en otra fecha''}.
	\UCpaso[\UCactor] El personal de seguridad informa al alumno.
	\UCpaso Fin del caso de uso.
\end{UCtrayectoriaA}
%--------------------------------------        
\begin{UCtrayectoriaA}{C}{Asistencia ya registrada}
	\UCpaso El sistema detecta asistencia previa del alumno en la misma fecha.
	\UCpaso El sistema muestra mensaje: {\bf ``El alumno ya tiene registrada su asistencia el día de hoy''}.
	\UCpaso[\UCactor] El personal de seguridad verifica si es un error.
	\UCpaso[\UCactor] El personal de seguridad puede:
	\begin{itemize}
		\item Registrar nuevamente (continúa al paso 7)
		\item Cancelar el registro
	\end{itemize}
\end{UCtrayectoriaA}
%--------------------------------------        

