% \IUref{IUAdmPS}{Administrar Planta de Selección}
% \IUref{IUModPS}{Modificar Planta de Selección}
% \IUref{IUEliPS}{Eliminar Planta de Selección}

%-------------------------------------- COMIENZA descripción del caso de uso.

%\begin{UseCase}[archivo de imágen]{UCX}{Nombre del Caso de uso}{
%--------------------------------------
\label{CU-08}
\begin{UseCase}{CU-08}{Consultar lista de alumnos inscritos a un ETS}{
		Este caso de uso permite al docente consultar la lista de los alumnos inscritos a un ETS asignado.
		
		Ademas, el caso de uso CU-10 (Consultar lista de asistencia de alumnos inscritos a los ETS) se ha unificado con CU-08 (Consultar lista de alumnos inscritos a un ETS) para mejorar la eficiencia y la experiencia del usuario. La funcionalidad de visualización de la asistencia, previamente contemplada en CU-10, se integra ahora directamente en la \IUref{IU08}{Pantalla Lista de asistencia de ETS} de CU-08, presentando la lista de alumnos y su estado de asistencia en una única interfaz.
	}
	\UCitem{Versión}{\color{Gray}2.0}
	\UCitem{Autor}{\color{Gray}De la cruz De la cruz Alejandra}
	\UCitem{Supervisa}{\color{Gray}Huertas Ramírez Daniel Martín}
	\UCitem{Actor}{\hyperlink{tDocente}{Docente}}
	\UCitem{Propósito}{Permitir al docente visualizar la lista de alumnos inscritos en un ETS para verificar su asistencia y acceder a la creación de reportes dentro del periodo permitido.}
	\UCitem{Entradas}{Solicitud de ver la lista de alumnos.}
	\UCitem{Origen}{\IUref{IU06}{Pantalla Información de ETS}}
	\UCitem{Salidas}{Lista de los alumnos inscritos al ETS (Boleta y Nombre) junto con un icono representativo del estado de su asistencia. Mensajes informativos sobre el periodo de reporte y permisos.}
	\UCitem{Destino}{\IUref{IU08}{Pantalla Lista de asistencia de ETS}.}
	\UCitem{Precondiciones}{El docente debe haber iniciado sesión (\textbf{\hyperref[cu:CU-01]{CU-01 Iniciar sesión del sistema móvil}}) y haber consultado la información detallada del ETS (\textbf{\hyperref[cu:CU-06]{CU-06 Mostrar información de los ETS asignados}}).}
	\UCitem{Postcondiciones}{El docente ha visualizado la lista de asistencia de los alumnos inscritos al ETS y puede acceder a la creación de reportes si se encuentra dentro del periodo permitido y tiene los permisos necesarios.}
	\UCitem{Errores}{
		E1: El sistema pierde la conexión al intentar recuperar la lista de asistencia y se muestra el mensaje \textbf{ ``Error al consultar la base de datos. Intente nuevamente más tarde.''}.
	}
	\UCitem{Tipo}{Caso de uso primario}
	\UCitem{Observaciones}{Cada alumno en la lista se presenta con dos botones: uno con su boleta y nombre, y otro con un icono representativo de su estado de asistencia. El botón con la boleta y nombre del alumno redirige a la \textbf{Pantalla Reporte} (\IUref{IUE07}{Creación del reporte}). El botón con el icono redirige a la \IUref{IU20}{Pantalla mostrar la foto e información del alumno} **solo si la hora actual se encuentra entre 10 minutos antes del inicio del ETS y 2 horas después del inicio del ETS.** Si se quiere conocer todos los iconos y su significado, porfavor revisar \textbf{\hyperref[sec:tablaIconosAsistencia]{Tabla de Iconos de Estado de Asistencia}.}}
\end{UseCase}
%--------------------------------------
\begin{UCtrayectoria}
	\UCpaso[\UCactor] El docente presiona el botón \IUbutton{Ir a la lista de alumnos} desde la pantalla \IUref{IU06}{Pantalla Información de ETS}.
	\UCpaso El sistema verifica la existencia del ETS. \Trayref{C}
	\UCpaso El sistema recupera la lista de alumnos inscritos en el ETS junto con su estado de asistencia. \Trayref{A}, \Trayref{B}
	\UCpaso El sistema muestra la lista de alumnos en la \IUref{IU08}{Pantalla Lista de asistencia de ETS}. Cada alumno se muestra con un botón que contiene su boleta y nombre (habilitado según el periodo de reporte y rol del docente), y un botón con el icono representativo de su estado de asistencia.
	\UCpaso[\UCactor] El docente visualiza la lista de alumnos.
	\UCpaso Para cada alumno, el sistema verifica si la hora actual se encuentra dentro del periodo permitido para generar reportes (10 minutos antes hasta 2 horas después del inicio del ETS). \Trayref{D}, \Trayref{E}
	\UCpaso El sistema verifica si el docente es el aplicador del ETS para el alumno seleccionado antes de permitir el acceso a la creación del reporte. \Trayref{F}
	\UCpaso[\UCactor] Si el periodo es válido y el docente es el aplicador, al presionar el botón con la boleta y nombre del alumno, el sistema lo redirige a la \textbf{Pantalla Reporte} (\IUref{IUE07}{Creación del reporte}).
	\UCpaso[\UCactor] Al presionar el botón con el icono, el sistema lo redirige a la \IUref{IU20}{Pantalla mostrar la foto e información del alumno}.
\end{UCtrayectoria}

%--------------------------------------
\begin{UCtrayectoriaA}{A}{El ETS no tiene alumnos inscritos}
	\UCpaso El sistema muestra el mensaje: \textbf{ ``No hay alumnos inscritos al ETS.''}.
	\UCpaso[\UCactor] El docente presiona el botón \IUbutton{Regresar} para volver a la pantalla \IUref{IU06}{Pantalla Información de ETS}.
	\UCpaso Fin de la trayectoria alternativa.
\end{UCtrayectoriaA}
%--------------------------------------
\begin{UCtrayectoriaA}{B}{Error en la conexión con la base de datos}
	\UCpaso El sistema intenta recuperar la lista de alumnos inscritos.
	\UCpaso Ocurre un error en la conexión con la base de datos.
	\UCpaso El sistema muestra el mensaje de error: \textbf{ ``Error al consultar la base de datos. Intente nuevamente más tarde.''}.
	\UCpaso[\UCactor] El docente presiona el botón \IUbutton{Aceptar} para cerrar el mensaje.
	\UCpaso[\UCactor] El docente puede intentar la consulta nuevamente.
	\UCpaso Fin de la trayectoria alternativa.
\end{UCtrayectoriaA}
%--------------------------------------
\begin{UCtrayectoriaA}{C}{El ETS no existe}
	\UCpaso El sistema no encuentra el ETS correspondiente.
	\UCpaso El sistema muestra el mensaje de error: \textbf{ ``El ETS seleccionado no es válido.''}.
	\UCpaso[\UCactor] El docente presiona el botón \IUbutton{Regresar} para volver a la pantalla \IUref{IU06}{Pantalla Información de ETS}.
	\UCpaso Fin de la trayectoria alternativa.
\end{UCtrayectoriaA}
%--------------------------------------
\begin{UCtrayectoriaA}{D}{Aún no es periodo para crear reportes}
	\UCpaso El sistema verifica que la hora actual es anterior a 10 minutos antes de la hora de inicio del ETS.
	\UCpaso El sistema muestra el mensaje: \textbf{ ``Aún no es periodo para crear los reportes. Faltan (tiempo)''}.
	\UCpaso El botón con la boleta y nombre del alumno se muestra deshabilitado o no interactivo.
	\UCpaso[\UCactor] El docente visualiza el mensaje.
	\UCpaso Fin de la trayectoria alternativa.
\end{UCtrayectoriaA}
%--------------------------------------
\begin{UCtrayectoriaA}{E}{El periodo para registrar los reportes ha concluido}
	\UCpaso El sistema verifica que la hora actual es posterior a 2 horas después de la hora de inicio del ETS.
	\UCpaso El sistema muestra el mensaje: \textbf{ ``El periodo para registrar los reportes ha concluido.''}.
	\UCpaso El botón con la boleta y nombre del alumno se muestra deshabilitado o no interactivo.
	\UCpaso[\UCactor] El docente visualiza el mensaje.
	\UCpaso Fin de la trayectoria alternativa.
\end{UCtrayectoriaA}
%--------------------------------------
\begin{UCtrayectoriaA}{F}{Docente no autorizado para ver información del alumno}
	\UCpaso El sistema verifica que el docente no es el aplicador del ETS para el alumno seleccionado.
	\UCpaso El sistema muestra el mensaje: \textbf{ ``Usted no está autorizado para crear el reporte de este alumno.''}.
	\UCpaso El botón con la boleta y nombre del alumno se muestra deshabilitado o no interactivo.
	\UCpaso[\UCactor] El docente visualiza el mensaje.
	\UCpaso Fin de la trayectoria alternativa.
\end{UCtrayectoriaA}

%-------------------------------------- TERMINA descripción del caso de uso.
