% !TeX root = ../ejemplo.tex

%--------------------------------------
\begin{UseCase}{CU-33}{Dar de alta personal de seguridad}{

    Permitir al personal de gestión escolar dar de alta un personal de seguridad.
}
    \UCitem{Versión}{\color{Gray}1}
    \UCitem{Autor}{\color{Gray}Huertas Ramírez Daniel Martín}
    \UCitem{Supervisa}{\color{Gray}De la cruz De la cruz Alejandra.}
    \UCitem{Actor}{Personal de gestión escolar}
    \UCitem{Propósito}{ Permitir al personal de gestión escolar dar de alta a un personal de seguridad.}
    \UCitem{Entradas}{CURP, Turno, Cargo, Sexo y Nombre}
    \UCitem{Origen}{Teclado}
    \UCitem{Salidas}{Muestra mensaje {\bf MSG-37} ``Personal de seguridad dado de alta con éxito''.}
    \UCitem{Destino}{Pantalla \IUref{IU28}{Consultar lista de personal de seguridad}}
    \UCitem{Precondiciones}{El Personal de gestión escolar debe de haber iniciado sesión.}
    \UCitem{Postcondiciones}{El personal de seguridad es dado de alta y sus datos se guardan en la base de datos.}
    \UCitem{Errores}{

        E1: Cuando se pierde la conexión durante el proceso, los procesos se cancelan y se muestra el mensaje {\bf MSG-28}  ``El proceso no se pudo realizar por un fallo de red.''

        E2: Cuando falta algún dato requerido entonces el sistema muestra el mensaje {\bf MSG-29}{``Los campos no están correctamente llenados.''}

        E3: Cuando el CURP del personal de seguridad ya está registrado en el sistema muestra el mensaje {\bf MSG-38}{``El CURP ya ha sido asociado a este personal de seguridad con anterioridad u otro personal de seguridad.''}

    }
    \UCitem{Tipo}{ Extiende de CU-32 Consultar lista de personal de seguridad}
    \UCitem{Observaciones}{Ninguna}

\end{UseCase}
%-------------------------------------- 

\begin{UCtrayectoria}

    \UCpaso[\UCactor] El Personal de gestión escolar accede a la pantalla \IUref{IU29}{ Dar de alta personal de seguridad}\label{CU33.introduceDatos} desde la pantalla \IUref{IU28}{Consultar lista de personal de seguridad} apretando el botón \IUbutton{Dar de alta personal de seguridad} e introduce los datos del personal de seguridad CURP, Turno, Cargo, Sexo y Nombre.
    \UCpaso[\UCactor] El Personal de gestión escolar oprime el botón \IUbutton{Dar de alta personal de seguridad}.
    \UCpaso El sistema revisa que los datos del personal de seguridad sean válidos.
    \UCpaso El sistema verifica que el CURP no haya sido registrado con anterioridad.
    \UCpaso[\UCactor] El Personal de gestión escolar es redirigido a la pantalla \IUref{IU28}{Consultar lista de personal de seguridad }.

\end{UCtrayectoria}
