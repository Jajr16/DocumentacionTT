% !TeX root = ../ejemplo.tex

%--------------------------------------
\begin{UseCase}{CU-24}{Consultar lista de periodo de ETS }{

    Permitir al Personal de gestión escolar consultar la lista de los periodos de ETS.
}
    \UCitem{Versión}{\color{Gray}1}
    \UCitem{Autor}{\color{Gray}Huertas Ramírez Daniel Martín}
    \UCitem{Supervisa}{\color{Gray}De la cruz De la cruz Alejandra.}
    \UCitem{Actor}{Personal de gestión escolar}
    \UCitem{Propósito}{Mostrar todos los periodos de ETS dados de alta con su respectiva información.}
    \UCitem{Entradas}{Ninguna}
    \UCitem{Origen}{Mouse}
    \UCitem{Salidas}{Ninguna}
    \UCitem{Destino}{Ninguna}
    \UCitem{Precondiciones}{ El Personal de gestión escolar debe de haber iniciado sesión}
    \UCitem{Postcondiciones}{Ninguna.}
    \UCitem{Errores}{
        E2: Cuando no hay ningún periodo de ETS dado de alta se muestra el mensaje {\bf MSG-31}  ``Ningún periodo de ETS ha sido dado de alta.''
    }
    \UCitem{Tipo}{ Extiende de CU42 Iniciar sesión de personal escolar web}
    \UCitem{Observaciones}{Ninguna}

\end{UseCase}
%-------------------------------------- 

\begin{UCtrayectoria}
    \UCpaso[\UCactor] El Personal de gestión escolar accede a la pantalla \IUref{IU24}{Consultar lista de periodo de ETS}\label{CU24.introduceDatos} a través del menú lateral que tiene en todas las pantallas apretando el botón \IUbutton{Consultar lista de periodo de ETS}.
    \UCpaso El sistema muestra la información de todos los periodos de ETS.
    \UCpaso[\UCactor] El Personal de gestión escolar revisa los periodos de ETS.
\end{UCtrayectoria}
