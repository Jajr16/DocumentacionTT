%-------------------------------------- COMIENZA descripción del caso de uso.

\label{CU-19}
\begin{UseCase}{CU-19}{Probar reconocimiento facial}{
		Permitir que el alumno pruebe la funcionalidad de reconocimiento facial para asegurar que su rostro sea reconocido correctamente por el sistema.
	}
	\UCitem{Versión}{\color{Gray}2.0}
	\UCitem{Autor}{\color{Gray}De la cruz De la cruz Alejandra}
	\UCitem{Supervisa}{\color{Gray}Huertas Ramírez Daniel Martín}
	\UCitem{Actor}{\hyperlink{Alumno}{Alumno}}
	\UCitem{Propósito}{Permitir al alumno verificar que el sistema de reconocimiento facial funciona correctamente con su rostro.}
	\UCitem{Entradas}{Selección del botón \IUbutton{Probar reconocimiento facial} en la \IUref{IUE03}{Pantalla de saludo del alumno}.}
	\UCitem{Origen}{Interacción del alumno con la \IUref{IUE03}{Pantalla de saludo del alumno}.}
	\UCitem{Salidas}{Retroalimentación visual en la \IUref{IU19}{Pantalla Reconocimiento facial alumno} indicando la precisión del reconocimiento y una imagen del rostro capturado (si aplica). Mensaje de error si la cámara no se activa o falla el reconocimiento.}
	\UCitem{Destino}{\IUref{IU19}{Pantalla Reconocimiento facial alumno}.}
	\UCitem{Precondiciones}{El alumno ha iniciado sesión en la aplicación móvil (\textbf{\hyperref[CU-02]{CU-02 Iniciar sesión del alumno}}).}
	\UCitem{Postcondiciones}{El alumno ha visualizado el resultado de la prueba de reconocimiento facial.}
	\UCitem{Errores}{
		
			E1: Cuando no se puede capturar la fotografía del alumno se muestra el mensaje \textbf{ ´´Error al capturar la fotografía.´´}
			
			E2: Cuando hay un fallo en el reconocimiento facial se muestra el mensaje \textbf{ ´´Error al realizar el reconocimiento facial.´´}
			 
			E3: Cuando falla la conexión se muestra el mensaje \textbf{ ´´Error de conexión.´´}
		
	}
	\UCitem{Tipo}{Caso de uso primario}
	\UCitem{Observaciones}{La precisión del reconocimiento facial se mostrará al alumno.}
\end{UseCase}
%--------------------------------------
\begin{UCtrayectoria}
	\UCpaso[\UCactor] Selecciona el botón \IUbutton{Probar reconocimiento facial } desde la pantalla \IUref{IUE03}{Pantalla de saludo del alumno}.
	\UCpaso El sistema activa la cámara del dispositivo y muestra la \IUref{IU19}{Pantalla Reconocimiento facial alumno} con una vista previa de la cámara. \Trayref{A}, \Trayref{B}, \Trayref{C}
	\UCpaso[\UCactor] El alumno se posiciona frente a la cámara y presiona el botón \IUbutton{Probar}.
	\UCpaso El sistema captura la imagen del rostro del alumno y la envía para el reconocimiento facial.
	\UCpaso El sistema recibe el resultado del reconocimiento facial (precisión).
	\UCpaso El sistema muestra en la \IUref{IU19}{Pantalla Reconocimiento facial alumno} la precisión del reconocimiento:
	\begin{itemize}
		\item Si la precisión es alta (>= 80\%), se muestra un mensaje indicando que el reconocimiento fue exitoso y la precisión. 
		\item Si la precisión es media (>= 60\% y < 80\%), se muestra un mensaje indicando que la identidad podría ser dudosa y la precisión.
		\item Si la precisión es baja (< 60\%), se muestra un mensaje indicando que no se encontró una coincidencia y la precisión.
	\end{itemize}
	\UCpaso[\UCactor] El alumno revisa el resultado.
\end{UCtrayectoria}
%--------------------------------------
\begin{UCtrayectoriaA}{A}{Error al capturar la fotografía}
	\UCpaso El sistema intenta activar la cámara y capturar la fotografía, pero ocurre un error.
	\UCpaso El sistema muestra un mensaje de error: \textbf{``Error al capturar la fotografía: [detalle del error]''.}
	\UCpaso[\UCactor] El alumno puede intentar nuevamente.
	\UCpaso Fin de la trayectoria alternativa.
\end{UCtrayectoriaA}
%--------------------------------------
\begin{UCtrayectoriaA}{B}{Error al realizar el reconocimiento facial}
	\UCpaso El sistema captura la fotografía, pero ocurre un error al realizar el reconocimiento facial.
	\UCpaso El sistema muestra un mensaje de error: \textbf{ ``Error al realizar el reconocimiento facial: [detalle del error]''.}
	\UCpaso[\UCactor] El alumno puede intentar nuevamente.
	\UCpaso Fin de la trayectoria alternativa.
\end{UCtrayectoriaA}
%--------------------------------------
\begin{UCtrayectoriaA}{C}{Error de conexión}
	\UCpaso Ocurre un error al intentar conectar con el servidor para el reconocimiento facial.
	\UCpaso El sistema muestra un mensaje de error: \textbf{ ``Error de conexión.''}
	\UCpaso[\UCactor] El alumno verifica su conexión a internet o intenta nuevamente.
	\UCpaso Fin de la trayectoria alternativa.
\end{UCtrayectoriaA}
%--------------------------------------
\begin{UCtrayectoriaA}{C1}{Reconocimiento Facial Exitoso (>= 80\%)}
	\UCpaso El sistema muestra: ``Es casi seguro que el alumno es quien dice ser. Precisión del reconocimiento facial: [precisión]\%''
	\UCpaso Fin de la trayectoria alternativa.
\end{UCtrayectoriaA}
%--------------------------------------
\begin{UCtrayectoriaA}{C2}{Identidad Dudosa (>= 60\% y < 80\%)}
	\UCpaso El sistema muestra: ``Es dudosa la identidad del alumno. la precisión del reconocimiento facial: [precisión]\%''
	\UCpaso Fin de la trayectoria alternativa.
\end{UCtrayectoriaA}
%--------------------------------------
\begin{UCtrayectoriaA}{C3}{No Coincidencia (< 60\%)}
	\UCpaso El sistema muestra: ``El casi seguro que el alumno no es quien dice ser. Precisión del reconocimiento facial: menor al 60\%''
	\UCpaso Fin de la trayectoria alternativa.
\end{UCtrayectoriaA}
%--------------------------------------
\begin{UCtrayectoriaA}{C4}{Error en el Reconocimiento Facial (Detallado)}
	\UCpaso El sistema realiza el reconocimiento facial pero ocurre un error específico.
	\UCpaso El sistema muestra: \textbf{ ``Error al realizar el reconocimiento facial: [detalle del error]''.}
	\UCpaso Fin de la trayectoria alternativa.
\end{UCtrayectoriaA}
%-------------------------------------- TERMINA descripción del caso de uso.

