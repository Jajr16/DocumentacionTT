% \IUref{IUAdmPS}{Administrar Planta de Selección}
% \IUref{IUModPS}{Modificar Planta de Selección}
% \IUref{IUEliPS}{Eliminar Planta de Selección}

%-------------------------------------- COMIENZA descripción del caso de uso.

%\begin{UseCase}[archivo de imágen]{UCX}{Nombre del Caso de uso}{
%--------------------------------------
% !TeX root = ../ejemplo2.tex
\label{CU-06}
\begin{UseCase}{CU-06}{Mostrar información de los ETS asignados}{
		Este caso de uso permite al docente visualizar la información detallada de los ETS que tiene asignados.
	}
	\UCitem{Versión}{\color{Gray}2.0}
	\UCitem{Autor}{\color{Gray}De la cruz De la cruz Alejandra}
	\UCitem{Supervisa}{\color{Gray}Huertas Ramírez Daniel Martín}
	\UCitem{Actor}{\hyperlink{tDocente}{Docente}}
	\UCitem{Propósito}{Permite al docente visualizar la información detallada de cada ETS que tiene asignado.}
	\UCitem{Entradas}{Selección de un ETS.}
	\UCitem{Origen}{\IUref{IU05}{Pantalla de Consultar ETS}}
	\UCitem{Salidas}{Detalle de ETS asignado: Tipo de ETS, Unidad de Aprendizaje, Periodo, Fecha, Turno, Cupo, Duración, Salones (con Tipo de Salón por cada salón), Hora. O indicación de error.}
	\UCitem{Destino}{\IUref{IU06}{Pantalla Información de ETS}}
	\UCitem{Precondiciones}{El docente debe haber iniciado sesión (\textbf{\hyperref[CU-01]{CU-01 Iniciar sesión del sistema móvil}}) y haber consultado la lista de ETS asignados (\textbf{\hyperref[CU-05]{CU-05 Consultar ETS asignados}}).}
	\UCitem{Postcondiciones}{El docente ha visualizado la información detallada del ETS seleccionado.}
	\UCitem{Errores}{
	
			E1: Cuando no exista detalles del ETS se muestra el mensaje\textbf {``Ocurrió un error al desplegar los detalles del ETS.''}.
			
			E2: Cuando no se pueda acceder a la base de datos se muestra el mensaje \textbf { ``Error al consultar la base de datos. Intente nuevamente más tarde.''}.
		
	}
	\UCitem{Tipo}{Caso de uso primario}
	\UCitem{Observaciones}{Si el docente es el aplicador del ETS, visualizará un botón llamado "Solicitar remplazo" que lo lleva a la \IUref{IU07}{Pantalla Solicitar remplazo}. También visualiza un botón llamado "Ir a la lista de alumnos" que lo lleva a la \IUref{IU08}{Pantalla Lista de alumnos inscritos a un ETS}.}
\end{UseCase}
%--------------------------------------
\begin{UCtrayectoria}
	\UCpaso[\UCactor] El docente selecciona el ETS que desea visualizar desde la \IUref{IU05}{Pantalla de Consultar ETS}.
	\UCpaso[\UCsist] El sistema recupera la información detallada del ETS seleccionado. \Trayref{A}, \Trayref{B}
	\UCpaso El sistema despliega la información detallada del ETS en la \IUref{IU06}{Pantalla Información de ETS}, incluyendo: Tipo de ETS, Unidad de Aprendizaje, Periodo, Fecha, Turno, Cupo, Duración, Salones (con Tipo de Salón por cada salón) y Hora.
	\UCpaso[\UCactor] El docente visualiza la información detallada del ETS.
	\UCpaso Si hay alumnos inscritos en el ETS, el docente visualiza un botón \IUbutton{Ir a la lista de alumnos}.
	\UCpaso[\UCactor] Si el docente presiona el botón \IUbutton{Ir a la lista de alumnos}, el sistema lo redirige a la \IUref{IU08}{Pantalla Lista de asistencia de ETS}.
	\UCpaso Si el docente es el aplicador del ETS, visualiza un botón \IUbutton{Solicitar remplazo}.
	\UCpaso[\UCactor] Si el docente presiona el botón \IUbutton{Solicitar remplazo}, el sistema lo redirige a la \IUref{IU07}{Pantalla Solicitar remplazo}.
\end{UCtrayectoria}
%--------------------------------------
\begin{UCtrayectoriaA}{A}{No hay detalles disponibles para el ETS seleccionado}
	\UCpaso El sistema muestra el mensaje: \textbf{``Ocurrió un error al desplegar los detalles del ETS.''}
	\UCpaso[\UCactor] El docente presiona el botón \IUbutton{Regresar} para volver a la lista de ETS (\IUref{IU05}{Pantalla de Consultar ETS}).
	\UCpaso Fin de la trayectoria alternativa.
\end{UCtrayectoriaA}

%--------------------------------------
\begin{UCtrayectoriaA}{B}{Error en la conexión con la base de datos}
	\UCpaso El sistema intenta recuperar la información detallada del ETS.
	\UCpaso Ocurre un error en la conexión con la base de datos.
	\UCpaso El sistema muestra el mensaje de error: \textbf{``Error al consultar la base de datos. Intente nuevamente más tarde.''}.
	\UCpaso[\UCactor] El docente presiona el botón \IUbutton{Aceptar} para cerrar el mensaje.
	\UCpaso[\UCactor] El docente puede intentar la consulta nuevamente o presionar el botón \IUbutton{Regresar} para volver a la pantalla anterior (\IUref{IU05}{Pantalla de Consultar ETS}).
	\UCpaso Fin de la trayectoria alternativa.
\end{UCtrayectoriaA}

%-------------------------------------- TERMINA descripción del caso de uso.

