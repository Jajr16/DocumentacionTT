% !TeX root = ../ejemplo.tex

%--------------------------------------
\begin{UseCase}{CU-32}{Consultar lista de personal de seguridad}{

    Permitir al Personal de gestión escolar consultar la lista con la información de las personas que esta registradas como personal de seguridad.
}
    \UCitem{Versión}{\color{Gray}1}
    \UCitem{Autor}{\color{Gray}Huertas Ramírez Daniel Martín}
    \UCitem{Supervisa}{\color{Gray}De la cruz De la cruz Alejandra.}
    \UCitem{Actor}{Personal de gestión escolar}
    \UCitem{Propósito}{Mostrar una lista con la información de las personas que están registradas como personal de seguridad.}
    \UCitem{Entradas}{Ninguna}
    \UCitem{Origen}{Teclado}
    \UCitem{Salidas}{Ninguna}
    \UCitem{Destino}{Pantalla \IUref{IU28}{Consultar lista de personal de seguridad}}
    \UCitem{Precondiciones}{ El Personal de gestión escolar debe de haber iniciado sesión}
    \UCitem{Postcondiciones}{Ninguna.}
    \UCitem{Errores}{

        E1: Cuando no hay ningún usuario personal de seguridad dado de alta se muestra el mensaje {\bf MSG-29}  ``No hay personal de seguridad dado de alta.''
        
    }
    \UCitem{Tipo}{ Extiende de CU42 Iniciar sesión de personal escolar web}
    \UCitem{Observaciones}{Ninguna}

\end{UseCase}
%-------------------------------------- 

\begin{UCtrayectoria}
\UCpaso[\UCactor] El Personal de gestión escolar accede a la pantalla \IUref{IU28}{ Consultar lista de personal de seguridad }\label{CU32.introduceDatos} desde la pantalla \IUref{IUE05}{de saludo de personal de gestión escolar } apretando el botón \IUbutton{Consultar lista de personal de seguridad}.
\UCpaso El sistema muestra la información de todo el personal de seguridad.
\UCpaso[\UCactor] El Personal de gestión escolar revisa la información del personal de seguridad dado de alta.
\end{UCtrayectoria}


