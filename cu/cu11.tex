% !TeX root = ../ejemplo.tex

%--------------------------------------
\label{CU-11}
\begin{UseCase}{CU-11}{Mostrar la foto e información del alumno}{
		Permitir que los docentes puedan revisar el reporte de asistencia del alumno, donde podrán ver información detallada y elementos multimedia asociados al registro de asistencia.
	}
	\UCitem{Versión}{\color{Gray}2.0}
	\UCitem{Autor}{\color{Gray}Huertas Ramírez Daniel Martín}
	\UCitem{Supervisa}{\color{Gray}De la cruz De la cruz alejandra.}
	\UCitem{Actor}{\hyperlink{tDocente}{Docente}}
	\UCitem{Propósito}{Permitir a los docentes revisar los detalles de la asistencia de un alumno específico a un ETS.}
	\UCitem{Entradas}{Selección de un alumno desde la \IUref{IU08}{Lista de asistencia de ETS}.}
	\UCitem{Origen}{\IUref{IU08}{Lista de asistencia de ETS}}
	\UCitem{Salidas}{Información detallada del alumno y su reporte de asistencia, incluyendo (si aplica): foto de la credencial, foto del reconocimiento facial (con precisión), boleta, nombre completo, CURP, carrera, unidad de aprendizaje del ETS, periodo del ETS, turno del ETS, materia del ETS, tipo de ETS, fecha de ingreso, hora de ingreso, nombre del docente aplicador, razón del reporte, motivo del rechazo. Mensajes indicando la ausencia de reporte o de imágenes.}
	\UCitem{Destino}{\IUref{IU20}{Pantalla mostrar la foto e información del alumno}}
	\UCitem{Precondiciones}{El docente debe haber iniciado sesión (\textbf{\hyperref[CU-01]{CU-01 Iniciar sesión del sistema móvil}}) y haber consultado la lista de alumnos inscritos al ETS (\textbf{\hyperref[CU-08]{CU-08 Consultar lista de alumnos inscritos a un ETS}}).}
	\UCitem{Postcondiciones}{El docente ha visualizado la información detallada del alumno seleccionado y su reporte de asistencia.}
	\UCitem{Errores}{
		E1: Cuando se pierde la conexión durante el proceso, los procesos se cancelan y se muestra el mensaje \textbf{\hyperref[msg:CU11-E1]{MSG-CU11-E1}} ``El proceso no se pudo realizar por un fallo de red.''
	}
	\UCitem{Tipo}{Caso de uso primario}
	\UCitem{Observaciones}{En caso de que las imágenes (foto de credencial o reconocimiento facial) no se carguen, se mostrará el texto Sin Imagen en su lugar.}
\end{UseCase}
%--------------------------------------
\begin{UCtrayectoria}
	\UCpaso[\UCactor] El docente selecciona un alumno de la lista en la \IUref{IU08}{Lista de asistencia de ETS}.
	\UCpaso El sistema recupera la información del alumno y su reporte de asistencia. \Trayref{A}, \Trayref{B}
	\UCpaso El sistema muestra en la \IUref{IU20}{Pantalla mostrar la foto e información del alumno}: boleta, nombre completo, CURP, carrera, unidad de aprendizaje del ETS, periodo del ETS, turno del ETS, materia del ETS, tipo de ETS, fecha de ingreso, hora de ingreso, nombre del docente aplicador y razón del reporte.
	\UCpaso Si el reporte es un rechazo, se muestra el motivo del rechazo.
	\UCpaso Si existe una foto de la credencial, se muestra. Si no, se muestra "Sin Imagen".
	\UCpaso Si hubo verificación por reconocimiento facial, se muestra la foto del reconocimiento facial y la precisión. Si no, no se muestra esta información.
	\UCpaso[\UCactor] El docente revisa la información del alumno.
\end{UCtrayectoria}

%--------------------------------------
\begin{UCtrayectoriaA}{A}{No se ha creado reporte para este alumno (dentro del periodo)}
	\UCpaso El sistema verifica que no existe un reporte de asistencia para el alumno seleccionado y que la hora actual no ha superado las 2 horas posteriores al inicio del ETS.
	\UCpaso El sistema muestra el mensaje: \textbf{\hyperref[msg:CU11-A1]{MSG-CU11-A1}} ``No se ha creado reporte para este alumno.''
	\UCpaso[\UCactor] El docente visualiza el mensaje.
	\UCpaso Fin de la trayectoria alternativa.
\end{UCtrayectoriaA}

%--------------------------------------
\begin{UCtrayectoriaA}{B}{El alumno no se presentó al ETS (fuera del periodo)}
	\UCpaso El sistema verifica que no existe un reporte de asistencia para el alumno seleccionado y que la hora actual ha superado las 2 horas posteriores al inicio del ETS.
	\UCpaso El sistema muestra el mensaje: \textbf{\hyperref[msg:CU11-B1]{MSG-CU11-B1}} ``El alumno no se presentó al ETS.''
	\UCpaso[\UCactor] El docente visualiza el mensaje.
	\UCpaso Fin de la trayectoria alternativa.
\end{UCtrayectoriaA}
%-------------------------------------- 




