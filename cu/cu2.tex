% !TeX root = ../ejemplo.tex

%--------------------------------------
\label{CU-02}
\begin{UseCase}{CU-02}{Consultar calendario escolar}{
		Permitir que los usuarios vean el calendario escolar y puedan solicitar al sistema que les mencione cuántos días faltan para que inicie el próximo periodo de ETS.
	}
	\UCitem{Versión}{\color{Gray}1}
	\UCitem{Autor}{\color{Gray}Huertas Ramírez Daniel Martín}
	\UCitem{Supervisa}{\color{Gray}De la Cruz de la Cruz Alejandra.}
	\UCitem{Actor}{\hyperlink{tUsuario}{Usuario}}
	\UCitem{Propósito}{Que el usuario consulte cuántos días faltan para el próximo periodo de ETS.}
	\UCitem{Entradas}{Solicitud de consulta.}
	\UCitem{Origen}{Pantalla táctil}
	\UCitem{Salidas}{Mensaje indicando los días faltantes para el periodo de ETS o si ya es periodo de ETS.}
	\UCitem{Destino}{Pantalla de la aplicación móvil.}
	\UCitem{Precondiciones}{El usuario debe haber iniciado sesión.}
	\UCitem{Postcondiciones}{El usuario recibe información sobre la fecha del próximo periodo de ETS.}
	\UCitem{Errores}{
		E1: Cuando no se ha registrado el siguiente periodo de ETS, el sistema muestra el mensaje \textbf{``Aún no está registrado el siguiente periodo de ETS.''}.
		
		E2: Cuando se pierde la conexión durante el proceso, los procesos se cancelan y se muestra el mensaje \textbf{ ``Conexión perdida.''}
	}
	\UCitem{Tipo}{Caso de uso primario}
	\UCitem{Observaciones}{Ninguna}
\end{UseCase}
%--------------------------------------

\begin{UCtrayectoria}
	\UCpaso[\UCactor] El usuario accede a la pantalla \IUref{IU02}{Pantalla Consultar calendario escolar} para la app móvil mediante el botón con forma de calendario visible en cualquier pantalla excepto la de inicio de sesión\label{CU02.accedePantalla}.
	\UCpaso[\UCactor] El usuario solicita conocer cuántos días faltan para que el periodo de ETS inicie, oprimiendo el botón \IUbutton{Calcular cuántos días faltan para el periodo de ETS}.
	\UCpaso El sistema recupera la fecha de inicio del próximo periodo de ETS registrada en el sistema.
	\UCpaso El sistema calcula la diferencia en días entre la fecha actual y la fecha de inicio del próximo periodo de ETS.
	\UCpaso Si la fecha de inicio del periodo de ETS es posterior a la fecha actual, el sistema muestra el mensaje: \textbf{ ``Faltan (cantidad de días) días para el periodo de ETS.''}
	\UCpaso Si la fecha de inicio del periodo de ETS es igual o anterior a la fecha actual, el sistema muestra el mensaje: \textbf{ ``Actualmente es periodo de ETS.''}
	\UCpaso [\UCactor] El usuario visualiza la cantidad de días que faltan para el periodo del ETS o si ya es el periodo de ETS.
\end{UCtrayectoria}







