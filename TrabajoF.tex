
\chapter{Trabajo futuro}

En este capítulo se mencionará lo que se tiene planeado realizar para el siguiente periodo de TT en ella se describen las actividades que se debe realizar para el funcionamiento del sistema, además se mencionan algunas mejoras que se deben realizar

Para la implementación del backend, se utilizará Spring Boot como framework principal, permitiendo establecer una comunicación eficiente entre Kotlin y la base de datos. Se aprovecharán las diversas herramientas que ofrece Spring Boot, para facilitar la interacción con la base de datos y gestionar las operaciones. 
Ahora bien, para la implementación de la base de datos se utilizará PostgreSQL como sistema gestor de base de datos.

Por otro lado, para el desarrollo de la parte móvil se planea desarrollar una serie de puntos:
\begin{itemize}
    
    \item Implementar canales seguros DE HTTP para el intercambio de datos sensibles. 
    \item Optimizacion de métodos de lista y repositorio para mejorar el rendimiento en consultas. 
    \item Implementación de un módulo para que Kotlin envíe imágenes directamente a la BD. 
    \item Programar las pantallas del sistema. 
    \item un módulo de notificación qué permitirá a los usuarios mantener a los usuarios informados sobre las modificaciones que puedan surgir en los ETS. 
    \item Implementar un esquema de encriptacion de datos para la protección de datos.
\end{itemize}

Finalmente, con los resultados obtenidos en esta primera parte del trabajo terminal se buscará mejorar el modelo de extracción de características para la generación de embeddings que permitan realizar la identificación de usuarios siguiendo un enfoque de verificación, esto permitirá la creación de una base de datos vectorial persistente para finalmente implementar el proceso de inferencia/reconocimiento en tiempo real.