\section{Caso de prueba CP042 - Asignar docente de reemplazo}

\begin{longtable}{|p{5cm}|p{10cm}|}
	\hline
	\multicolumn{2}{|c|}{\textbf{CASO DE PRUEBA - CP042 Asignar docente de reemplazo}} \\
	\hline
	\textbf{ID del Caso de Prueba} & CP042 - Asignar docente de reemplazo \\
	\hline
	\textbf{Descripción} & Validar que el jefe de departamento o el presidente de academia pueda visualizar las solicitudes de reemplazo y asignar a un docente, cambiando el estatus de la solicitud de "Pendiente" a "Aceptada" o "Rechazada". \\
	\hline
	\textbf{Supuestos y Condiciones Previas} & 
	\begin{itemize}
		\item El jefe de departamento o presidente de academia ha iniciado sesión correctamente.
		\item Existen solicitudes de reemplazo previamente enviadas por docentes.
		\item Se dispone de una lista de docentes disponibles para asignación.
	\end{itemize} \\
	\hline
	\textbf{Datos de Prueba} & 
	\begin{itemize}
		\item Solicitud de reemplazo con estatus “Pendiente”.
		\item Docente disponible para asignación.
	\end{itemize} \\
	\hline
	\textbf{Pasos a Ejecutar} & 
	\begin{enumerate}
		\item Iniciar sesión como jefe de departamento o presidente de academia.
		\item Seleccionar la opción “Solicitudes de reemplazo” desde el menú.
		\item Visualizar las solicitudes pendientes.
		\item Seleccionar una solicitud.
		\item Asignar a un docente como reemplazo.
		\item Enviar la solicitud con la decisión correspondiente (Aceptar o Rechazar).
	\end{enumerate} \\
	\hline
	\textbf{Resultado Esperado} & 
	La solicitud debe actualizarse correctamente con el docente asignado y su estatus debe cambiar a “Aceptada” o “Rechazada”, según la decisión. \\
	\hline
	\textbf{Errores Detectados} & 
	\begin{itemize}
		\item Las solicitudes de reemplazo no se cargaban en la vista de jefe de departamento/presidente de academia.
		\item El estatus “Pendiente” no se actualizaba tras aceptar o rechazar una solicitud.
	\end{itemize} \\
	\hline
	\textbf{Resultado Real y Condiciones Posteriores} & 
	\begin{itemize}
		\item Se ajustó la lógica de obtención de solicitudes desde el backend, lo que permitió cargarlas correctamente en la interfaz.
		\item Se corrigió la lógica de cambio de estado, permitiendo que el estatus pase de “Pendiente” a “Aceptada” o “Rechazada” correctamente.
		\item Se verificó la actualización de los datos tanto en la base como en la interfaz.
	\end{itemize} \\
	\hline
	\textbf{Estado} & 
	\textbf{Aprobado} - Tras correcciones, la funcionalidad fue validada exitosamente. \\
	\hline
\end{longtable}
