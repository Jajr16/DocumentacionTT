\section{Caso de prueba CP009 - Tomar asistencias a los ETS}

\begin{longtable}{|p{5cm}|p{10cm}|}
	\hline
	\multicolumn{2}{|c|}{\textbf{CASO DE PRUEBA - CP009 Tomar asistencias a los ETS}} \\
	\hline
	\textbf{ID del Caso de Prueba} & CP009 - Tomar asistencias a los ETS \\
	\hline
	\textbf{Descripción} & Validar que el docente pueda crear reportes de asistencias e incidencias. \\
	\hline
	\textbf{Supuestos y Condiciones Previas} & 
	\begin{itemize}
		\item El docente ha iniciado sesión en el sistema.
		\item Existen ETS dados de alta.
		\item El docente eligió un ETS.
		\item Hay información del ETS seleccionado.
		\item Hay alumnos inscritos al ETS.
		
	\end{itemize} \\
	\hline
	\textbf{Datos de Prueba} & 
	\begin{itemize}
		\item razón, motivo y si se realiza reconocimiento facial; foto y precisión. 
	\end{itemize} \\
	\hline
	\textbf{Pasos a Ejecutar} & 
	\begin{enumerate}
		\item Iniciar sesión como docente.
		\item Seleccionar la opción ´´ETS´´ desde el menú.
		\item Elegir un ETS disponible en la lista.
		\item Dar clic en el ETS.
		\item Revisa los datos mostrados.
		\item El docente elije la opcion ´´Ir a la lista de alumnos´´
		\item El docente revisa la lista de alumnos inscritos al ETS
		\item El docente elige un alumno para crear un reporte de asistencia/incidencia.
	\end{enumerate} \\
	\hline
	\textbf{Resultado Esperado} & 
	Se crea el reporte correctamente. \\	
	\hline
	\textbf{Errores Detectados} & 
	\begin{itemize}
		\item Al tomar la foto para el reconocimiento facial en algunos dispositivos salia volteada y no dejaba que se realizara la correctamente el reconocimiento facial.
		\item Al mandar la imagen al servidor de la red neuronal, si la imagen es muy grande el servidor corta la conexión. 
		\item La imagen se estira si no se re escala correctamente y altera la información para el reconocimiento facial.
	\end{itemize} \\
	\hline
	\textbf{Resultado Real y Condiciones Posteriores} & 
	\begin{itemize}
		\item Para el error de la imagen volteada, e aplico un método que detecta la orientación de la foto y la rota al estado normal tomando en cuenta 0 grados como el estado de orientación correcto y dependiendo de la orientación se voltea hasta llegar a los 0 grados. 
		\item Para el error del tamaño al enviar al servidor se aplico un método de re escalado a un tamaño de 640 x 480 pixeles, esto después de la corrección de la orientación.
		\item Para el error de que la foto se estire, se realizo una escala entre las dimensiones de la foto para mantener esa escala al re dimensionar la foto.
		\item Se verificó el correcto funcionamiento tanto en base de datos como en la interfaz.
	\end{itemize} \\
	\hline
	\textbf{Estado} & 
	\textbf{Aprobado} - la funcionalidad opera correctamente. \\
	\hline
\end{longtable}