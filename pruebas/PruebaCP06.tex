\section{Caso de prueba CP006 - Mostrar información de los ETS asignados}

\begin{longtable}{|p{5cm}|p{10cm}|}
	\hline
	\multicolumn{2}{|c|}{\textbf{CASO DE PRUEBA - CP006 Mostrar información de los ETS asignados}} \\
	\hline
	\textbf{ID del Caso de Prueba} & CP006 - Mostrar información de los ETS asignados \\
	\hline
	\textbf{Descripción} & Validar que el docente pueda revisar los datos específicos del ETS seleccionado. \\
	\hline
	\textbf{Supuestos y Condiciones Previas} & 
	\begin{itemize}
		\item El docente ha iniciado sesión en el sistema.
		\item Existen ETS dados de alta.
		\item El docente eligió un ETS.
		\item Hay información del ETS seleccionado.
	\end{itemize} \\
	\hline
	\textbf{Datos de Prueba} & 
	\begin{itemize}
		\item El docente se le muestra los detalles de ETS asignado: tipo de ETS, unidad de aprendizaje, periodo, fecha, turno, cupo, duración, salones (con tipo de salón por cada salón), Hora.
	\end{itemize} \\
	\hline
	\textbf{Pasos a Ejecutar} & 
	\begin{enumerate}
		\item Iniciar sesión como docente.
		\item Seleccionar la opción ´´ETS´´ desde el menú.
		\item Elegir un ETS disponible en la lista.
		\item Dar clic en el ETS.
		\item Revisa los datos mostrados.
	\end{enumerate} \\
	\hline
	\textbf{Resultado Esperado} & 
	Se le muestra al docente los datos del ETS seleccionado. \\
	\hline
	\textbf{Errores Detectados} & 
	\begin{itemize}
		\item En ocasiones al cambiar rápido entre ETS la aplicación se cerraba abruptamente.
		
	\end{itemize} \\
	\hline
	\textbf{Resultado Real y Condiciones Posteriores} & 
	\begin{itemize}
		\item Se modifico el orden de obtención y asignación de los datos, para que nos aseguráramos que los datos fueran  obtenidos antes ser asignados a las variables (evitar asignación de null y/o datos vacíos).
		\item Se verificó el correcto funcionamiento tanto en base de datos como en la interfaz.
	\end{itemize} \\
	\hline
	\textbf{Estado} & 
	\textbf{Aprobado} - El error fue corregido y la funcionalidad opera correctamente. \\
	\hline
\end{longtable}