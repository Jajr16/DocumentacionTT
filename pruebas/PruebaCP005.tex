\section{Caso de prueba CP005 - Consultar ETS asignados}

\begin{longtable}{|p{5cm}|p{10cm}|}
	\hline
	\multicolumn{2}{|c|}{\textbf{CASO DE PRUEBA - CP005 Consultar ETS asignados}} \\
	\hline
	\textbf{ID del Caso de Prueba} & CP005 - Consultar ETS asignados \\
	\hline
	\textbf{Descripción} & Validar que el docente pueda consultar todos lo ETS y los ETS que tiene que aplicar . \\
	\hline
	\textbf{Supuestos y Condiciones Previas} & 
	\begin{itemize}
		\item El docente ha iniciado sesión en el sistema.
		\item Existen ETS asignados al docente.
		\item Existen ETS dados de alta.
	\end{itemize} \\
	\hline
	\textbf{Datos de Prueba} & 
	\begin{itemize}
		\item nombre de usuario del docente.
		\item Lista de ETS dados de alta.
	\end{itemize} \\
	\hline
	\textbf{Pasos a Ejecutar} & 
	\begin{enumerate}
		\item Iniciar sesión como docente.
		\item Seleccionar la opción ´´ETS´´ desde el menú.
	\end{enumerate} \\
	\hline
	\textbf{Resultado Esperado} & 
	Los ETS que están dados de alta se mostraran en pantalla. \\
	\hline
	\textbf{Errores Detectados} & 
	\begin{itemize}
		\item Para el filtro de ETS, mas específicamente el de "Mis ETS" no e mostraba ningún ETS.

	\end{itemize} \\
	\hline
	\textbf{Resultado Real y Condiciones Posteriores} & 
	\begin{itemize}
		\item Se modifico la lógica del envió de la solicitud de obtención los ETS para que devolvieran 2 listas en vez de una, una con todos los ETS sin ningun filtro y otra filtrando con el nombre del usuario del docente para que solo se mostraran los ETS donde el docente es el aplicador.
		\item Se verificó el correcto funcionamiento tanto en base de datos como en la interfaz.
	\end{itemize} \\
	\hline
	\textbf{Estado} & 
	\textbf{Aprobado} - El error fue corregido y la funcionalidad opera correctamente. \\
	\hline
\end{longtable}