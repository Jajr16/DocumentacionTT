\section{Caso de prueba CP11 - Mostrar la foto e información del alumno}

\begin{longtable}{|p{5cm}|p{10cm}|}
	\hline
	\multicolumn{2}{|c|}{\textbf{CASO DE PRUEBA - CP11 Mostrar la foto e información del alumno}} \\
	\hline
	\textbf{Caso de uso relacionado} & \hyperref[CU-11]{CU-11 Mostrar la foto e información del alumno}\\
	\hline
	\textbf{Descripción} & Validar que el docente revisar los reportes creados por el o por otros docentes. \\
	\hline
	\textbf{Supuestos y Condiciones Previas} & 
	\begin{itemize}
		\item El docente ha iniciado sesión en el sistema.
		\item Existen ETS dados de alta.
		\item El docente eligió un ETS.
		\item Hay información del ETS seleccionado.
		\item Hay alumnos inscritos al ETS.
		
	\end{itemize} \\
	\hline
	\textbf{Datos de Prueba} & 
	\begin{itemize}
		\item Reporte. 
	\end{itemize} \\
	\hline
	\textbf{Pasos a Ejecutar} & 
	\begin{enumerate}
		\item Iniciar sesión como docente.
		\item Seleccionar la opción ´´ETS´´ desde el menú.
		\item Elegir un ETS disponible en la lista.
		\item Dar clic en el ETS.
		\item Revisa los datos mostrados.
		\item El docente elije la opción ´´Ir a la lista de alumnos´´
		\item El docente revisa la lista de alumnos inscritos al ETS
		\item El docente elige el reporte de un alumno.
	\end{enumerate} \\
	\hline
	\textbf{Resultado Esperado} & 
	El docente revisa el reporte. \\	
	\hline
	\textbf{Errores Detectados} & 
	\begin{itemize}
		\item La el dato de la hora y fecha de la creación del reporte salia mal por un error de tipos de datos.

	\end{itemize} \\
	\hline
	\textbf{Resultado Real y Condiciones Posteriores} & 
	\begin{itemize}
		\item Antes de mostrar la hora y la fecha son convertidos a un tipo de dato String.
		\item Se verificó el correcto funcionamiento tanto en base de datos como en la interfaz.
	\end{itemize} \\
	\hline
	\textbf{Estado} & 
	\textbf{Aprobado} - la funcionalidad opera correctamente. \\
	\hline
\end{longtable}