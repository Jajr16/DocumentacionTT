\section{Caso de prueba CP08 - Consultar lista de alumnos inscritos a un ETS}

\begin{longtable}{|p{5cm}|p{10cm}|}
	\hline
	\multicolumn{2}{|c|}{\textbf{CASO DE PRUEBA - CP08 Consultar lista de alumnos inscritos a un ETS}} \\
	\hline
	\textbf{Caso de uso relacionado} & \hyperref[CU-08]{CU-08 Consultar lista de alumnos inscritos a un ETS} \\
	\hline
	\textbf{Descripción} & Validar que el docente pueda revisar la lista de alumnos inscritos al ETS seleccionado. \\
	\hline
	\textbf{Supuestos y Condiciones Previas} & 
	\begin{itemize}
		\item El docente ha iniciado sesión en el sistema.
		\item Existen ETS dados de alta.
		\item El docente eligió un ETS.
		\item Hay información del ETS seleccionado.
		\item Hay alumnos inscritos al ETS.
		
	\end{itemize} \\
	\hline
	\textbf{Datos de Prueba} & 
	\begin{itemize}
		\item La lista de alumnos inscritos al ETS.
	\end{itemize} \\
	\hline
	\textbf{Pasos a Ejecutar} & 
	\begin{enumerate}
		\item Iniciar sesión como docente.
		\item Seleccionar la opción ´´ETS´´ desde el menú.
		\item Elegir un ETS disponible en la lista.
		\item Dar clic en el ETS.
		\item Revisa los datos mostrados.
		\item El docente elije la opcion ´´Ir a la lista de alumnos´´
		\item El docente revisa la lista de alumnos inscritos al ETS
	\end{enumerate} \\
	\hline
	\textbf{Resultado Esperado} & 
	Se le muestra al docente la lista de alumnos inscritos al ETS. \\	
	\hline
	\textbf{Errores Detectados} & 
	\begin{itemize}
		\item Ninguno.
		
	\end{itemize} \\
	\hline
	\textbf{Resultado Real y Condiciones Posteriores} & 
	\begin{itemize}

		\item Se verificó el correcto funcionamiento tanto en base de datos como en la interfaz.
	\end{itemize} \\
	\hline
	\textbf{Estado} & 
	\textbf{Aprobado} - la funcionalidad opera correctamente. \\
	\hline
\end{longtable}