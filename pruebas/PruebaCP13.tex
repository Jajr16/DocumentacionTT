\section{Caso de prueba CP13 - Consultar alumnos mediante búsqueda por boleta}

\begin{longtable}{|p{5cm}|p{10cm}|}
	\hline
	\multicolumn{2}{|c|}{\textbf{CASO DE PRUEBA - CP13 Consultar alumnos mediante búsqueda por boleta}} \\
	\hline
	\textbf{Caso de uso relacionado} & \hyperref[CU-13]{CU-13 Consultar alumnos mediante búsqueda por boleta} \\
	\hline
	\textbf{Descripción} & Validar la funcionalidad de consultar alumnos inscritos a un ETS, mediante búsqueda por número de boleta. \\
	\hline
	\textbf{Supuestos y Condiciones Previas} & 
	\begin{itemize}
		\item El personal de seguridad ha iniciado sesión correctamente.
		\item Existen alumnos inscritos a un ETS para la fecha consultada.
	\end{itemize} \\
	\hline
	\textbf{Datos de Prueba} & 
	\begin{itemize}
		\item Las boletas de los alumnos deberan ser válidas y deben estar inscritos en al menos un ETS.
		\item La fecha de consulta del ETS debe estar activa.
	\end{itemize} \\
	\hline
	\textbf{Pasos a Ejecutar} & 
	\begin{enumerate}
		\item Iniciar sesión como personal de seguridad.
		\item Seleccionar la opción " Consultar Alumnos" desde eñ menú.
		\item Ingresar el número de boleta en el buscador.
		\item Ejecutar la búsqueda y observar la lista de alumnos inscritos a ETS.
	\end{enumerate} \\
	\hline
	\textbf{Resultado Esperado} & 
	Los alumnos inscritos a un ETS en la fecha consultada y con la boleta ingresada deben mostrarse correctamente en la pantalla. \\
	\hline
	\textbf{Errores Detectados} &
	\begin{itemize}
		\item Inicialmente, no se mostraban alumnos al seleccionar "Consultar Alumnos" desde el menú.
		\item La lista de alumnos inscritos a ETS ese día aparecía vacía.
		\item Se verificó que la lógica de implementación no estaba retornando datos correctamente.
	\end{itemize} \\
	\hline
	\textbf{Resultado Real y Condiciones Posteriores} & 
	\begin{itemize}
		\item Tras corrección, el sistema mostró correctamente la lista de alumnos inscritos al ETS filtrada por boleta.
		\item Se validó que la consulta a la base de datos y el filtrado funcionaron correctamente.
	\end{itemize} \\
	\hline
	\textbf{Estado} & 
	\textbf{Aprobado} - El error fue corregido y la funcionalidad se encuentra funcionando. \\
	\hline
\end{longtable}